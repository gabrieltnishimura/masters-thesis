% ------------------------------------------------------------------------
% ------------------------------------------------------------------------
% Tese de Mestrado da Escola Politécnica da USP
% Integrantes: Gabriel Takaoka Nishimura
% ------------------------------------------------------------------------
% ------------------------------------------------------------------------

\documentclass[
	12pt,				% tamanho da fonte
	openright,			% capítulos começam em pág ímpar
	oneside,			% para impressão em recto e verso. Oposto a 
	a4paper,			% tamanho do papel.
	hyphens,			% url longa nas referencias
	english,			% idioma adicional para hifenização
	english				% o último idioma é o principal do documento
]{abntex2}
% ---
%  PACOTES
% ---

% Pacotes básicos
\usepackage{lmodern}			% Usa a fonte Latin Modern
\usepackage[T1]{fontenc}		% Selecao de codigos de fonte.
\usepackage[utf8]{inputenc}		% Conversão automática dos acentos
\usepackage{lastpage}			% Usado pela Ficha catalográfica
\usepackage{indentfirst}		% Indenta o primeiro parágrafo de cada seção.
\usepackage{color}				% Controle das cores
\usepackage{amssymb}			% Less than or equal to
\usepackage{subcaption}			% Subfigure
\usepackage{float}				% Gráficos
\usepackage{microtype} 			% para melhorias de justificação
\usepackage{listings}			% para listagem de código %
\usepackage{amsmath}			% para construção de sistemas lineares%
\usepackage{graphicx}
\usepackage{pgfplots}
\usepackage{lscape}
\usepackage{longtable}
\usepackage{multirow}


\graphicspath{ {images/} }

% Pacotes de citações
\usepackage[brazilian,hyperpageref]{backref}	 % Paginas com as citações
\usepackage[alf,abnt-url-package=url]{abntex2cite}	% Citações padrão ABNT

% ---
% CONFIGURAÇÕES DE PACOTES
% ---

% ---
% Configurações do pacote
\newenvironment{spmatrix}[1]{
	\def\mysubscript{#1}\mathop\bgroup\begin{pmatrix}
	}
{\end{pmatrix}\egroup_{\textstyle\mathstrut\mysubscript}}
% ---

% ---
% Configurações do pacote float
\newfloat{chart}{tbhp}{loc} %section
\floatname{chart}{Charts}
\floatstyle{plaintop}
\restylefloat{chart}
\newcommand{\listofcharts}{\listof{chart}{Charts List}}
% ---
% Configurações do pacote backref
\renewcommand{\backrefpagesname}{Citation(s) on page(s):~}
\renewcommand{\backref}{}
\renewcommand*{\backrefalt}[4]{
	\ifcase #1 %
	No citation.%
	\or
	Cited on page #2.%
	\else
	Cited #1 times in pages #2.%
	\fi}%

% ---
% ---
% Informações de dados para CAPA e FOLHA DE ROSTO
% ---
\titulo{Fale Alguma Coisa - a citizen science collected brazilian portuguese speech corpus}
\autor{Gabriel Takaoka Nishimura}
\local{São Paulo}
\data{2021}
\orientador{Prof. Dr. Bruno Carvalho de Albertini}
\instituicao{São Paulo University -  USP \par
			Polytechnic School \par
			Computer Engineering with emphasys on Computer}
\tipotrabalho{Mestrado}
\preambulo{Research Proposal presented to the Polytechnic School of the University of São Paulo}
% ---

% ---
% Configurações de aparência do PDF final
\definecolor{blue}{RGB}{41,5,195} % cor azul

% informações do PDF
\makeatletter
\hypersetup{
	%pagebackref=true,
	pdftitle={\@title},
	pdfauthor={\@author},
	pdfsubject={\imprimirpreambulo},
	pdfcreator={LaTeX with abnTeX2},
	pdfkeywords={abnt}{latex}{abntex}{abntex2}{trabalho acadêmico},
	colorlinks=true,       		% false: boxed links; true: colored links
	linkcolor=blue,          	% color of internal links
	citecolor=blue,        		% color of links to bibliography
	filecolor=magenta,      	% color of file links
	urlcolor=blue,
	bookmarksdepth=4
}
\makeatother
% ---

% ---
% Espaçamentos entre linhas e parágrafos
% ---
\setlength{\parindent}{1.3cm} % O tamanho do parágrafo
\setlength{\parskip}{0.2cm}  % espacamento entre um paragrafo e outro

% ---
% compila o indice
% ---
\makeindex
% ---

% ----
% Início do documento
% ----
\begin{document}

	% Seleciona o idioma do documento
	\selectlanguage{english}

	% Retira espaço extra obsoleto entre as frases.
	\frenchspacing

	% ----------------------------------------------------------
	% ELEMENTOS PRÉ-TEXTUAIS
	% ----------------------------------------------------------
	\pretextual

	% ---
	% Capa
	% ---
	\imprimircapa
	% ---

	% ---
	% Folha de rosto
	% ---
	\imprimirfolhaderosto*
	% ---

	% ---
	% RESUMOS
	% ---
	% ---
% Arquivo com o resumo da Tese de Mestrado do aluno
% Gabriel Takaoka Nishimura da Escola Politécnica da Universidade de São Paulo
% ---

\setlength{\absparsep}{18pt}
\begin{resumo}
    The research and advancement of scientific knowledge can be very daunting, especially without a scientific background. To 
    Citizen science is a powerful tool to help scientists creating new scientific knowledge
    
    acquire general public contributions. It helps boost 
    
	\vspace{\onelineskip}
	\noindent 
	
	\textbf{Key-words}: citizen science, gamification, speech corpus, natural language processing
\end{resumo}

	% ---

	% ---
	% inserir lista de ilustrações
	% ---
	%\pdfbookmark[0]{\listfigurename}{lof}
	%\listoffigures*
	%\cleardoublepage
	% ---

	% ---
	% inserir lista de gráficos
	% ---
	%\pdfbookmark[0]{\listtablename}{lot}
	%\listofcharts
	%\cleardoublepage
	% ---

	% ---
	% inserir lista de tabelas
	% ---
	%\pdfbookmark[0]{\listtablename}{lot}
	%\listoftables*
	%\cleardoublepage
	% ---

	% ---
	% inserir lista de abreviaturas e siglas
	% ---
	
	% ---

	% ---
	% inserir o sumario
	% ---
	\pdfbookmark[0]{\contentsname}{toc}
	\tableofcontents*
	\cleardoublepage
	% ---

	% ----------------------------------------------------------
	% ELEMENTOS TEXTUAIS
	% ----------------------------------------------------------
	\textual

	% ----------------------------------------------------------
	% Introdução
	% ----------------------------------------------------------
	\chapter*[Introduction]{Introduction}
\label{chap:introduction}
\addcontentsline{toc}{chapter}{Introduction}

The research and advancement of scientific knowledge can be very daunting, especially without a scientific background. Requirements such as a well-defined methodology, high data quality, and a strict revision process may present some barriers to the common citizen. However, with the recent advancement in technology~\cite{newman2012future}, connectivity~\cite{newman2012future}, and interest in science by the public~\cite{silvertown2009new}, there has been a rising trend in the general population contributing to science without a formal scientific background, what is being called Citizen Science \cite{mckinley2017citizen}.

Citizen science refers to research that engages nonprofessionals in the process of creating new scientific knowledge \cite{bonney2014next}. Referred to as citizen scientists, these nonprofessionals may participate in a variety of tasks of varying complexity, from simple tasks such as data gathering or classification \cite{barker2013pascal}, to even complex ones such as solving algorithms \cite{cooper2010predicting}. They may act as contributors and collaborators, but can also have a more proactive role as a project leader \cite{robinson2018ten}.

Citizen science is not new, as it has formally existed for more than one century now. However, its reach was reduced, since it was mostly focused on ecological and environmental sciences. Some initiatives date back to 1890, such as the Cooperative Weather Service, where amateurs send collected weather data to the National Weather Service, and to 1966 by the North American Breeding Bird Survey with more than 670 publications references, where nonprofessionals map avian species distribution throughout North America over time \cite{kosmala2016assessing}. This practice has shown (1) a constant influx of citizen science projects over the years \footnote{Extraction from \url{citizenscience.gov}, and Zooniverse.} (figure \ref{fig:growth-citizen-science-projects}) as well as (2) a steady rise in articles published on the topic \footnote{Query String: "citizen science" in Scholar Google, filtered by one year intervals} (as seen in figure \ref{fig:growth-publications}).

\pgfplotstableread[row sep=\\,col sep=&]{
    year & publications \\
    2012 & 3670 \\
    2013 & 5100 \\
    2014 & 6360 \\
    2015 & 9360 \\
    2016 & 12100 \\
    2017 & 14800 \\
    2018 & 16100 \\
    2019 & 21300 \\
    2020 & 17800 \\
    2021 & 29000 \\
    2022 & 17900 \\
}\publicationdata

\pgfplotstableread[row sep=\\,col sep=&]{
    year & initiatives \\
    2000- & 38 \\
    2013- & 137 \\
    2014 & 44 \\
    2015 & 40 \\
    2016 & 173 \\
    2017 & 18 \\
    2018 & 16 \\
    2019 & 14 \\
    2020 & 7 \\
    2021 & 7 \\
    2022 & 1 \\
}\citizensciencegovextraction

\pgfplotstableread[row sep=\\,col sep=&]{
    year & initiatives \\
    2000- & 0 \\
    2013- & 15 \\
    2014 & 7 \\
    2015 & 16 \\
    2016 & 23 \\
    2017 & 45 \\
    2018 & 51 \\
    2019 & 79 \\
    2020 & 72 \\
    2021 & 68 \\
    2022 & 44 \\
}\zooniverseextraction

\begin{figure}[ht]
    \centering
    \caption{Growth of citizen science initiatives over time}
    \label{fig:growth-citizen-science-projects}
    \begin{tikzpicture}
        \begin{axis}[
                ybar,
                bar width=.5cm,
                width=.9\textwidth,
                height=.35\textwidth,
                % legend style={at={(0.8,0.9)}, anchor=north,legend columns=1,legend rows=2,legend cell align=left},
                legend style={at={(0.5,-0.25)}, anchor=north,legend columns=-1},
                symbolic x coords={2000-,2013-,2014,2015,2016,2017,2018,2019,2020,2021,2022},
                xtick=data,
                nodes near coords,
                nodes near coords align={vertical},
                ymin=0,ymax=200,
                ylabel={initiatives},
                xlabel={years},
            ]
            \addplot table[x=year,y=initiatives]{\citizensciencegovextraction};
            \addlegendentry{Citizen Science Gov}
            \addplot table[x=year,y=initiatives]{\zooniverseextraction};
            \addlegendentry{Zooniverse}
        \end{axis}
    \end{tikzpicture}
    \caption*{Source: Author}
\end{figure}

\begin{figure}[ht]
    \centering
    \caption{Growth of published articles on citizen science from 2012 to 2022}
    \label{fig:growth-publications}
    \begin{tikzpicture}
        \begin{axis}[
                ybar,
                bar width=.5cm,
                width=.9\textwidth,
                height=.35\textwidth,
                legend style={at={(0.5,-0.25)}, anchor=north,legend columns=-1},
                symbolic x coords={2012,2013,2014,2015,2016,2017,2018,2019,2020,2021,2022},
                xtick=data,
                nodes near coords,
                nodes near coords align={vertical},
                ymin=0,ymax=35000,
                ylabel={publications},
                xlabel={years},
            ]
            \addplot table[x=year,y=publications]{\publicationdata};
            \addlegendentry{Google Scholar}
        \end{axis}
    \end{tikzpicture}
    \caption*{Source: Author. Scholar Google search done in 11/13/2022 with query String: "citizen science", filtered by one year intervals (i.e.: 2012 filter is 2011\~2012)}
\end{figure}

This research field could be classified into three categories, based on volunteer involvement \cite{follett2015analysis}: (1) Contributory, where participants contribute to data collection and sometimes help analyze and disseminate results \cite{bonney2009citizen}, (2) Collaborative, where citizens also analyze samples, design the study, interpret the data, draw conclusions and disseminate results \cite{faridani2009networked} and (3) Co-created, where they participate in all stages of the project, including defining questions, developing the hypotheses, drawing conclusions, discussing results and answering new questions \cite{hill2012notes}.

As to why would a citizen collaborate, there could be several reasons depending on the project itself: contribution to the advancement of science or the project, desire to learn, personal interests, entertainment, among others \cite{tinati2016because}. This poses an engagement challenge, since public participation is vital to the result of the research, and is one of the studies of citizen science theory \cite{bowser2013using}.

At the other end of the spectrum, scientists lead citizen science initiatives. They enlist amateurs to contribute to projects, but also devise validation techniques from the collected data. They are also central to the role of acquiring sponsors in the government scenario, many of which are important to the feasibility of these initiatives.

With the rise in interest from the public, this initiative is able to achieve otherwise impossible results. According to \cite{theobald2015global}, in 2014, 1.3 million volunteers participated in 388 research projects related to biodiversity alone, contributing up to \$2.5 billion of in-kind labor annually.

This is just one of the many examples showing the potential of citizen science. It also comes with the realization that the public represents a free source of labor, skills, computational power, and even finance \cite{silvertown2009new}. Such perception could implicate in ethical concerns depending on how the data is gathered or processed, but also on how should the nonprofessional be (if they are) rewarded. Due to this and other concerns, the European Citizen Science Association conceived the "Ten principles of citizen science" \cite{robinson2018ten}. They establish some key principles to follow as good practice when applying citizen science concepts to research. Many of the principles focus on the protection of citizen scientist rights, since those individuals deserve feedback, acknowledgment, and participation rights to the research.

One significant factor supporting this growth is the availability of technical tools for disseminating information about projects and gathering data from the public \cite{silvertown2009new}. The widespread use of smartphones allows scientists to develop platforms easily accessible to everyone. This is combined with advancements in software usability on these platforms, resulting in a better first time usage as well as recurring contributions. In some research topics, low-cost hardware is stimulating robust data collection with less resources, providing unprecedented opportunities for knowledge generation \cite{buytaert2014citizen}. Online citizen science platforms (such as Zooniverse) provide an easy gateway to participation in a diverse set of projects, and have a high number of contributors \footnote{2,221,469 extracted from \url{https://www.zooniverse.org/} at 13/01/2021}.

Citizen participation in platforms like Zooniverse is the key to breakthroughs in scientific outcomes. Therefore, many studies focus on public engagement and community sustainability \cite{aristeidou2017profiles}. In these online communities, there is a high attrition rate recorded (\cite{nov2011technology} \cite{ponciano2015finding}), as well as dabbling behaviour \cite{eveleigh2014designing} in participation, even though it is recognized that user engagement is a necessary ingredient in the success of virtual environments \cite{verhagen2015benefitting}. Besides many of the reasons a citizen would collaborate, the look and feel of the project itself could change perspective and interpretation on how would a nonprofessional engage with the initiative.

One of the many techniques to overcome these engagement obstacles is throughout the use of gamification \cite{bowser2013using}. It refers to the addition of game elements to enhance user experience and engagement within non-game applications. Tasks are made to look more like games, employing scoring and competition, for instance. They are common in the private sector and have spread to education, health, government, and science. A recent report on consumer entertainment shows that, in the past six months, four of every five United States customers have played a video game (\autoref{fig:entertainment-report}). This illustrates the relevance of the segment. DFC Intelligence also released its 2020 Dossier with statements that nearly half of 3.1 billion gamers play games on their smartphones only. Gamers represent a significant portion of the population, so by gamifying them, citizen science projects scientists have a larger reach to this community.

\begin{figure}[ht]
    \centering
    \caption{2020 Report on Entertainment Category Engagement; 79\% of the population is taken by gamers.}
    \includegraphics[width=\linewidth]{images/game.png}
    \caption*{Source: 2020 Evolution of Entertainment / NPD Group}
    \label{fig:entertainment-report}
\end{figure}

Foldit, designed by researchers at the University of Washington, is a game in which gamers solve protein folding patterns, a central challenge in biochemistry, by virtually wiggling, shaking and pulling shapes to create small stable structures, as well as developing their own algorithms for solving protein folding \cite{bourzac2008enlisting}. This citizen science online game has had more than 800,000 registered players \cite{foldit2021players}, identified a potential target for HIV development, and redesigned a catalyst for the Diels-Alder reactions \cite{kreitmair2019citizen}. EyeWire is another example of the success of citizen science gamified environments. With more than 200,00 gamers from 145 countries, EyeWire allows users to map neurons in the retina, filling and extending areas missed by artificial intelligence \cite{kreitmair2019citizen}.

\section*{Computer Science}

Once focused on ecological and environmental sciences, the citizen science practice has a much larger range now, covering topics such as linguistics \cite{svendsen2018dynamics}, astronomy \cite{marshall2015ideas}, hydrology \cite{buytaert2014citizen} etc. Out of these interest areas, one that could see further development with initiatives using citizen science is Computer Science.

Citizen science can provide a solution for the demands of conducting data rich scientific research amongst 
professional scientists \cite{greenhill2014playing}. Some of these demands include time, material costs and labor incurred, particularly for tasks which are not suitable for analysis using computer algorithms \cite{silvertown2009new}. There lies an opportunity for Computer Science, where there are many data intensive requirements: data gathering and classification is one of the key actions of nonprofessionals, and also one of the key necessities for Computer Science data-driven approaches.

\section*{Natural Language Processing}

Natural Language Processing is a subfield of Computer Science, and one with increasing emphasis on data-driven approaches \cite{moore2003comparison}. Natural language processing explores how computers can understand and manipulate human language in text or speech format. Researches in this area include (but are not limited to) sentiment analysis, sentence prediction, text translation, text-to-speech conversion, and speech recognition.

Also known as automatic speech recognition (ASR), computer speech recognition, or speech-to-text (STT), speech recognition is a field that studies the recognition and translation of spoken language into text by computers. It has been an intensive research area for decades, but has seen growth led by increased demand on ASR systems in the mobile environment \cite{yu2016automatic}, with virtual assistants, such as Alexa, Siri, Google Now.

Speech recognition is a class of machine learning that has seen many different approaches over time: stochastically modelling with Hidden Markov Models  \cite{gales2008application}, artificial intelligence learning with Neural Networks \cite{graves2013speech}, non-stochastical modelling \cite{burget2003nonrandomattr} and even hybrid approaches \cite{wang2020transformer} have been made to solve this complex interdisciplinary field.

However, these voice modeling strategies are highly dependant on the quality and quantity of data provided. Factors such as noisy speech data, nonhomogeneous recordings, different microphones within the same dataset, and even speech disorders could limit proper analysis, affect accuracy, and even change speech predictions \cite{cooke2001robust}. 

\section*{Speech Corpus}

This type of voice dataset, also known as Speech Corpus (or Speech Corpora in plural), curate a collection of audio recordings of a spoken language. Some of them also have additional text files with transcriptions of the words spoken. Although they are widely found in the literature with robust recording procedures and analysis - such as TIMIT \cite{Lamel1992timmit}, DIRHA \cite{Ravanelli2016dirha} and the more recent \cite{chanchaochai2018globaltimit}) -, these datasets are structured in such a way that the speakers are individually selected. This could lead to bias problems \cite{bender2018data}, but also limits the number of voices recorded. Unfortunately, most speech corpora are recorded for the English language \cite{LeRouxVincent2014TRdatasets} and research is limited on the data quantity necessary to enable speech recognition systems and other voice applications.

To circumvent the quantity issue, some corpora are able to be constantly updated with user input. These online datasets can be accessed via web browser. Some examples are VoxForge and Common Voice \cite{ardila2019common}. However, the former has poor user interface and little usage over time, and the latter has no structured analysis on bias mitigation, relying on crowdsourced information.

This work identifies the Citizen Science practice as a potential candidate to create a robust Speech Corpus for the low-resource Brazilian Portuguese language. A gamifyied application will be constructed to collect the data and engage users, applying data validation techniques to mitigate bias problems afterwards. The curated data will be published in an open-source platform to further research on speech recognition in this low-resource language.

\section*{Objective}

The main focus of this project lies in the construction and validation of a speech corpus for the Brazilian Portuguese language while also applying concepts of citizen science and gamification. The validated corpus and anonymized data will be available to the public.

\section*{Method}

To achieve the aforementioned objective, this project will:

\begin{itemize}
    \item Research what characterizes a speech corpus in the literature;
    \item Construct an application to record anonymized speech data;
    \item Create a list of phrases which will be read by the public/the users;
    \item Add gamification elements to enhance public engagement;
    \item Validate recorded data and apply statistical methods to ensure data quality
    \item Publish dataset under open-source license
\end{itemize}

\section*{Document structure}

This document is divided in 7 chapters:

\begin{itemize}
    \item This chapter presents the context, motivation, and objective of the proposed research;
    \item Chapter \ref{chap:background} details the background and foundation related to this research, detailing concepts such as citizen science, gamification, natural language processing and speech corpora;
    \item Chapter \ref{chap:related-work} presents related work on corpus creation, as well as a systematic literature review, to ensure this research delivers a relevant contribution in the speech corpus research space;
    \item Chapter \ref{chap:proposal} contains the proposed work to create the corpus, listing all tasks in the process;
    \item Chapter \ref{chap:proposal-application} details the creation of the Fale Alguma Coisa ecossystem as well as the seletion of phrases sets delivered to contributors;
    \item Chapter \ref{chap:proposal-public-submission} explains the publication of the WebApp to a public URL;
    \item Chapter \ref{chap:proposal-data-analysis} delves into the details of validating the recorded data, generating a valid speech corpus;
    \item Lastly, chapter \ref{chap:proposal-data-publication} publicizes the previously created speech corpus under a open-source licence.
\end{itemize}

	% ----------------------------------------------------------

% 	% ----------------------------------------------------------
% 	% Metodologia
% 	% ----------------------------------------------------------
	\chapter{Background}

In this chapter, topics such as citizen science, gamification, and speech corpus will be further described. Additional terms relevant to the research will also be defined and clarified, to ensure all research topics are clear. 

\section{Citizen Science}

\subsection{Origins}

The citizen science term originates from two sources.

\subsection{Definition}

Citizen sc

\subsection{Classifications}

Below are some classifications in which citizen science projects can be divided:

\subsubsection{Volunteer Involvement}

An initial classification based on volunteer involvement \cite{follett2015analysis}: 
\begin{itemize}
    \item Contributory, where participants contribute to data collection and sometimes help analyze and disseminate results
    \item Collaborative, where citizens also analyze samples, design the study, interpret the data, draw conclusions and disseminate results
    \item Co-created, where they participate in all stages of the project, including defining questions, developing hypotheses, drawing conclusions, discussing results and answering new questions
\end{itemize}

\subsubsection{Goals of the study}

An alternative classification for these initiatives has been suggested by \cite{wiggins2011conservation}, and is based on the goals of the study:

\begin{itemize}
    \item Action projects, initiated by volunteers designed to encourage intervention in local concerns;
    \item Conservation projects, addressing natural resource management goals;
    \item Investigation projects, focusing on scientific research goals in a physical setting;
    \item Virtual projects, also focusing on scientific goals, but entirely based on information technology with all volunteer interaction occurring online;
    \item Education projects; often performed in the classroom or school grounds as part of the science curriculum.
\end{itemize}

\subsubsection{Topic of study}

An additional way of classifying citizen science projects is based on the topic of study, for example, astronomy, archaeology, and biology \cite{wiggins2011conservation}. 

\subsubsection{This work}

If these classifications are to be applied in this work, it should be categorized as a \textbf{contributory virtual speech corpus} citizen science project.

\subsection{Relevant Projects}

With the recent growth of citizen science, various breakthroughs were made possible. Below are some of the most relevant projects in the internet space.

\subsubsection{Foldit}

\begin{figure}[ht]
    \centering
    \includegraphics[width=\linewidth]{images/background/foldit-problem.png}
    \caption{Foldit - Unfolded (and unstable) Streptococcal Protein Puzzle \\ Source: \cite{foldit-protein-problem}}
    \label{fig:foldit-problem}
\end{figure}

Foldit, designed by researchers at the University of Washington, is a game in which gamers solve protein folding patterns, a central challenge in biochemistry, by virtually wiggling, shaking and pulling shapes to create small stable structures, as well as developing their own algorithms for solving protein folding \cite{bourzac2008enlisting}. 

\begin{figure}[ht]
    \centering
    \includegraphics[width=\linewidth]{images/background/foldit-solution.png}
    \caption{Foldit - Folded up Streptococcal Protein Puzzle \\ Source: \cite{foldit-protein-solution}}
    \label{fig:foldit-solution}
\end{figure}

As the moment of this article, Foldit has had 20 peer-reviewed articles in a number of journals and conferences \cite{foldit-publications}. Some relevant breakthroughs are a potential target for HIV drug development \cite{khatib2011crystal}, redesign of the catalyst for the Diels-Alder reaction \cite{eiben2012increased}, and improvement of cryo-electron microscopy atomic model building and refinement \cite{khatib2019building}.

\begin{table}[h]
    \centering
    \begin{tabular}{c|c}
        test & test \\
        Foldit & access
    \end{tabular}
    \caption{Caption}
    \label{tab:my_label}
\end{table}

\subsubsection{Zooniverse}

\begin{figure}[ht]
    \centering
    \includegraphics[width=\linewidth]{images/background/zooniverse.jpg}
    \caption{Zooniverse Platform, connecting volunteers with scientists \\ Source: \cite{zooniverse-logo}}
    \label{fig:foldit-solution}
\end{figure}

Zooniverse is a platform for citizen science projects. It connects more than a million volunteers around the world to assist professional researchers. The platform has a simple interface for input and classification of data, as well as the creation and management of projects.

This collaborative platform has enabled over 300 of scientific publications, with publications from the (1) discovery and classification of stars, planets, supernovas; humanities, animal identification, classification of whale calls, datasets etc.

\subsubsection{Galaxy Zoo}

https://www.zooniverse.org/projects/zookeeper/galaxy-zoo

\subsection{Ten Principles of Citizen Science}

According to the European Citizen Science Association, citizen science is a flexible concept which can be adapted and applied within diverse situations and disciplines. The association set out some key principles which as a community they believe underlie good practice in citizen science. Appendix \ref{app:ten-principles} lists all ten principles.

\section{Gamification}

\subsection{Crowdfunding}

\section{Natural Language Processing}

\subsection{Speech Recognition}

\section{Speech Corpus}

\subsection{}
	\chapter[Related Work]{Related Work}

In this section, we discuss how the literature has treated speech corpora creation, as well as the various conditions and variables considered in the process. Speech Corpus crafting itself is well established in the literature by TIMIT \cite{Lamel1992timmit} and SWITCHBOARD \cite{godfrey1992switchboard}. TIMIT creates a dataset of 6300 utterances by 630 speakers from different regions of the United States. The sentences were crafted to fit in one of the three categories: 1) dialect "shibboleth", 2) phonemically compact, and 3) phonetically diverse, but the selection itself was not well defined. Nevertheless, it is a very robust dataset with a time-aligned transcription and a usage guide to automatic speech recognition applications.

The CHiME articles \cite{christensen2010chime} \cite{barker2013pascal}, \cite{barker2018fifth} (and more), are also source of structured speech corpora creation, challenging researchers to better recognize speech within a everyday listening environment using multiple distant microphones. Since the focus of these works lies on nonoptimal recording conditions, detailed information on the noise background, noise level, recording style and speech material has been provided, as well as comprehensive postprocessing work.

A more recent work by \cite{chanchaochai2018globaltimit} attempts to extend the TIMIT functionality to other languages, by providing a method to create "TIMIT-like" datasets. These datasets are caracterized by having 1) Multiple (anonymously) identified speakers, 2) Wide range of phonetically representative inputs, 3) Wideband recordings with good acoustic quality, 4) Time-aligned lexical and phonemic transcripts and 5) Easily availability to anyone. The authors detail the speakers and sessions, the text corpus selection process, the recording procedures, as well as the transcription and alignment methods. At the moment, there have been five datasets created, with more planned or in progress.

As for crowdfunding speech corpora, VoxForge \cite{voxforge} is a viable alternative to contribute speech. Besides being a free speech corpus repository under GPL licence, VoxForge also incorporates an acoustic model archive for open source speech recognition engines. This corpus has over 132h, 39h, 20h, 57h of recorded English, French, Italian, and German \footnote{As of January 20th of 2021, per metric from \url{http://www.voxforge.org/en/Downloads}, \url{http://www.voxforge.org/fr/Downloads}, \url{http://www.voxforge.org/de/Downloads}, and \url{http://www.voxforge.org/de/Downloads}, respectively.}, as well a variety of acoustic models for CMU Sphinx, Julius and HTK. The VoxForge platform presents itself as an alternative to closed source speech recognition engines, since many corpora must be purchased under restrictive licenses. This contribution tool has, however, a very limited user interface and a not a well-defined recording validation process, which could invalidate the corpus robustness.

Common Voice \cite{ardila2019common} is a more recent platform for recording and listening to speech. It is a multilingual speech corpus, containing, as of 11th of December of 2020\footnote{Common Voice Corpus 6.1 accessible on  \url{https://commonvoice.mozilla.org/en/datasets}}, 1,686h of validated speech in English, and over 50h of validated Portuguese speech, in addition to other 27 languages. The main feature of this platform is the crowdsourced validation process, which all recordings must go through so that they are included in the validated corpus. All speech data is free and available under CC-0, the most permissive of licenses, and some recordings contain speaker demographics. One caveat though, is the lack of statistical analysis of the validated data, to ensure it is able to provide a quality dataset for speech recognition services. Nevertheless, the corpus was empirically tested through Mozilla’s DeepSpeech Speech-to-Text, achieving an average Character Error Rate improvement of 5.99 ± 5.48 for twelve target languages (German, French, Italian, Turkish, Catalan, Slovenian, Welsh, Irish, Breton, Tatar, Chuvash, and Kabyle).
	\chapter{Systematic Literature Review}
\label{chap:slr}

To identify the characteristics of a speech corpus, a systematic literature was performed \cite{kitchenham2009systematic}.

The search was executed on August 1st, 2019 and the electronic database used was Web of Science. Since corpora are also presented at conferences, all article types were considered in the search, not limited by publication date. The whole process to choose relevant papers is defined below:

\section{Search in papers database}

The first step in the process is searching for papers in the specified database. To allow reproducibility, the search query is presented in table \ref{tab:search-terms}. No criteria were applied in the search, therefore considering the title, abstract, keywords, and the entire paper for indexing. It is also noted that conference proceedings are also included in this search. In this step, 157 results were yielded. This result was further filtered to 23, which contained the terms in the title.

\begin{table}[h]
    \centering
    \begin{tabular}{|c|c|}
        \hline Digital Library & Search terms \\ \hline
        Web of Science & TOPIC: ("speech corpus" OR "speech corpora")  \\ \hline
    \end{tabular}
    \caption{Search terms used in the SLR}
    \label{tab:search-terms}
\end{table}

\section{Inclusions and exclusions}

To filter relevant results, we define the criteria to include or exclude the articles based on each abstract read:

\subsection{Inclusion Criteria}

\begin{itemize}
    \item Creates a speech corpus
    \item Define or discuss speech corpus creation
\end{itemize}

\subsection{Exclusion Criteria}

\begin{itemize}
    \item Full text not available in the electronic document
    \item Does not contain "speech corpus" or "speech corpora" strings in title, abstract, or keywords.
\end{itemize}

This step reduced the number of articles to 14.

\section{Article Analysis}

After applying the proper filtering as defined in the previous section, the remaining articles can be fully read to find the results. The whole filtering process is summarized in \ref{tab:filtering}.

\begin{table}[h]
    \centering
    \begin{tabular}{|c|c|}
        \hline Step & Results \\ \hline
        Initial search & 157 \\ \hline
        Title filtering & 23 \\ \hline
        Abstract reading & 14 \\ \hline
    \end{tabular}
    \caption{SLR - Filtering of results}
    \label{tab:filtering}
\end{table}

\section{Findings}

To analytically organize our findings, a content-analysis approach was used. To choose the categories in which the content analysis is applied, we adapted the work from \cite{queiroz2019blockchain}, resulting in table \ref{tab:content-analysis}. As for the corpus categories, we sought a comparison work from  \cite{LeRouxVincent2014TRdatasets}, in which a comprehensive table was provided and adapted to fit our categorization needs.

\begin{table}[h]
    \centering
    \begin{tabular}{|p{4cm}|p{11cm}|}
        \hline Category & Explanation/Example \\ \hline
        Total number of papers published & Number of publications by year \\ \hline
        Application area & Speech Corpus, Learning English Pronunciation, Automatic Speech Recognition, Crowdsourcing \\ \hline
        Application context & Speech Corpus Creation, Speech Corpus Automated Analysis, Dialect Analysis \\ \hline
        Corpus Language & English (Dialect), Arabic, Multilingual, etc \\ \hline
    \end{tabular}
    \caption{Categories for content analysis}
    \label{tab:content-analysis}
\end{table}

\begin{table}[h]
    \centering
    \begin{tabular}{|p{4cm}|p{11cm}|}
        \hline Category & Explanation/Example \\ \hline
        General attributes & scenario, total duration, sampling rate, number of distant or noisy microphones, demographics, transcription and time alignment and open access \\ \hline
        Speech attributes & duration of speech, speaking style, speakers in the room, speaker overlap\\ \hline
        Channel attributes & channel type, speaker location, speaker movements \\ \hline
        Noise attributes & stationary background noise, car noise, meeting noises, domestic noises, outdoor noises \\ \hline
        Available ground truth & reference speech signal, speaker location and orientation, paralinguistic attributes, noise events \\ \hline
    \end{tabular}
    \caption{Categories for speech corpus construction}
    \label{tab:speech-analysis}
\end{table}

\subsection{Publication by country and year}

Table \ref{tab:country-analysis} reports the number of papers, published by country and year, resulting from the research protocol application.

\begin{table}[h]
    \centering
    \begin{tabular}{|c|c|c|c|c|c|}
        \hline Country & 2005-2015 & 2016 & 2017 & 2018 & (\%) \\ \hline
        United States & - & 3 & 2 & - & 33\% \\ \hline 
        Japan & 3 & - & - & - & 20\% \\ \hline
        Emirates Arabs & - & - & 1 & 1 & 13\% \\ \hline
        India & - & - & 1 & - & 6.7\% \\ \hline
        Malasya & - & - & 1 & - & 6.7\% \\ \hline
        Sweden & - & - & 1 & - & 6.7\% \\ \hline
        China & - & 1 & - & - & 6.7\% \\ \hline
        Poland & 1 & - & - & - & 6.7\% \\ \hline
    \end{tabular}
    \caption{Countries for content analysis}
    \label{tab:country-analysis}
\end{table}

\subsection{Publication by type}

As for publication types, they have almost been equally divided: 46.7\% for articles and 53.3\% for proceedings papers.

\subsection{Speech Corpus studies' categorization}

In table \ref{tab:results-categorization}, each article is categorized according to table \ref{tab:content-analysis} definitions. The analysis is done afterwards in section \ref{sec:discussion}.

\begin{landscape}
\begin{longtable}{|p{3.5cm}|p{4.5cm}|p{9cm}|p{4.5cm}|}
\caption{Findings organized by speech corpus categorization as per table \ref{tab:content-analysis}}
\label{tab:results-categorization}
    \hline Authors & Application area & Context & Language \\ \hline
    \cite{almeman2018building} & Read Corpus & Multi-dialect Arabic corpora is scarce and not generated on mobile & Arabic \\ \hline
    \cite{dwivedi2017documenting} & Language Documentation & Revitalization of endangered languages through corpus creation & Kanauji of Kanpur \\ \hline
    \cite{bougrine2017altruistic} & Crowdsourcing & Crowdsourcing is a emerging and collaborative approach and can be effectively used to annotate linguistic resources & Arabic Algerian Dialects \\ \hline
    \cite{ng2017shefce} & Automatic syllable and phoneme detection & Pronunciation assessment studies in a bilingual context & Billingual (Cantonese, English) \\ \hline
    \cite{moore2017sheffield} & Read Speech & Goal oriented conversation & British English with a southern accent \\ \hline
    \cite{ramli2017first} & Storytelling Speech & Under-resourced language corpus creation for humanoid robot storyteller & Malay \\ \hline
    \cite{mansikkaniemi2017automatic} & Automatic speech alignment & Transcribed speech is a scarce and expensive resource & Finish \\ \hline
    \cite{goldman2016siwis} & Cross language speaker adaptation & Cross-lingual studies have no speech corpus from the same speakers & Bilingual and Trilingual from (English, French, German and Italian) \\ \hline
    \cite{liu2016sheffield} & Spontaneous Speech & Distant speech recognition with multi-channel speech corpus & English \\ \hline
    \cite{ruilan2016improving} & Read Corpus & Pronunciation characteristics in non-native English speakers & Dialectal English \\ \hline
    \cite{klessa2013paralingua} & Read Corpus & Paralinguistic features detection for forensics & Paralinguistic \\ \hline
    \cite{nagino2008building} & Corpus Enhancement & Low cost speech corpus creation with statistical multidimensional scaling method & Japanese \\ \hline
    \cite{zhang2008improved} & Phonetic words selection & Phonetically rich word selection from larger corpus & Chinese \\ \hline
    \cite{clopper2006nationwide} & Read Corpus & Large amount of speech by male and female from six dialect regions & Multiple-dialect English \\ \hline
\end{longtable}
\end{landscape}

\begin{landscape}
\begin{longtable}{|p{4cm}|p{3cm}|p{12cm}|p{3.5cm}|}
\caption{Speech corpora characteristics defined in table \ref{tab:speech-analysis}}
\label{tab:results-attributes}
\hline Category & Characteristic & Work(s) & Quantity (out of 14) \\ \hline
\multirow{7}{*}{General attributes} 
    & Scenario & 
    \cite{almeman2018building}, \cite{dwivedi2017documenting}, \cite{bougrine2017altruistic}, \cite{bougrine2017altruistic}, \cite{moore2017sheffield}, \cite{ramli2017first}, \cite{goldman2016siwis}, \cite{liu2016sheffield}, \cite{ruilan2016improving}, \cite{klessa2013paralingua}, \cite{nagino2008building},  \cite{clopper2006nationwide}, \cite{zhang2008improved} & 13
    \\ \cline{2-4} & Total duration &
    \cite{dwivedi2017documenting}, \cite{bougrine2017altruistic}, \cite{bougrine2017altruistic}, \cite{moore2017sheffield}, \cite{ramli2017first}, \cite{goldman2016siwis}, \cite{liu2016sheffield}, \cite{ruilan2016improving}, \cite{klessa2013paralingua}, \cite{nagino2008building}, \cite{clopper2006nationwide}, \cite{zhang2008improved} & 12
    \\ \cline{2-4} & Sampling Rate &
    \cite{almeman2018building}, \cite{dwivedi2017documenting}, \cite{bougrine2017altruistic}, \cite{ng2017shefce}, \cite{moore2017sheffield}, \cite{ramli2017first}, \cite{goldman2016siwis}, \cite{liu2016sheffield}, \cite{ruilan2016improving}, \cite{klessa2013paralingua}, \cite{clopper2006nationwide} & 11
    \\ \cline{2-4}& Distant or noisy microphones & 
    \cite{moore2017sheffield}, \cite{ramli2017first}, \cite{liu2016sheffield} & 3
    \\ \cline{2-4} & Demographics &
    \cite{dwivedi2017documenting}, \cite{bougrine2017altruistic}, \cite{ng2017shefce}, \cite{goldman2016siwis}, \cite{ruilan2016improving}, \cite{klessa2013paralingua}, \cite{clopper2006nationwide} & 7
    \\ \cline{2-4} & Transcription and time alignment & % manual?
    \cite{dwivedi2017documenting}, \cite{bougrine2017altruistic}, \cite{bougrine2017altruistic}, \cite{ng2017shefce}, \cite{moore2017sheffield}, \cite{ramli2017first}, \cite{goldman2016siwis}, \cite{liu2016sheffield}, \cite{ruilan2016improving}, \cite{klessa2013paralingua}, \cite{nagino2008building}, \cite{clopper2006nationwide} & 12
    \\ \cline{2-4} & Open access &
    \cite{almeman2018building}, \cite{moore2017sheffield}, \cite{liu2016sheffield} & 3
\\ \hline \multirow{3}{*}{Speech attributes}
    & Unique words & 
    \cite{bougrine2017altruistic}, \cite{bougrine2017altruistic}, \cite{ng2017shefce}, \cite{moore2017sheffield}, \cite{ramli2017first}, \cite{goldman2016siwis}, \cite{liu2016sheffield}, \cite{ruilan2016improving}, \cite{klessa2013paralingua}, \cite{nagino2008building}, \cite{clopper2006nationwide} & 11
    \\ \cline{2-4} & Speaking style &
    \cite{dwivedi2017documenting}, \cite{bougrine2017altruistic}, \cite{ng2017shefce}, \cite{moore2017sheffield}, \cite{ramli2017first}, \cite{goldman2016siwis}, \cite{nagino2008building}, \cite{clopper2006nationwide} & 8
    \\ \cline{2-4} & Number of speakers & 
    \cite{dwivedi2017documenting}, \cite{bougrine2017altruistic}, \cite{bougrine2017altruistic}, \cite{ng2017shefce}, \cite{moore2017sheffield}, \cite{ramli2017first}, \cite{goldman2016siwis},, \cite{ruilan2016improving} \cite{liu2016sheffield}, \cite{klessa2013paralingua}, \cite{nagino2008building}, \cite{clopper2006nationwide} & 14
\\ \hline \multirow{3}{*}{Channel attributes}
    & Channel type & 
    \cite{dwivedi2017documenting}, \cite{bougrine2017altruistic}, \cite{ng2017shefce}, \cite{moore2017sheffield}, \cite{ramli2017first}, \cite{goldman2016siwis}, \cite{liu2016sheffield}, \cite{ruilan2016improving}, \cite{nagino2008building}, \cite{clopper2006nationwide} & 10
    \\ \cline{2-4} & Speaker location & 
    \cite{dwivedi2017documenting}, \cite{ng2017shefce}, \cite{liu2016sheffield} & 3
    \\ \cline{2-4} & Speaker movements & 
    \cite{dwivedi2017documenting}, \cite{moore2017sheffield}, \cite{liu2016sheffield} & 3
\\ \hline \multirow{1}{*}{Noise attributes}
    & Noise type & N/A &
    0
\\ \hline \multirow{4}{*}{Available ground truth} 
    & Reference speech signal & 
    \cite{ng2017shefce}, \cite{moore2017sheffield}, \cite{ramli2017first}, \cite{liu2016sheffield} & 4
    \\ \cline{2-4} & Speaker location and orientation &
    \cite{bougrine2017altruistic}, \cite{moore2017sheffield}, \cite{ramli2017first}, \cite{liu2016sheffield} & 4
    \\ \cline{2-4} & Paralinguistic attributes &
    \cite{klessa2013paralingua} & 1
    \\ \cline{2-4} & Noise events & 
    \cite{ramli2017first} & 1
    \\ \hline
\end{longtable}
\end{landscape}

\section{Discussion}
\label{sec:discussion}

Each of our findings is discussed in the following subsections.

\subsection{Publications by country and year}

A simple analysis on table \ref{tab:country-analysis} infers that the research on speech corpus creation was stable until 2018, which had its last article published. This can be explained by the limited number of "Open Access" articles found and the small number of databases searched (Web Of Science only). A more permissive literature review should be able to better understand the conditions on which non-open articles are being released, as well as enhance coverage of the state-of-the-art in this field.

The table also categorizes the countries research quota. More developed countries such as United States and Japan have created more corpora over the years.

\subsection{Publications by type}

In the search executed, the remaining articles were almost equally separated by type. A great number of conference proceedings may indicate that the corpus creation process is not properly defined, as works vary in content and structure. Again, the main limitation of this work is the only database searched, therefore skewing analysis.

\subsection{Speech corpus categorization}

Table \ref{tab:content-analysis} contains the content-analysis for the works in the systematic literature review. Two of the columns - Application Area (1), and Language (2) - are be discussed below:

\subsubsection{Application Area}

Most of the corpora found in the articles were a Read Speech Corpus (\cite{almeman2018building}, \cite{ruilan2016improving}, \cite{klessa2013paralingua}, \cite{clopper2006nationwide}), revealing the lack of structured language documentation in the current research. This can be explained by the type of its content: a scripted content, with less bias from the speakers when compared to spontaneous corpora. Three of these works (\cite{ng2017shefce}, \cite{mansikkaniemi2017automatic} and \cite{nagino2008building}) worked on automating corpus creation by automatically aligning words, detecting syllables and phonemes, and even enhancing corpora, as currently most recording and alignment work is done manually.

\subsubsection{Languages}

There is a significant number of dialect and multilingual corpora research in the analyzed works. These variations of the same language - in different regions, or by different pronunciation by non-native speakers -, suggest that speech technology could become more personalized as computers understand less traditional utterances of the same language.

\subsection{Works by speech corpus characteristics}

Our findings condensed in table \ref{tab:results-categorization} illustrate the
current speech corpus creation characteristics used in the literature, each discussed below:

\subsubsection{Scenario}

Almost all corpora specified the scenario at which the recordings were executed. A meeting room (\cite{liu2016sheffield}, \cite{moore2017sheffield}); a anechoic booth (\cite{goldman2016siwis}); a quiet room inside a laboratory (\cite{ramli2017first}), etc. Although, each corpus is not limited to one scenario. For instance, \cite{almeman2018building} recorded in four different scenarios: inside the home, in a moving car, in a public place, and in a quiet place. Such characteristic is critical to corpus creation.

\subsubsection{Total duration}

12 out of the 14 studies analyzed described the total duration of the recordings. This illustrates the importance of being descriptive in the corpus.

\subsubsection{Sampling Rate}

The sampling rate is major to the corpus creation. It is cited in 11 out of the 14 works read. The reason it has so much importance lies on the corpus main usage: automatic speech recognition software. Knowing the frequency at which the recordings are done, the computer is able to better recognize subtleties in a more sampled audio, as well as simplify processing in a lower frequency sample rate.

\subsubsection{Noisy microphones}

Non-optimal recording conditions are normal. For instance, speaking on the phone in a public space generates a very high noisy background \cite{moore2017sheffield}). Moreover, speaking far from the microphone is also a common occurrence. The presence of works in this category emphasizes the need of noisy corpora to ensure automatic speech recognition software takes these factors into account.

\subsubsection{Demographics}

When recording accent (\cite{moore2017sheffield}), dialect (\cite{almeman2018building}) and pronunciation (\cite{ng2017shefce}), the definition of speaker demographics supports the understanding of the variations between the same language through research.

\subsubsection{Transcription and time alignment}

In the context of automatic speech recognition, a speech corpus containing  transcription and time alignment (additional to the recordings) enable computers and speech recognition specialists a more faithful analysis. Its importance can be verified by 85\% of the works evaluated.

%\subsubsection{Open access}

%Most corpora are closed to the public viewer. 

%\subsubsection{Unique words}
%\subsubsection{Speaking style}
\subsubsection{Number of speakers}

As well as the total duration, the number of speakers is a major characteristic of the corpus. More speakers lead to more representativeness, but also affect the corpus recording time. It is an important parameter and every paper analyzed describes it.

	\chapter[Proposal]{Proposal}
\label{chap:proposal}

This proposal focuses on detailing the necessary steps to the creation and publication of a crowdfunding speech corpus in Brazilian Portuguese. To this end, a virtual voice recording application will be developed (section \ref{sec:proposal-app}), focusing on adding gamification elements to enhance user engagement. Additionally, this application will also extract some relevant characteristics found in the systematic literature review in chapter \ref{chap:slr}, such as speaker demographics. Once the construction is finished, the application will be released, allowing general public submission (in section \ref{sec:proposal-public-submission}). After the submission period, the collected data will be analyzed and compiled to a speech corpus (\ref{sec:proposal-data-analysis}), which will be publicized to an open-source repository in section \ref{sec:proposal-data-publication}.

\section{Application}
\label{sec:proposal-app}

This section details the conception, documentation, and development process of the voice recording application, coined "Fale Alguma Coisa". Below, it will at times be referenced as "app", short for application, or "WebApp", short for Web Application. The main purpose of this app is to be able to record predetermined phrases from users. All other features support the engagement and usability through authentication, gamification, and explanatory elements. The following documentation structure is based on Pressman's book "Software Engineering: A Practitioner's Approach" \cite{pressman2014software}, and generated a proper Software Requirements Specification. Hence, some detail will be omitted to provide simplicity and overall understanding of the design process. Refer to appendix \ref{appendix:srs} to more details.

In this simplified section, the scope will be detailed \ref{sec:app-scope}, followed by the artefacts produced \ref{sec:app-artefacts}. Secondly, the functionality of the entire application will be documented using use cases (\ref{sec:simplified-use-cases}). Thirdly, the user interface will be elaborated by documenting the layouts (in \ref{sec:app-user-interface}) based on the use cases listed. One key factor in the app is the ability to teach science facts and trivia, which entails the need of proper phrase selection methods. These methods will be explained in the section \ref{sec:app-phrase-selection}. Lastly, the development process will be described, with additional details on the tools selected (\ref{sec:app-development}).

\begin{figure}[ht]
    \centering
    \caption{Fale Alguma Coisa app Logo}
     \includegraphics[width=\linewidth/2]{images/app/logo.jpg}
    \caption*{Source: Author}
    \label{fig:falealgumacoisa-logo}
\end{figure}

\subsection{Scope}
\label{sec:app-scope}

Towards contributing \textbf{nonprofessional scientists}, the Fale Alguma Coisa app should provide an easy gateway for the user to contribute his voice while having fun and learning various science facts and curiosities.

Towards researching \textbf{scientists}, the Fale Alguma Coisa app should provide a database of anonymized voice recordings, for scientists to extract and create speech corpus.

Directed towards anyone interested in learning and contributing to science.

\subsubsection{Outside scope}

However, some elements are outside of the scope of this system. Fale Alguma Coisa should \textbf{not}:
\begin{itemize}
    \item Allow for association of recording data and personal identification data (name, email);
    \item Support internationalization in the WebApp;
    \item Convert audio data into another format
    \item Support offline recording
\end{itemize}

\subsection{Artifacts}
\label{sec:app-artefacts}

This table presents the artefacts to be produced, and their respective location in this work.

\begin{table}[h]
\centering
\caption{Artefacts produced by the specification and their respective locations}
\label{tab:artifact-locations}
\begin{tabular}{|p{3.5cm}|p{2.5cm}|p{9cm}|}
    \hline 
    Artefact & Location & Description \\ \hline 
    Purpose & Appendix \ref{appendix:purpose} & Defines the purpose of the Software Requirements Specification documentation \\ \hline
    Scope & Appendix \ref{appendix:scope} & Artefacts, Objectives, Out of Scope \\ \hline
    System Overview & Appendix \ref{appendix:system-overview} & Project perspective, System Context, General Constraints, Assumptions and Dependencies \\ \hline
    Actor List & Appendix \ref{appendix:actor-list} & List of actors involved with the system \\ \hline 
    Simplified Requirements & Section \ref{sec-simplified-use-cases} & Simplified use cases \\ \hline
    Scenario-based Models & Appendix \ref{appendix:scenario-based-model} & Detailed use cases \\ \hline
    Class Models & Appendix \ref{appendix:domain-model} & Class diagrams to model the domain of the system, mapping relationship and collaboration \\ \hline
    Behavior Models & Appendix \ref{appendix:behavior-model} & State and sequence diagrams to fully document specific and more complex behavior \\ \hline
    Non-Function Requirements & Appendix \ref{appendix:non-fuctional-requirements} & i.e.: Usability, Performance, Security, Legal, Requirements, etc. \\ \hline
    Interface Requirements & Appendix \ref{appendix:interface-requirements} & Machine and External Systems interfaces \\ \hline
    Navigation Design & Appendix \ref{appendix:navigation-design} & Navigation Semantic Units \\ \hline
    User Interface Design & Appendix \ref{appendix:user-interface-design} & For each use case, a user interface will be developed and documented \\ \hline
    Application Architecture & Appendix \ref{appendix:architecture} & Before WebApp must have  \\ \hline
    Tools and Frameworks Selection & Appendix \ref{appendix:tools-selection} & To develop the documented application, a set of tools and frameworks shall be selected \\ \hline
    FaleAlgumaCoisa WebApp & Appendix \ref{appendix:webapp} & Description of developed web application \\ \hline
    FaleAlgumaCoisa Backend & Appendix \ref{appendix:backend} & Description of developed backend application \\ \hline
\end{tabular}
\caption*{Source: Author}
\end{table}

\subsection{Use Cases}
\label{sec:simplified-use-cases}

A simplified list of use cases is reproduced in tables \ref{tab:use-cases} below, separated by feature set. To a more complete specification, refer to section \ref{appendix:use-cases} in the Appendix.

\clearpage
\subsection{User Interface}
\label{sec:app-user-interface}

To ensure the development of the application is effective, an iterative design approach was taken. First, ideas for the app shaped a conceptual user journey. Second, this journey allowed the creation of layouts, which then passed through a validation process. Then, if aligned with the application concept, the layouts were developed. Otherwise, another design implementation occurred, undergoing further validation. The sections below detail the finalized artifacts.

\subsubsection{Color Scheme}

In color theory, colors are used to communicate meaning, but also affect mood, and perception \cite{agoston2013color}. The design color scheme defines a color palette to choose from when designing new visual elements. Applying the concept, a more colorful color scheme was chosen to lighten the mood of the application, as shown in figure \ref{fig:falealgumacoisa-color-scheme}.

\begin{figure}[ht]
    \centering
    \caption{Fale Alguma Coisa color scheme}
     \includegraphics[width=.8\linewidth]{images/app/colors.png}
    \caption*{Source: Author}
    \label{fig:falealgumacoisa-color-scheme}
\end{figure}

\clearpage
\subsubsection{Splash Screen}

As the user first enter the application, a splash screen will be shown to welcome him (mobile version in figure \ref{fig:falealgumacoisa-splash-page-design}). It contains the logo and an animation to draw the user's attention. After the animation, the home will be shown.

\begin{figure}[ht]
    \centering
    \caption{Fale Alguma Coisa Splash Page design}
    \includegraphics[width=\linewidth/2]{images/app/m-splash.jpg}
    \caption*{Source: Author}
    \label{fig:falealgumacoisa-splash-page-design}
\end{figure}

\subsubsection{Homepage}

After the splash animation, the homepage is shown. The mobile version can be seen below in figure \ref{fig:falealgumacoisa-home-page-design}. In this page, the call to action to start the recording is highlighted by the button at the center of the page, with text describing the project right below it. The login page is accessible through the link in the right upper corner. These few elements are placed to encourage the user to click on the recording, if he is a new user.

\begin{figure}[ht]
    \centering
    \caption{Fale Alguma Coisa Home Page design}
    \frame{\includegraphics[width=\linewidth/2]{images/app/m-home.jpg}}
    \caption*{Source: Author}
    \label{fig:falealgumacoisa-home-page-design}
\end{figure}

\subsubsection{Recording}

This page represents the core functionality of the website, allowing the user to record phrases with his voice. The recording is done through groups of phrases, called a theme, and is illustrated by the image \ref{fig:falealgumacoisa-recording-page-design} at the bottom. The main elements of the page are (1) the phrase highlighted in a rectangular box at the center of the page, and (2) the red recording button at the bottom.

\begin{figure}[ht]
    \centering
    \caption{Fale Alguma Coisa Recording Page designs}
    \begin{subfigure}{.5\textwidth}
      \centering
      \frame{\includegraphics[width=.9\linewidth]{images/app/recording/Journey_1.0.png}}
      \caption{Waiting to start recording}
      \label{fig:falealgumacoisa-recording-page-design-start}
    \end{subfigure}%
    \begin{subfigure}{.5\textwidth}
      \centering
      \frame{\includegraphics[width=.9\linewidth]{images/app/recording/Journey_1.2.png}}
      \caption{Confirm or delete recording}
      \label{fig:falealgumacoisa-recording-page-design-confirm}
    \end{subfigure}
    \caption*{Source: Author}
    \label{fig:falealgumacoisa-recording-page-design}
\end{figure}

\subsubsection{Terms of Service}

To comply with Brazil's General Data Protection Act (Law 13,709/2018), the application has to clarify data usage of the visitor of the website. The figure \ref{fig:falealgumacoisa-tos-page-design} defines the layout of the page with a simplified terms of service, as well as a button to accept or deny such terms.

\begin{figure}[ht]
    \centering
    \caption{Fale Alguma Coisa simplified terms of service page design}
    \frame{\includegraphics[width=\linewidth/2]{images/app/register/tos.png}}
    \caption*{Source: Author}
    \label{fig:falealgumacoisa-tos-page-design}
\end{figure}

\subsubsection{Dashboard}

When an unauthenticated user finishes recording its first theme, or when a user logs in, they are able to select from a list of themes to record. In this dashboard seen in figure \ref{fig:falealgumacoisa-dashboard-page-design}, they are also shown the number of points accumulated by the usage of the app, as well as able to open a menu and notification page.

\begin{figure}[ht]
    \centering
    \caption{Fale Alguma Coisa Dashboard Page design}
    \frame{\includegraphics[width=\linewidth/2]{images/app/dashboard/Dashboard.png}}
    \caption*{Source: Author}
    \label{fig:falealgumacoisa-dashboard-page-design}
\end{figure}

\subsubsection{Leaderboard}

The leaderboard layouts feature two visualizations, depending on the desired scope. Should the contributor want to see all top ranking users, the leaderboard (figure \ref{fig:falealgumacoisa-leaderboard-page-design-general}) is available. If the user only wants to check his friend rankings, the application displays a reduced list (figure \ref{fig:falealgumacoisa-leaderboard-page-design-friend}), based on the friends added to the platform. These layouts provide a way for users to compare their contributions, thus promoting competition. A social element is also included throughout the option to add friends.

\begin{figure}[ht]
    \centering
    \caption{Fale Alguma Coisa Leaderboard Page designs}
    \begin{subfigure}{.5\textwidth}
      \centering
      \frame{\includegraphics[width=.9\linewidth]{images/app/leaderboard/GeneralRanking.png}}
      \caption{General Leaderboard}
      \label{fig:falealgumacoisa-leaderboard-page-design-general}
    \end{subfigure}%
    \begin{subfigure}{.5\textwidth}
      \centering
      \frame{\includegraphics[width=.9\linewidth]{images/app/leaderboard/FriendsRanking.png}}
      \caption{Friends Leaderboard}
      \label{fig:falealgumacoisa-leaderboard-page-design-friend}
    \end{subfigure}
    \caption*{Source: Author}
    \label{fig:falealgumacoisa-leaderboard-page-design}
\end{figure}

\subsubsection{Friends}

The ability to add and track (follow) friends is possible with the layouts described below. If he knows the name of the user, the contributor can search for his friends using the Search Friends page (seen in figure \ref{fig:falealgumacoisa-friends-page-design-search}). The results return in the same page below the search input field (figure \ref{fig:falealgumacoisa-friends-page-design-results}), as a list of users. To add a friend, it is as simple as clicking the "Seguir" button in the right corner of the result list. This click toggles the button with a "Seguindo" text. If this new state is clicked, the friend is removed.

\begin{figure}[ht]
    \centering
    \caption{Fale Alguma Coisa Friends Page designs}
    \begin{subfigure}{.5\textwidth}
      \centering
      \frame{\includegraphics[width=.9\linewidth]{images/app/friends/FriendsSearch_1.0.png}}
      \caption{Search for friends}
      \label{fig:falealgumacoisa-friends-page-design-search}
    \end{subfigure}%
    \begin{subfigure}{.5\textwidth}
      \centering
      \frame{\includegraphics[width=.9\linewidth]{images/app/friends/FriendsSearch_1.3.png}}
      \caption{Search friends results}
      \label{fig:falealgumacoisa-friends-page-design-results}
    \end{subfigure}
    \caption*{Source: Author}
    \label{fig:falealgumacoisa-friends-page-design}
\end{figure}

\subsubsection{Refer Friends}

To earn more contribution points, the citizen scientist can also refer new friends using the Refer friends page. The mobile version of this page is illustrated in figure \ref{fig:falealgumacoisa-refer-page-design}, and has one button "Compartilhar", which enables the mobile sharing functionality. The link shared redirects to a registration page, and both users get 100 points after the registration of the reffered user.

\begin{figure}[ht]
    \centering
    \caption{Fale Alguma Coisa Refer Friends Page design}
    \frame{\includegraphics[width=\linewidth/2]{images/app/friends/Search_Friend_1.0.png}}
    \caption*{Source: Author}
    \label{fig:falealgumacoisa-refer-page-design}
\end{figure}

\subsubsection{Login and Registration}

The login and registration pages include essential features to the application: the ability to identify the user and maintain a history of recordings. The figure in \ref{fig:falealgumacoisa-login-page-design} represents a login flow possible to the contributor - using a username and password. It is also possible to use the "social login" through Facebook and Google, identifiable by the respective buttons in the layout.

\begin{figure}[ht]
    \centering
    \caption{Fale Alguma Coisa Login Page designs}
    \begin{subfigure}{.5\textwidth}
      \centering
      \frame{\includegraphics[width=.9\linewidth]{images/app/login/Login2.png}}
      \caption{Login with email and password}
      \label{fig:falealgumacoisa-login-page-design}
    \end{subfigure}%
    \begin{subfigure}{.5\textwidth}
      \centering
      \frame{\includegraphics[width=.9\linewidth]{images/app/register/Register1.0.png}}
      \caption{Registration with \textit{pen name} selection}
      \label{fig:falealgumacoisa-registration-page-design-pen-name}
    \end{subfigure}
    \caption*{Source: Author}
    \label{fig:falealgumacoisa-login-and-registration-page-design}
\end{figure}

To register with the Fale Alguma Coisa app, the user must choose his \textit{pen name} (figure \ref{fig:falealgumacoisa-registration-page-design-pen-name}) and add some demographics information (seen in figure \ref{fig:falealgumacoisa-registration-page-design-demographics}. The registration flow can be triggered in two ways, and each have their differences:

\begin{itemize}
    \item Using the "social login" button. In this format, the user authenticates using the social media account and therefore does not need to provide email and password. Subsequent logins should use this method as well.
    \item Using a manual registration button ("Quero me cadastrar"). In addition to the name and demographics, the user must also provide his email and password as a last step (figure \ref{fig:falealgumacoisa-registration-page-design-email-password}).
\end{itemize}

\begin{figure}[ht]
    \centering
    \caption{Fale Alguma Coisa Registration steps designs}
    \begin{subfigure}{.5\textwidth}
      \centering
      \frame{\includegraphics[width=.9\linewidth]{images/app/register/Register2.1.png}}
      \caption{Demographics step}
      \label{fig:falealgumacoisa-registration-page-design-demographics}
    \end{subfigure}%
    \begin{subfigure}{.5\textwidth}
      \centering
      \frame{\includegraphics[width=.9\linewidth]{images/app/register/Register3.0.png}}
      \caption{Email and password step}
      \label{fig:falealgumacoisa-registration-page-design-email-password}
    \end{subfigure}
    \caption*{Source: Author}
    \label{fig:falealgumacoisa-registration-page-design}
\end{figure}

\subsubsection{Delete User}

If necessary, the user should be able to delete its user data, while still contributing his voice to the speech corpus. The following figures (\ref{fig:falealgumacoisa-delete-user-data-page-design-1} and \ref{fig:falealgumacoisa-delete-user-data-page-design-2}) include the design of the layout for this deletion flow.

\begin{figure}[ht]
    \centering
    \caption{Fale Alguma Coisa Delete User Data steps designs}
    \begin{subfigure}{.5\textwidth}
      \centering
      \frame{\includegraphics[width=.9\linewidth]{images/app/delete-user/FinishAccount1.0.png}}
      \caption{Deletion confirmation}
      \label{fig:falealgumacoisa-delete-user-data-page-design-1}
    \end{subfigure}%
    \begin{subfigure}{.5\textwidth}
      \centering
      \frame{\includegraphics[width=.9\linewidth]{images/app/delete-user/FinishAccount2.0.png}}
      \caption{Scope of deletion}
      \label{fig:falealgumacoisa-delete-user-data-page-design-2}
    \end{subfigure}
    \caption*{Source: Author}
    \label{fig:falealgumacoisa-delete-user-data-page-design}
\end{figure}

\clearpage
\subsection{Phrase Selection}
\label{sec:app-phrase-selection}

\subsection{Development}
\label{sec:app-development}

This section details the development process of the Fale Alguma Coisa app.

\subsubsection{Web app}

To allow easier access to the voice recording app, a web-based application will be developed. This factor positively influences the capacity of the app to update over time, when compared to an application developed in a native environment. It also enables users over mobile and desktop to access the same application, and although the layout may have to be redesigned, most logic is reused.

\subsubsection{Mobile First}

The design will use a mobile first approach, to ensure the user flow will be optimized when he is using a mobile device. The desktop flow will be designed and developed afterwards.

\subsubsection{Tools Selection}

To develop the application, a selection of tools was made. The table \ref{tab:tools-selection} details the selected tools and explains each choice.

\begin{table}[h]
    \centering
    \begin{tabular}{|c|c|c|}
        \hline Category & Selection & Explanation \\
        \hline Design & Zeplin & Easy sharing  \\ 
        \hline Desig &  & access \\ 
        \hline Galaxy Zoo & access & access \\ 
        \hline Christmas Audubom Birdwatch & access & access \\ \hline 
    \end{tabular}
    \caption{Contribution for online citizen science projects}
    \label{tab:cs-contributions}
\end{table}


\section{General Public Submission}
\label{sec:proposal-public-submission}

\section{Data Analysis}
\label{sec:proposal-data-analysis}

\section{Data Publication}
\label{sec:proposal-data-publication}
	\apendices 
\chapter{Ten principles of citizen science}
\label{app:ten-principles}

\begin{itemize}
    \item Citizen science projects actively involve citizens in scientific endeavour that generates new knowledge or understanding. 
    \subitem Citizens may act as contributors, collaborators or as project leaders and have a meaningful role in the project.
    \item Citizen science projects have a genuine science outcome. For example, answering a research question or informing conservation action, management decisions or environmental policy.
    \item Both the professional scientists and the citizen scientists benefit from taking part. Benefits may include the publication of research outputs, learning opportunities, personal enjoyment, social benefits, satisfaction through contributing to scientific evidence, for example, to address local, national and international issues, and through that, the potential to influence policy.
    \item Citizen scientists may, if they wish, participate in multiple stages of the scientific process. This may include developing the research question, designing the method, gathering and analysing data, and communicating the results.
    \item Citizen scientists receive feedback from the project. For example, how their data are being used and what the research, policy or societal outcomes are.
    \item Citizen science is considered a research approach like any other, with limitations and biases that should be considered and controlled for. However unlike traditional research approaches, citizen science provides opportunity for greater public engagement and democratisation of science.
    \item Citizen science project data and metadata are made publicly available and where possible, results are published in an open-access format. Data sharing may occur during or after the project, unless there are security or privacy concerns that prevent this.
    \item Citizen scientists are acknowledged in project results and publications.
    \item Citizen science programmes are evaluated for their scientific output, data quality, participant experience and wider societal or policy impact.
    \item The leaders of citizen science projects take into consideration legal and ethical issues surrounding copyright, intellectual property, data-sharing agreements, confidentiality, attribution and the environmental impact of any activities.
\end{itemize}

\chapter{Software Requirements Specification}
\label{appendix:srs}

\section{Introduction}
\subsection{Purpose}

The purpose of this Requirements Specification is to document the software requirements of Fale Alguma Coisa, a citizen science voice recording app. It should be used by anyone with the intent of understanding what are the capabilities of the system. This document will not focus on how the system will be built, but serves as the basis for the design and implementation. The design is documented within the Design Model in appendix \ref{appendix:design-model}. Details on implementation are documented in the corresponding Software Design Description (SDD) document in the appendix \ref{appendix:software-design-description}.

\subsection{Scope}

\subsubsection{Objectives}

Towards contributing \textbf{nonprofessional scientists}, the Fale Alguma Coisa app should provide an easy gateway for the user to contribute his voice while having fun and learning various science facts and curiosities.

Towards researching \textbf{scientists}, the Fale Alguma Coisa app should provide a database of anonymized voice recordings, for scientists to extract and create speech corpus.

\subsubsection{Outside scope}

However, some elements are outside of the scope of this system. Fale Alguma Coisa should \textbf{not}:
\begin{itemize}
    \item Allow for association of recording data and personal identification data (name, email);
    \item Support internationalization in the WebApp;
    \item Convert audio data into another format
    \item Support offline recording
\end{itemize}

\subsubsection{Artifacts}

This document identifies the following artifacts to be produced:

\begin{itemize}
    \item Context
    \item Domain Model
    \item Use Case Requirements (Actors, Diagrams, Specifications)
    \item Activity Diagrams
    \item Non-Functional Requirements
    \item Interface Requirements
    \item WebApp for user recordings
    \item Backend with database
\end{itemize}

\subsection{Overview of Document}

This overview provides a summary of the contents of each section of this document.

\begin{itemize}
    \item Introduction
    \item System Overview
    \item Domain Model
    \item Requirements
\end{itemize}

\subsection{References}

\begin{itemize}
    \item Based on: \cite{naoufel2014requirements}
    \item Supporting literature: \cite{pressman2014software}
\end{itemize}

\section{System Overview}

The System Overview section presents the system context and design, and also discusses the background of the Fale Alguma Coisa project.

\subsection{Name}

This system is called "Fale Alguma Coisa". This is a phrase in brazillian portuguese to order someone to "say something". It is also common to hear "fala alguma coisa", but this was changed to remove the alliteration effect, providing easier spelling.

\subsection{System Context}

This system is originated from the need of recording anonymized user voices throughout the use of gamification and citizen science.

\subsection{General Constraints}

This system is mainly impacted by constraints in data privacy. Recordings should never be associated with the users personal data (such as email), but should have metadata to aid in corpus construction, such as age interval, gender, and dialect.

\subsection{Assumptions and Dependencies}

The system heavily depends on the phrases that will be spoken. Therefore it is necessary to provide this application with a comprehensive list of phrases before the application general publication.

\section{Analysis Model}

To accurately describe the software requirements of a project, it is important to understand the requirements from many different perspectives. The analysis model proposed by \cite{pressman2014software} encompasses four different modeling elements (perspectives) that can be used to provide the most effective bridge to software design. They are listed in the subsections below as (1) scenario based models, (2) class-based elements model, (3) behavioral elements, and (4) flow-oriented elements. The analysis can lead to the derivation of one or more of these modeling elements, since one model may not completely describe certain functionalities of the app.

\subsection{Scenario-based models}

This model depicts how the user interacts with the system and the specific sequence of activities that occur as the software is used. One common implementation for this specification is through user stories. To provide scenario-model completeness, the table at \ref{fig:eyewire-game-interface} describes every scenario mapped of the WebApp. They are, however, not extensive to the functionality of the application, complemented by the remaining elements. Each of the scenarios are categorized by epic. Furthermore, each epic is described in the table \ref{tab:falealgumacoisa-epics}.

\begin{table}[h]
    \centering
    \caption{WebApp Fale Alguma Coisa Epics}
    \label{tab:falealgumacoisa-epics}
    \begin{tabular}{|p{3cm}|p{10cm}|}
        \hline Epic & Description \\
        \hline Points & This epic represents the user stories affecting the homepage, such as the splash screen, call to action button, terms of service, etc. \\
        \hline Recording & The recording epic represents the most important feature in the application, as the citizen will use it to record his voice. It also provides supporting features, such as skipping phrases and resuming the recording session.\\ 
        \hline Dashboard & The dashboard epic lists all user stories related to the dashboard page, such as where the user will be able to choose themes to record, open the menu, check his level, etc. \\
        \hline Gamification & The gamification epic details the engagement component of the application. Elements such as leaderboards, points and levels are described. They add a sense of progress to the user experience. \\
        \hline Social & To allow social interaction with other users, this epic lists social features to be added to the application. \\
        \hline Login and Registration & This application should provide user authentication to enable data management and progress saving. \\
        \hline
    \end{tabular}
    \caption*{Source: Author}
\end{table}

\begin{table}[h]
\caption{User Stories categorized to the home epic for the Fale Alguma Coisa WebApp}
\label{tab:falealgumacoisa-home-epic}
\centering
\begin{tabular}{|p{2cm}|p{1cm}|p{10cm}|}
\hline
Epic & Code & User Story \\ \hline
\multirow{5}{*}{Home} 
    & US01 & I, as an unregistered citizen, would like to view an animated introductory screen in the application, so that I feel more inside a native app. \\ \cline{2-3} 
    & US02 & I, as an unregistered citizen, would like to know more about Fale Alguma Coisa from a explanatory text in the homepage, so that I understand more about the project. \\ \cline{2-3} 
    & US03 & I, as a registered citizen, would like to easily go from the homepage to the sign-in page, so that I can login to my account. \\ \cline{2-3} 
    & US04 & I, as an unregistered citizen, would like to click a call to action button to go from the homepage to the recording page from the homepage, so that I can contribute my voice. \\ \cline{2-3} 
    & US05 & I, as an unregistered citizen, would like to read the terms of service and privacy policy of Fale Alguma Coisa, so that I understand better what the service has to offer and what kind of data will be recorded. \\ \hline
\end{tabular}
\caption*{Source: Author}
\end{table}

\begin{table}[h]
\caption{User Stories categorized to the recording epic for the Fale Alguma Coisa WebApp}
\label{tab:falealgumacoisa-recording-epic}
\centering
\begin{tabular}{|p{2cm}|p{1cm}|p{10cm}|}
\hline
Epic & Code & User Story \\ \hline
\multirow{9}{*}{Recording} 
    & US10 & I, as an unregistered citizen, would like to view and accept the terms of service before recording, so that I understand how my data is being used. \\ \cline{2-3} 
    & US11 & I, as an unregistered citizen, would like to properly configure my microphone before recording, so that I can record without interruption. \\ \cline{2-3} 
    & US12 & I, as a citizen, would like to read science phrases with definitions and curiosities, so that I learn about subjects as I am contributing. \\ \cline{2-3} 
    & US13 & I, as a citizen, would like to read phrases grouped by theme, so that I can learn more from each subject as I am contributing. \\ \cline{2-3} 
    & US14 & I, as a citizen, would like to read a tutorial explaining how to record, so that I learn how to properly record phrases. \\ \cline{2-3} 
    & US15 & I, as a citizen, would like to see animations on each step of the recording (enter the page, start the recording, stop the recording), so that I feel more engaged with the application. \\ \cline{2-3} 
    & US16 & I, as a citizen, would like to skip a phrase when I (1) do not know how to pronounce, or (2) find a foreign word, or (3) another specified reason, so that I only speak the correct phrases. \\ \cline{2-3} 
    & US17 & I, as a unregistered citizen, would like to stop this recording session by returning home when clicking the logo and confirming the exit, so that I can resume it afterwards. \\ \cline{2-3} 
    & US18 & I, as a registered citizen, would like to return to the dashboard after clicking the logo and confirming the exit, so that I can resume it afterwards. \\ \hline
\end{tabular}
\caption*{Source: Author}
\end{table}

\begin{table}[h]
\caption{User Stories categorized to the dashboard epic for the Fale Alguma Coisa WebApp}
\label{tab:falealgumacoisa-dashboard-epic}
\centering
\begin{tabular}{|p{2cm}|p{1cm}|p{10cm}|}
\hline
Epic & Code & User Story \\ \hline
\multirow{6}{*}{Dashboard} 
    & US20 & I, as an unregistered citizen, would like to easily register my data through a button click, so that I can enjoy all features of the logged area. \\ \cline{2-3} 
    & US21 & I, as a registered citizen, would like to see my actions in a dashboard after logging in, so that I can better contribute to the project. \\ \cline{2-3} 
    & US22 & I, as a registered citizen, would like to see a list of recommended themes to speak, so that I can choose one from the list. \\ \cline{2-3} 
    & US23 & I, as a registered citizen, would like to view my progress level, so that I know how far have I progressed in my contributions. \\ \cline{2-3} 
    & US24 & I, as a registered citizen, would like to open the menu, so that I know which are my possible actions in the app. \\ \cline{2-3} 
    & US25 & I, as a registered citizen, would like to check notifications, so that I understand what happened while I was gone. \\ \hline
\end{tabular}
\caption*{Source: Author}
\end{table}

\begin{table}[h]
\caption{User Stories categorized to the gamification epic for the Fale Alguma Coisa WebApp}
\label{tab:falealgumacoisa-gamification-epic}
\centering
\begin{tabular}{|p{2.5cm}|p{1cm}|p{10cm}|}
\hline
Epic & Code & User Story \\ \hline
\multirow{6}{*}{Gamification}
    & US30 & I, as a registered citizen, would like to get 100 points when I record my first phrase, so that I can engage better in the application. \\ \cline{2-3} 
    & US31 & I, as a registered citizen, would like to get 400 points when I record my first theme, so that I can engage better in the application. \\ \cline{2-3} 
    & US32 & I, as a registered citizen, would like to get 300 points when I record subsequent themes, so that I can engage better in the application. \\ \cline{2-3} 
    & US33 & I, as a registered citizen, would like to get 500 points when I register my data, so that I can better engage with the application. \\ \cline{2-3} 
    & US34 & I, as a registered citizen, would like to measure my points through a level, so that I can more easily compare myself with other users. \\ \hline
\end{tabular}
\caption*{Source: Author}
\end{table}

\begin{table}[h]
\caption{User Stories categorized to the social epic for the Fale Alguma Coisa WebApp}
\label{tab:falealgumacoisa-social-epic}
\centering
\begin{tabular}{|p{2cm}|p{1cm}|p{10cm}|}
\hline
Epic & Code & User Story \\ \hline
\multirow{6}{*}{Social}
    & US40 & I, as a registered citizen, would like to know who are the top contributors in the space and where am I in the list, so that I can compete against them. \\ \cline{2-3} 
    & US41 & I, as a registered citizen, would like to know where my friends are in a more customized leaderboard, so that I can compete against them. \\ \cline{2-3} 
    & US42 & I, as a registered citizen, would like to add a friend, so that I can check them in the friends leaderboard afterwards. \\ \cline{2-3} 
    & US43 & I, as a registered citizen, would like to check notifications, so that I can know what happened when I was away. \\ \cline{2-3} 
    & US44 & I, as a registered citizen, would like to know when someone added me through notifications, so that I can add them back later. \\ \cline{2-3} 
    & US45 & I, as a registered citizen, would like to refer friends to the application, so that I can play with them. \\ \hline
\end{tabular}
\caption*{Source: Author}
\end{table}

\begin{table}[h]
\caption{User Stories categorized to the login and registration epic for the Fale Alguma Coisa WebApp}
\label{tab:falealgumacoisa-login-and-registration-epic}
\centering
\begin{tabular}{|p{3cm}|p{1cm}|p{10cm}|}
\hline
Epic & Code & User Story \\ \hline
\multirow{4}{*}{\shortstack[l]{Login and \\ Registration}} 
    & US50 & I, as an unregistered citizen, would like to login (using social login - Facebook / Google) on the app, so that I can login later and save my progress. \\ \cline{2-3} 
    & US51 & I, as a unregistered citizen, would like to register my anonymous data on my first login, so that I can provide better metadata to my recordings afterwards. \\ \cline{2-3} 
    & US52 & I, as a registered citizen, would like to update my account data, so that I can provide accurate metadata on the recordings. \\ \cline{2-3} 
    & US53 & I, as a registered citizen, would like to remove my account data (and my recordings, if necessary), so that I can remove my metadata from this application. \\ \hline
\end{tabular}
\caption*{Source: Author}
\end{table}

\subsection{Class models}

\subsection{Behavioral models}

\subsection{Flow models}

 

% 	% ----------------------------------------------------------


% 	% ----------------------------------------------------------
% 	% Prepara pdf para iniciar o bookmark na raiz
% 	% ----------------------------------------------------------
% 	\phantompart

% 	% ----------------------------------------------------------
% 	% Conclusão
% 	% ----------------------------------------------------------
% 	%\include{conclusao}
% 	% ----------------------------------------------------------

% 	% ----------------------------------------------------------
% 	% ELEMENTOS PÓS-TEXTUAIS
% 	% ----------------------------------------------------------
% 	\postextual
% 	% ----------------------------------------------------------

% 	% ----------------------------------------------------------
% 	% Referências bibliográficas
% 	% ----------------------------------------------------------
	\bibliography{referencias}

\end{document}