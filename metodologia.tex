	\chapter[Related Work]{Related Work}
	
	In this section, we discuss how the literature has treated speech corpora creation, as well as the various conditions and variables considered in the process. Speech Corpus crafting itself is well stablished in the literature, by TIMIT \cite{Lamel1992timmit} and SWITCHBOARD \cite{godfrey1992switchboard}. TIMIT creates a dataset of 6300 utterances by 630 speakers from different regions of the United States. The sentences were crafted to fit in one of the three categories: 1) dialect "shibboleth", 2) phonemically-compact and 3) phonetically-diverse, but the selection itself was not well defined. Nevertheless, it is a very robust dataset with a time-aligned transcription and a usage guide to automatic speech recognition applications.
    
    The CHiME articles \cite{christensen2010chime} \cite{barker2013pascal}, \cite{barker2018fifth} (and more), are also source of structured speech corpora creation, challenging researchers to better recognize speech within a everyday listening environment using multiple distant microphones. Since the focus of these works lies on non-optimal recording conditions, detailed information on the noise background, noise level, recording style and speech material has been provided, as well as comprehensive post-processing work.
    
    A more recent work by \cite{chanchaochai2018globaltimit} attempts to extend the TIMIT functionality to other languages, by providing a method to create "TIMIT-like" datasets. These datasets are caracterized by having 1) Multiple (anonymously) identified speakers, 2) Wide range of phonetically-representative inputs, 3) Wideband recordings with good acoustic quality, 4) Time-aligned lexical and phonemic transcripts and 5) Easily availability to anyone. The authors detail the speakers and sessions, the text corpus selection process, the recording procedures, as well as the transcriptions and alignment methods. At the moment, there have been five datasets created, with more planned or in progress.

	\chapter[Methodology]{Methodology}
	
    This chapter should include the methology of the work. Based on the previous works, a parameter baseline should already been stablished. The content here should present the necessary steps to reach the previously defined baseline quality.

    \chapter{Results}

    This chapter should present the obtained results.

	\chapter{Conclusion}

    Lastly, this chapter should present all that has been done in the paper to reach the stablished objective, as well as the results obtained.
