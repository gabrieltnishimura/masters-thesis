\chapter[Proposal]{Proposal}

This proposal focuses on detailing the necessary steps to the creation and publication of a crowdfunding speech corpus in Brazilian Portuguese. To this end, a virtual voice recording application will be developed, focusing on adding gamification elements to enhance user engagement. Additionally, this application will also extract some relevant characteristics found in the systematic literature review in \ref{chap:slr}. Once the construction is finished, the application will be made available, enabling general public submission. After a submission period, the collected data will be analyzed and compiled to a speech corpus, which will be publicized to an open-source repository.

\section{Application}

This section details the conception, documentation, and development process of the voice recording app, coined "Fale Alguma Coisa".

\begin{figure}[ht]
    \centering
    \caption{Fale Alguma Coisa app Logo}
    \includegraphics[width=\linewidth/2]{images/app/logo.jpg}
    \caption*{Source: Author}
    \label{fig:foldit-solution}
\end{figure}

\subsection{Concept}

The Fale Alguma Coisa app should provide an easy gateway to contribute your voice while having fun and learning various science-related themes.

\subsubsection{Target Audience}

This concept is directed towards anyone interested in learning and contributing to science.

\subsection{Design}

To ensure the development of the application is effective, an iterative design approach was taken. First, ideas for the app shaped a conceptual user journey. Second, this journey allowed the creation of layouts, which then passed through a validation process. If aligned with the application concept, the layouts were developed. Otherwise, another design implementation occurred, undergoing further validation.

\subsubsection{Color Scheme}

In color theory, colors are used to communicate meaning, but also affect mood, and perception. Thus, a more colorful scheme of colors was chosen to lighten the mood of the application.

\begin{figure}[ht]
    \centering
    \caption{Fale Alguma Coisa color scheme Logo}
    \includegraphics[width=\linewidth/2]{images/app/colors.png}
    \caption*{Source: Author}
    \label{fig:foldit-solution}
\end{figure}

\subsection{Development}

This section details the development process of the voice recording app, coined "Fale Alguma Coisa".

\subsection{Requirements}

The development of most software starts with its documentation. Below are listed the functional and technical requirements of this application, divided by features.

\subsection{Web app}

To allow easier access to the voice recording app, a web-based application will be developed. This factor positively influences the capacity to update over time, when compared to a native application. It also enables users over mobile and desktop to access the same application, and most logic can be reused, although the layout may have to be redesigned.

\subsection{Mobile First}

The design will use a mobile first approach, to ensure the user flow will be optimized when he is using a mobile device. The desktop flow will be designed and developed afterwards.

\subsubsection{Splash Screen}

As the user first enter the application, a splash screen will be shown to welcome him. It contains the logo and an animation to draw the user's attention.

\begin{itemize}
    \item Splash screen
\end{itemize}

\section{General Public Submission}

\section{Data Analysis}

\section{Data Publication}