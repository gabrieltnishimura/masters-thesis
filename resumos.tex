% ---
% Arquivo com o resumo da Tese de Mestrado do aluno
% Gabriel Takaoka Nishimura da Escola Politécnica da Universidade de São Paulo
% ---

\setlength{\absparsep}{18pt}
\begin{resumo}
Public participation in scientific research, also known as citizen science, is a powerful tool that helps scientists advance knowledge, driving many breakthroughs by the sheer volume of nonprofessional contributions. This type of research became feasible not only through the advancement in technology, connectivity, and interest in science by the public; but also by some techniques to engage users, such as gamification. This addition of game elements is widely used in the private sector in a wide range of fields, and the academic field also identified this as an opportunity to engage and retain crowd-sourced environments. The combination allows citizen science to employ labor in tasks not suitable for automation, such as data gathering or classification, whilst allowing a fun and engaging experience. 

Many research fields in Computer Science highly depend on this kind of human-generated data, with requirements on quantity and quality. The research in this dissertation focuses on creating Natural Language Processing, as it also has such data-intensive requirements, more specifically voice recognition. Many approaches to training systems in voice recognition require voice datasets (or speech corpora), with multiple hours of voice recording and transcriptions. This work will create such dataset by using a citizen science gamified approach. A gamified voice recording web application is conceptualized, developed, and deployed to accept user contributions. After a submission period, the data acquired is validated and filtered, generating a speech corpus. This dataset is publicly available through the CC-BY open-source license.
	\vspace{\onelineskip}
	\noindent 
	
	\textbf{Key-words}: citizen science, gamification, speech corpus, natural language processing, brazilian portuguese
\end{resumo}
