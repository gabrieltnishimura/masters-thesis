% ---
% Arquivo com o resumo da Tese de Mestrado do aluno
% Gabriel Takaoka Nishimura da Escola Politécnica da Universidade de São Paulo
% ---

\setlength{\absparsep}{18pt}
\begin{resumo}
    Citizen science is a powerful tool to help scientists to advance scientific knowledge, driving many breakthroughs by the sheer volume of nonprofessional contributions. This is made possible by the advancement in technology, connectivity and interest in science by the public; but also by some techniques to engage users, namely, gamification. The addition of game elements is widely used in the private sector, however, the academic field is also able to benefit from it,
    retaining usage in crowd-sourced environments. This combination allows for citizen science to employ labor in tasks not suitable for automation, such as data gathering or classification, whilst allowing a fun and engaging experience. Many research fields in Computer Science highly depend on this kind of human generated data, with requirements on quantity and quality. This research focuses on Natural Language Processing, as it also has such data intensive requirements, more specifically voice recognition. Many approaches to train systems in voice recognition require voice datasets, or speech corpora, with multiple hours of voice recording and transcriptions. This work identifies citizen science gamified approaches as an opportunity to create such a dataset. Therefore, a gamified voice recording web application is conceptualized, developed, and deployed to accept user contributions. After a submission period, data acquired is validated and filtered, generating a speech corpus. This dataset is made publicly available through the CC-BY open-source license.
	\vspace{\onelineskip}
	\noindent 
	
	\textbf{Key-words}: citizen science, gamification, speech corpus, natural language processing
\end{resumo}
