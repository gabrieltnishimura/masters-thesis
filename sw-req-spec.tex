\chapter{Software Requirements Specification}
\label{appendix:srs}

\section{Introduction}
\subsection{Purpose}

The purpose of this Requirements Specification is to document the software requirements of Fale Alguma Coisa, a citizen science voice recording app. It should be used by anyone with the intent of understanding what are the capabilities of the system. This document will not focus on how the system will be built, but serves as the basis for the design and implementation. The design is documented within the Design Model in appendix \ref{appendix:design-model}. Details on implementation are documented in the corresponding Software Design Description (SDD) document in the appendix \ref{appendix:software-design-description}.

\subsection{Scope}

\subsubsection{Objectives}

Towards contributing \textbf{nonprofessional scientists}, the Fale Alguma Coisa app should provide an easy gateway for the user to contribute his voice while having fun and learning various science facts and curiosities.

Towards researching \textbf{scientists}, the Fale Alguma Coisa app should provide a database of anonymized voice recordings, for scientists to extract and create speech corpus.

\subsubsection{Outside scope}

However, some elements are outside of the scope of this system. Fale Alguma Coisa should \textbf{not}:
\begin{itemize}
    \item Allow for association of recording data and personal identification data (name, email);
    \item Support internationalization in the WebApp;
    \item Convert audio data into another format
    \item Support offline recording
\end{itemize}

\subsubsection{Artifacts}

This document identifies the following artifacts to be produced:

\begin{itemize}
    \item Context
    \item Domain Model
    \item Use Case Requirements (Actors, Diagrams, Specifications)
    \item Activity Diagrams
    \item Non-Functional Requirements
    \item Interface Requirements
    \item WebApp for user recordings
    \item Backend with database
\end{itemize}

\subsection{Overview of Document}

This overview provides a summary of the contents of each section of this document.

\begin{itemize}
    \item Introduction
    \item System Overview
    \item Domain Model
    \item Requirements
\end{itemize}

\subsection{References}

\begin{itemize}
    \item Based on: \cite{naoufel2014requirements}
    \item Supporting literature: \cite{pressman2014software}
\end{itemize}

\section{System Overview}

The System Overview section presents the system context and design, and also discusses the background of the Fale Alguma Coisa project.

\subsection{Name}

This system is called "Fale Alguma Coisa". This is a phrase in brazillian portuguese to order someone to "say something". It is also common to hear "fala alguma coisa", but this was changed to remove the alliteration effect, providing easier spelling.

\subsection{System Context}

This system is originated from the need of recording anonymized user voices throughout the use of gamification and citizen science.

\subsection{General Constraints}

This system is mainly impacted by constraints in data privacy. Recordings should never be associated with the users personal data (such as email), but should have metadata to aid in corpus construction, such as age interval, gender, and dialect.

\subsection{Assumptions and Dependencies}

The system heavily depends on the phrases that will be spoken. Therefore it is necessary to provide this application with a comprehensive list of phrases before the application general publication.

\section{Domain Model}



\section{Requirements}

To accurately describe the software requirements of a project, it is important to understand the requirements from many different perspectives. The analysis model proposed by \cite{pressman2014software} encompasses four different modeling elements (perspectives) that can be used to provide the most effective bridge to software design. They are listed in the subsections below as (1) scenario based models, (2) class-based elements model, (3) behavioral elements, and (4) flow-oriented elements. The analysis can lead to the derivation of one or more of these modeling elements, since one model may not completely describe certain functionalities of the app. Additionally, the actor list will be provided in the table \ref{tab:actor-list} below:

\begin{table}[h]
\centering
\caption{WebApp Fale Alguma Coisa Actor List}
\label{tab:actor-list}
\begin{tabular}{|p{2.5cm}|p{3.5cm}|p{7cm}|}
    \hline 
    Actor & Responsibilities & Description  \\ \hline
    Unregistered Citizen & Record phrases & User that has not logged on in the WebApp, and has not added his metadata \\ \hline
    Registered Citizen & Record phrases with metadata & User that has added his metadata on the WebApp \\ \hline 
    Validation Scientist & Validate recorded phrases & Download and validate recordings \\ \hline
    Corpus User & Download validated phrases & Access the open speech corpus and download it \\ \hline
\end{tabular}
\caption*{Source: Author}
\end{table}

\subsection{Scenario-based models}

This model depicts how the user interacts with the system and the specific sequence of activities that occur as the software is used. One common implementation for this specification is through use cases. To provide scenario-model completeness, the following sections will describe every use case mapped to the WebApp. They are, however, not extensive to the functionality of the application, some use cases are complemented by the models cited above. Each of the use cases are categorized by feature. Furthermore, each feature is described in the table \ref{tab:falealgumacoisa-features}.

\begin{table}[h]
    \centering
    \caption{WebApp Fale Alguma Coisa Features}
    \label{tab:falealgumacoisa-features}
    \begin{tabular}{|p{3cm}|p{10cm}|}
        \hline Epic & Description \\
        \hline Home & Represents the use cases affecting the homepage, such as the splash screen, call to action button, terms of service, etc. \\
        \hline Recording & Represents the most important feature in the application, as the citizen will use it to record his voice. It also provides supporting features, such as skipping phrases and resuming the recording session.\\ 
        \hline Dashboard & Lists all user stories related to the dashboard page, such as where the user will be able to choose themes to record, open the menu, check his level, etc. \\
        \hline Gamification & Details the engagement component of the application. Elements such as leaderboards, points and levels are described. They add a sense of progress to the user experience. \\
        \hline Social & To allow social interaction with other users, this feature lists social features to be added to the application. \\
        \hline Login and Registration & The application should provide user authentication to enable data management and progress saving. \\
        \hline
    \end{tabular}
    \caption*{Source: Author}
\end{table}

\subsubsection{UC01 - View Home Splash}

\begin{table}[h]
    \centering
    \caption{WebApp Fale Alguma Coisa Features}
    \label{tab:falealgumacoisa-features}
    \begin{tabular}{|p{3cm}|p{10cm}|}
        \hline Use Case ID & View Home Splash \\
        \hline Home & Represents the use cases affecting the homepage, such as the splash screen, call to action button, terms of service, etc. \\
        \hline Recording & Represents the most important feature in the application, as the citizen will use it to record his voice. It also provides supporting features, such as skipping phrases and resuming the recording session.\\ 
        \hline Dashboard & Lists all user stories related to the dashboard page, such as where the user will be able to choose themes to record, open the menu, check his level, etc. \\
        \hline Gamification & Details the engagement component of the application. Elements such as leaderboards, points and levels are described. They add a sense of progress to the user experience. \\
        \hline Social & To allow social interaction with other users, this feature lists social features to be added to the application. \\
        \hline Login and Registration & The application should provide user authentication to enable data management and progress saving. \\
        \hline
    \end{tabular}
    \caption*{Source: Author}
\end{table}











\begin{table}[h]
\caption{Simplified Home Use Cases for the Fale Alguma Coisa WebApp}
\label{tab:falealgumacoisa-simplified-home}
\centering
\begin{tabular}{|p{1cm}|p{3cm}|p{10cm}|}
\hline
    Code & Use Case Name & Description \\ \hline
    UC01 & View Home Splash & An unregistered citizen wants to view an animated introductory screen (splash) when entering the application, so that he feels more inside a native app. \\ \hline
    UC02 & View Home Description & An unregistered citizen wants to check the initiative description, so that he understands more about the Fale Alguma Coisa citizen science project. \\ \hline
    UC03 & Navigate Home Login & A registered citizen wants to navigates from the homepage to the sign-in page, so that he logins (or register) to his account. \\ \hline
    UC04 & Navigate Home Recording & An unregistered citizen wants to easily navigate from the homepage to the recording page, so that he can contribute his voice. \\ \hline
    UC05 & Read Home Terms & An unregistered citizen wants to reads the terms of service (and privacy policy) of Fale Alguma Coisa, so that he understands better what the service has to offer and what kind of data will be recorded. \\ \hline
\end{tabular}
\caption*{Source: Author}
\end{table}

\clearpage
\subsubsection{Recording}

Represents the most important feature in the application, as the citizen will use it to record his voice. It also provides supporting features, such as skipping phrases and resuming the recording session. The list of simplified use cases is listed in table \ref{tab:falealgumacoisa-simplified-recording}; with the complete version available in the Appendix \ref{appendix:scenario-based-model}

\begin{table}[h]
\caption{Simplified Recording Use Cases for the Fale Alguma Coisa WebApp}
\label{tab:falealgumacoisa-simplified-recording}
\centering
\begin{tabular}{|p{1cm}|p{3cm}|p{10cm}|}
\hline
    Code & Use Case Name & Description \\ \hline
    UC10 & Accept Terms & An unregistered citizen would like to view and accept the terms of service before recording, so that he understands how his data is being used. \\ \hline
    UC11 & Configure Microphone & An unregistered citizen would like to properly configure my microphone before recording, so that he can record without interruption. \\ \hline
    UC12 & Record Phrases & A citizen would like to read science phrases with definitions and curiosities, so that he learns about subjects as he is contributing. \\ \hline
    UC13 & Group Phrases & A citizen would like to read phrases grouped by theme, so that he can learn more from each subject as he is contributing. \\ \hline
    UC14 & Read Tutorial & A citizen would like to read a tutorial explaining how to record, so that he learns how to properly record phrases. \\ \hline
    UC15 & Watch Recording Animations & A citizen would like to see animations on each step of the recording (enter the page, start the recording, stop the recording), so that he feels more engaged with the application. \\ \hline
    UC16 & Skip Phrase & A citizen would like to skip a phrase when he (1) does not know how to pronounce, or (2) finds a foreign word, or (3) finds another specified reason, so that he only speaks the correct phrases. \\ \hline
    UC17 & Stop Recording Session & A citizen would like to stop this recording session by clicking the logo and confirming the exit, so that he can resume it afterwards. \\ \hline
\end{tabular}
\caption*{Source: Author}
\end{table}

\clearpage
\subsubsection{Dashboard}

Table \ref{tab:falealgumacoisa-simplified-dashboard} lists all user stories related to the dashboard page, such as where the user will be able to choose themes to record, open the menu, check his level, etc.

\begin{table}[h]
\caption{Simplified Dashboard Use Cases for the Fale Alguma Coisa WebApp}
\label{tab:falealgumacoisa-simplified-dashboard}
\centering
\begin{tabular}{|p{1cm}|p{3cm}|p{10cm}|}
\hline
    Code & Use Case Name & Description \\ \hline
    US20 & Navigate Register & An unregistered citizen would like to easily register his data through a button click, so that he can enjoy all features of the logged area. \\ \hline
    US21 & View Actions & A registered citizen would like to see his actions in a dashboard after logging in, so that he can better contribute to the project. \\ \hline
    US22 & Recommend Themes & A registered citizen would like to see a list of recommended themes to speak, so that he can choose one from the list. \\ \hline
    US23 & View Progress Level & A registered citizen would like to view my progress level, so that he know how far have he  progressed in my contributions. \\ \hline
    US24 & Open Menu & A registered citizen would like to open the menu, so that he knows which are his possible actions in the app. \\ \hline
    US25 & View Notifications & A registered citizen would like to check notifications, so that he understands what happened while he was gone. \\ \hline
\end{tabular}
\caption*{Source: Author}
\end{table}

\subsubsection{Social}

To allow the social interaction with other users, the table \ref{tab:falealgumacoisa-simplified-social} lists social use cases to be added to the application. Not all social interaction is listed in this feature, as there are some elements in gamification, such as competition.

\begin{table}[h]
\caption{Simplified Social Use Cases for the Fale Alguma Coisa WebApp}
\label{tab:falealgumacoisa-simplified-social}
\centering
\begin{tabular}{|p{1cm}|p{3cm}|p{10cm}|}
\hline
    Code & Use Case Name & Description \\ \hline
    US30 & Add Friend & A registered citizen would like to add a friend, so that he can check them in the friends leaderboard afterwards. \\ \hline
    US31 & View Notifications & A registered citizen would like to check notifications, so that he can know what happened when he was away. \\ \hline
    US32 & Receive Notifications & A registered citizen would like to know when someone added him through notifications, so that he can add them back later. \\ \hline
    US33 & Refer Friends & A registered citizen would like to refer friends to the application, so that he can play with them afterwards. \\ \hline
\end{tabular}
\caption*{Source: Author}
\end{table}

\subsubsection{Gamification}

Table \ref{tab:falealgumacoisa-simplified-gamification} details the engagement component of the application. Elements such as leaderboards, points and levels are described. They add competition and a sense of progress to the user experience, supporting the core need of this system, which is phrase recording.

\begin{table}[h]
\caption{Simplified Gamification Use Cases for the Fale Alguma Coisa WebApp}
\label{tab:falealgumacoisa-simplified-gamification}
\centering
\begin{tabular}{|p{1cm}|p{3cm}|p{10cm}|}
\hline
    Code & Use Case Name & Description \\ \hline
    US40 & Earn First Recording & A registered citizen would like to get 100 points when he records his first phrase, so that he can engage better in the application. \\ \hline
    US41 & Earn First Theme & A registered citizen would like to get 400 points when he records his first theme, so that he can engage better in the application. \\ \hline
    US42 & Earn Theme & A registered citizen would like to get 300 points when he records subsequent themes, so that he can engage better in the application. \\ \hline
    US43 & Earn Registration & A registered citizen would like to get 500 points when he registers his speaker data, so that he can better engage with the application. \\ \hline
    US44 & Calculate Level & A registered citizen would like to measure his points through a level, so that he can more easily compare himself with other users. \\ \hline
    US45 & Compete Top Players & A registered citizen would like to know who are the top contributors in the space and where he is in the list, so that he can compete against them. \\ \hline
    US46 & Compete Friends & A registered citizen would like to know where his friends are in the friends leaderboard, so that he can compete against them. \\ \hline
\end{tabular}
\caption*{Source: Author}
\end{table}

\subsubsection{Login and Registration}

The application should provide user authentication to enable data management and progress saving. Table \ref{tab:falealgumacoisa-simplified-login-and-registration} lists all use cases referring to this feature, and is the key to generating recordings with speaker metadata.

\begin{table}[h]
\caption{User Stories categorized to the login and registration epic for the Fale Alguma Coisa WebApp}
\label{tab:falealgumacoisa-simplified-login-and-registration}
\centering
\begin{tabular}{|p{1cm}|p{3cm}|p{10cm}|}
\hline
    Code & Use Case Name & Description \\ \hline
    US50 & Login Social & An unregistered citizen would like to login (using social login - Facebook / Google) on the app, so that he can login later and save my progress. \\ \hline
    US51 & Register Metadata & A unregistered citizen would like to register his anonymous data on his first login, so that he can provide better metadata to my recordings afterwards. \\ \hline
    US52 & Update Account & A registered citizen would like to update his account data, so that he can provide accurate metadata on the recordings. \\ \hline
    US53 & Delete Account & A registered citizen would like to remove his account data (and his recordings, if necessary), so that he can remove my metadata from this application. \\ \hline
\end{tabular}
\caption*{Source: Author}
\end{table}










\subsubsection{Usability}

The usability epic details user stories related to the user experience with the app.

\begin{itemize}
    \item I, as a citizen, would like to contribute my voice using my mobile devices and desktop computer.
\end{itemize}

\subsection{Class models}

\subsection{Behavioral models}

\subsection{Flow models}

\section{Design Model}

