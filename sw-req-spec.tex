\chapter{Software Requirements Specification}
\label{appendix:srs}

\section{Introduction}
\subsection{Purpose}

The purpose of this Requirements Specification is to document the software requirements of Fale Alguma Coisa, a citizen science voice recording app. It should be used by anyone with the intent of understanding what are the capabilities of the system. This document will not focus on how the system will be built, but serves as the basis for the design and implementation. The design is documented within the Design Model in appendix \ref{appendix:design-model}. Details on implementation are documented in the corresponding Software Design Description (SDD) document in the appendix \ref{appendix:software-design-description}.

\subsection{Scope}

\subsubsection{Objectives}

Towards contributing \textbf{nonprofessional scientists}, the Fale Alguma Coisa app should provide an easy gateway for the user to contribute his voice while having fun and learning various science facts and curiosities.

Towards researching \textbf{scientists}, the Fale Alguma Coisa app should provide a database of anonymized voice recordings, for scientists to extract and create speech corpus.

\subsubsection{Outside scope}

However, some elements are outside of the scope of this system. Fale Alguma Coisa should \textbf{not}:
\begin{itemize}
    \item Allow for association of recording data and personal identification data (name, email);
    \item Support internationalization in the WebApp;
    \item Convert audio data into another format
    \item Support offline recording
\end{itemize}

\subsubsection{Artifacts}

This document identifies the following artifacts to be produced:

\begin{itemize}
    \item Context
    \item Domain Model
    \item Use Case Requirements (Actors, Diagrams, Specifications)
    \item Activity Diagrams
    \item Non-Functional Requirements
    \item Interface Requirements
    \item WebApp for user recordings
    \item Backend with database
\end{itemize}

\subsection{Overview of Document}

This overview provides a summary of the contents of each section of this document.

\begin{itemize}
    \item Introduction
    \item System Overview
    \item Domain Model
    \item Requirements
\end{itemize}

\subsection{References}

\begin{itemize}
    \item Based on: \cite{naoufel2014requirements}
    \item Supporting literature: \cite{pressman2014software}
\end{itemize}

\section{System Overview}

The System Overview section presents the system context and design, and also discusses the background of the Fale Alguma Coisa project.

\subsection{Name}

This system is called "Fale Alguma Coisa". This is a phrase in brazillian portuguese to order someone to "say something". It is also common to hear "fala alguma coisa", but this was changed to remove the alliteration effect, providing easier spelling.

\subsection{System Context}

This system is originated from the need of recording anonymized user voices throughout the use of gamification and citizen science.

\subsection{General Constraints}

This system is mainly impacted by constraints in data privacy. Recordings should never be associated with the users personal data (such as email), but should have metadata to aid in corpus construction, such as age interval, gender, and dialect.

\subsection{Assumptions and Dependencies}

The system heavily depends on the phrases that will be spoken. Therefore it is necessary to provide this application with a comprehensive list of phrases before the application general publication.

\section{Analysis Model}

To accurately describe the software requirements of a project, it is important to understand the requirements from many different perspectives. The analysis model proposed by \cite{pressman2014software} encompasses four different modeling elements (perspectives) that can be used to provide the most effective bridge to software design. They are listed in the subsections below as (1) scenario based models, (2) class-based elements model, (3) behavioral elements, and (4) flow-oriented elements. The analysis can lead to the derivation of one or more of these modeling elements, since one model may not completely describe certain functionalities of the app.

\subsection{Scenario-based models}

This model depicts how the user interacts with the system and the specific sequence of activities that occur as the software is used. One common implementation for this specification is through use cases. To provide scenario-model completeness, the following sections will describe every use case mapped to the WebApp. They are, however, not extensive to the functionality of the application, some use cases are complemented by the models cited above. Each of the use cases are categorized by feature. Furthermore, each feature is described in the table \ref{tab:falealgumacoisa-features}.

\begin{table}[h]
    \centering
    \caption{WebApp Fale Alguma Coisa Features}
    \label{tab:falealgumacoisa-features}
    \begin{tabular}{|p{3cm}|p{10cm}|}
        \hline Epic & Description \\
        \hline Home & Represents the use cases affecting the homepage, such as the splash screen, call to action button, terms of service, etc. \\
        \hline Recording & Represents the most important feature in the application, as the citizen will use it to record his voice. It also provides supporting features, such as skipping phrases and resuming the recording session.\\ 
        \hline Dashboard & Lists all user stories related to the dashboard page, such as where the user will be able to choose themes to record, open the menu, check his level, etc. \\
        \hline Gamification & Details the engagement component of the application. Elements such as leaderboards, points and levels are described. They add a sense of progress to the user experience. \\
        \hline Social & To allow social interaction with other users, this feature lists social features to be added to the application. \\
        \hline Login and Registration & The application should provide user authentication to enable data management and progress saving. \\
        \hline
    \end{tabular}
    \caption*{Source: Author}
\end{table}


\subsubsection{Usability}

The usability epic details user stories related to the user experience with the app.

\begin{itemize}
    \item I, as a citizen, would like to contribute my voice using my mobile devices and desktop computer.
\end{itemize}

\subsection{Class models}

\subsection{Behavioral models}

\subsection{Flow models}

 
