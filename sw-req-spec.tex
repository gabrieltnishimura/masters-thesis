\chapter{Software Requirements Specification}
\label{appendix:srs}

\section{Introduction}
\subsection{Purpose}
\label{appendix:srs:purpose}

The purpose of this Requirements Specification is to document the software requirements of Fale Alguma Coisa, a citizen science voice recording app. It should be used by anyone with the intent of understanding what are the capabilities of the system. This document will not focus on how the system will be built, but serves as the basis for the design and implementation. The design is documented within the Design Model in appendix \ref{appendix:design-model}. Details on implementation are documented in the corresponding Software Design Description (SDD) document in the appendix \ref{appendix:software-design-description}.

\subsection{Scope}
\label{appendix:srs:scope}

\subsubsection{Objectives}

Towards contributing \textbf{nonprofessional scientists}, the Fale Alguma Coisa app should provide an easy gateway for the user to contribute his voice while having fun and learning various science facts and curiosities.

Towards researching \textbf{scientists}, the Fale Alguma Coisa app should provide a database of anonymized voice recordings, for scientists to extract and create speech corpus.

\subsubsection{Outside scope}

However, some elements are outside of the scope of this system. Fale Alguma Coisa should \textbf{not}:
\begin{itemize}
    \item Allow for association of recording data and personal identification data (name, email);
    \item Support internationalization in the WebApp;
    \item Convert audio data into another format
    \item Support offline recording
\end{itemize}

\subsubsection{Artefacts}
\label{appendix:srs:artefacts}

This document identifies the following artifacts to be produced:

\begin{itemize}
    \item Context
    \item Domain Model
    \item Use Case Requirements (Actors, Diagrams, Specifications)
    \item Activity Diagrams
    \item Non-Functional Requirements
    \item Interface Requirements
    \item WebApp for user recordings
    \item Backend integrating with database
    \item Database structure
\end{itemize}

\subsection{Overview of Document}

This overview provides a summary of the contents of each section of this document.

\begin{itemize}
    \item Introduction
    \item System Overview
    \item Domain Model
    \item Requirements
\end{itemize}

\subsection{References}

\begin{itemize}
    \item Based on: \cite{naoufel2014requirements}
    \item Supporting literature: \cite{pressman2014software}
\end{itemize}

\section{System Overview}
\label{appendix:srs:system-overview}

The System Overview section presents the system context and design, and also discusses the background of the Fale Alguma Coisa project.

\subsection{Name}

This system is called "Fale Alguma Coisa". This is a phrase in brazillian portuguese to order someone to "say something". It is also common to hear "fala alguma coisa", but this was changed to remove the alliteration effect, providing easier spelling.

\subsection{System Context}

This system is originated from the need of recording anonymized user voices throughout the use of gamification and citizen science.

\subsection{General Constraints}

This system is mainly impacted by constraints in data privacy. Recordings should never be associated with the users personal data (such as email), but should have metadata to aid in corpus construction, such as age interval, gender, and dialect.

\subsection{Assumptions and Dependencies}

The system heavily depends on the phrases that will be spoken. Therefore, it is necessary to provide this application with a comprehensive list of phrases before the application general publication.

\section{Requirements}
\label{appendix:srs:requirements}

To accurately describe the software requirements of a project, it is important to understand the requirements from many different perspectives. The analysis model proposed by \cite{pressman2014software} encompasses four different modeling elements (perspectives) that can be used to provide the most effective bridge to software design. They are listed in the subsections below as (1) scenario based models, (2) class-based elements model, (3) behavioral elements, and (4) flow-oriented elements. The analysis can lead to the derivation of one or more of these modeling elements, since one model may not completely describe certain functionalities of the app. Additionally, the actor list will be provided in the table \ref{tab:actor-list} below:

\begin{table}[h]
\centering
\caption{WebApp Fale Alguma Coisa Actor List}
\label{tab:actor-list}
\begin{tabular}{|p{2.5cm}|p{3.5cm}|p{7cm}|}
    \hline 
    Actor & Responsibilities & Description  \\ \hline
    Unregistered Citizen & Record phrases & User that has not logged on in the WebApp, and has not added his metadata \\ \hline
    Registered Citizen & Record phrases with metadata & User that has added his metadata on the WebApp \\ \hline 
    Validation Scientist & Validate recorded phrases & Download and validate recordings \\ \hline
    Corpus User & Download validated phrases & Access the open speech corpus and download it \\ \hline
\end{tabular}
\caption*{Source: Author}
\end{table}

\subsection{Scenario-based models}
\label{appendix:srs:scenario-based-models}

This model depicts how the user interacts with the system and the specific sequence of activities that occur as the software is used. One common implementation for this specification is through use cases. To provide scenario-model completeness, the following sections will describe every use case mapped to the WebApp. They are, however, not extensive to the functionality of the application, some use cases are complemented by the models cited above. Each of the use cases are categorized by feature. Furthermore, each feature is described in the table \ref{tab:falealgumacoisa-features}.

\begin{table}[h]
    \centering
    \caption{WebApp Fale Alguma Coisa Features}
    \label{tab:falealgumacoisa-features}
    \begin{tabular}{|p{3cm}|p{10cm}|}
        \hline Epic & Description \\
        \hline Home & Represents the use cases affecting the homepage, such as the splash screen, call to action button, terms of service, etc. \\
        \hline Recording & Represents the most important feature in the application, as the citizen will use it to record his voice. It also provides supporting features, such as skipping phrases and resuming the recording session.\\ 
        \hline Dashboard & Lists all user stories related to the dashboard page, such as where the user will be able to choose themes to record, open the menu, check his level, etc. \\
        \hline Gamification & Details the engagement component of the application. Elements such as leaderboards, points and levels are described. They add a sense of progress to the user experience. \\
        \hline Social & To allow social interaction with other users, this feature lists social features to be added to the application. \\
        \hline Login and Registration & The application should provide user authentication to enable data management and progress saving. \\
        \hline
    \end{tabular}
    \caption*{Source: Author}
\end{table}

\clearpage
\subsubsection{Home Use Cases}

\begin{table}[h]
\centering
\caption{UC01 - View Home}
\label{uc:01}
\begin{tabular}{|p{3cm}|p{1cm}|p{5cm}|p{5cm}|}
\hline
Use Case:       & \multicolumn{3}{p{11cm}|}{View Home} \\ \hline
Use Case ID     & \multicolumn{3}{p{11cm}|}{UC01} \\ \hline
Description     & \multicolumn{3}{p{11cm}|}{After entering the application, the unregistered citizen is shown an quick animated introductory screen (splash). After the splash, the homepage is displayed with information on the Fale Alguma Coisa citizen science project, as well as statistics on the progress of the recordings and contributors.} \\ \hline
Rationale       & \multicolumn{3}{p{11cm}|}{So that they feel more inside a native app and know more about the initiative} \\ \hline
Precondition    & \multicolumn{3}{p{11cm}|}{None} \\ \hline
Post-condition  & \multicolumn{3}{p{11cm}|}{Application shows the home page} \\ \hline
Actor           & \multicolumn{3}{p{11cm}|}{Unregistered citizen} \\ \hline
Trigger         & \multicolumn{3}{p{11cm}|}{Enters the application} \\ \hline
Main flow of events & Step  & User action & System response \\ \hline
                    & 1     & User enters the application & A full screen image with the logo, title and description of the app is shown \\ \hline
                    & 2     & User waits for a short period of time (<200ms) & The homepage is displayed, with information and statistics \\ \hline
Exceptions      & \multicolumn{3}{p{11cm}|}{None} \\ \hline
\end{tabular}
\caption*{Source: Author}
\end{table}

\begin{table}[h]
\centering
\caption{UC02 - Read Home Terms or Privacy Policy}
\label{uc:02}
\begin{tabular}{|p{3cm}|p{1cm}|p{5cm}|p{5cm}|}
\hline
Use Case:       & \multicolumn{3}{p{11cm}|}{Read Home Terms or Privacy Policy} \\ \hline
Use Case ID     & \multicolumn{3}{p{11cm}|}{UC02} \\ \hline
Description     & \multicolumn{3}{p{11cm}|}{In the homepage, should the user want to read the terms of service or privacy policy of the Fale Alguma Coisa app, a simple link in the footer is displayed.} \\ \hline
Rationale       & \multicolumn{3}{p{11cm}|}{The user understands what they are being subjected when using the service, as well how their data is being used} \\ \hline
Precondition    & \multicolumn{3}{p{11cm}|}{None} \\ \hline
Post-condition  & \multicolumn{3}{p{11cm}|}{Terms of service or privacy policy is shown} \\ \hline
Actor           & \multicolumn{3}{p{11cm}|}{Citizen, not necessarily logged} \\ \hline
Trigger         & \multicolumn{3}{p{11cm}|}{Clicks the footer link} \\ \hline
Main flow of events & Step  & User action & System response \\ \hline
                    & 1     & User enters the application & Splash animation is shown \\ \hline
                    & 2     & User waits for animation & Home page is displayed \\ \hline
                    & 3     & User clicks on the "Terms of Service" or "Privacy Policy" link in the footer & Page is displayed with information\\ \hline
Exceptions      & \multicolumn{3}{p{11cm}|}{None} \\ \hline
\end{tabular}
\caption*{Source: Author}
\end{table}

\begin{table}[h]
\centering
\caption{UC03 - Navigate Login}
\label{uc:03}
\begin{tabular}{|p{3cm}|p{1cm}|p{5cm}|p{5cm}|}
\hline
Use Case:       & \multicolumn{3}{p{11cm}|}{Navigate Login} \\ \hline
Use Case ID     & \multicolumn{3}{p{11cm}|}{UC03} \\ \hline
Description     & \multicolumn{3}{p{11cm}|}{A registered citizen wants to navigates from the home-page to the sign-in page, so that they login (or register) to their account} \\ \hline
Rationale       & \multicolumn{3}{p{11cm}|}{Access to the registered application} \\ \hline
Precondition    & \multicolumn{3}{p{11cm}|}{User is registered to the Fale Alguma Coisa app} \\ \hline
Post-condition  & \multicolumn{3}{p{11cm}|}{User reaches login page} \\ \hline
Actor           & \multicolumn{3}{p{11cm}|}{Registered citizen} \\ \hline
Trigger         & \multicolumn{3}{p{11cm}|}{Enters the application} \\ \hline
Main flow of events & Step  & User action & System response \\ \hline
                    & 1     & User enters the application & Splash animation is shown \\ \hline
                    & 2     & User clicks the login button in the header & Login page is displayed \\ \hline
Exceptions      & \multicolumn{3}{p{11cm}|}{None} \\ \hline
\end{tabular}
\caption*{Source: Author}
\end{table}

\clearpage
\subsubsection{Recording Use Cases}

\begin{table}[h]
\centering
\caption{UC10 - Accept Terms}
\label{uc:10}
\begin{tabular}{|p{3cm}|p{1cm}|p{5cm}|p{5cm}|}
\hline
Use Case:       & \multicolumn{3}{p{11cm}|}{Accept Terms} \\ \hline
Use Case ID     & \multicolumn{3}{p{11cm}|}{UC10} \\ \hline
Description     & \multicolumn{3}{p{11cm}|}{An unregistered citizen would like to view and accept the terms of service before recording, so that they understands how his data is being used.} \\ \hline
Rationale       & \multicolumn{3}{p{11cm}|}{Transparency for the user} \\ \hline
Precondition    & \multicolumn{3}{p{11cm}|}{User is in the homepage and has not accepted the terms of service} \\ \hline
Post-condition  & \multicolumn{3}{p{11cm}|}{Acceptance of terms is recorded} \\ \hline
Actor           & \multicolumn{3}{p{11cm}|}{Unregistered citizen} \\ \hline
Trigger         & \multicolumn{3}{p{11cm}|}{Clicks the record button} \\ \hline
Main flow of events & Step  & User action & System response \\ \hline
                    & 1     & User clicks the record button & Terms of service page is displayed \\ \hline
                    & 2     & User reads and accepts the Terms of Service & Acceptance is recorded and the user is redirected to the next step of the recording flow \\ \hline
Exceptions      & \multicolumn{3}{p{11cm}|}{If the user does not accept the terms, they are redirected back to the homepage.} \\ \hline
\end{tabular}
\caption*{Source: Author}
\end{table}

\begin{table}[h]
\centering
\caption{UC11 - Configure Microphone}
\label{uc:11}
\begin{tabular}{|p{3cm}|p{1cm}|p{5cm}|p{5cm}|}
\hline
Use Case:       & \multicolumn{3}{p{11cm}|}{Configure Microphone} \\ \hline
Use Case ID     & \multicolumn{3}{p{11cm}|}{UC11} \\ \hline
Description     & \multicolumn{3}{p{11cm}|}{An unregistered citizen would like to properly configure my microphone before recording, so that he can record without interruption.} \\ \hline
Rationale       & \multicolumn{3}{p{11cm}|}{Proper configuration of the microphone before the recording flow is started.} \\ \hline
Precondition    & \multicolumn{3}{p{11cm}|}{User has accepted the terms of service and has not yet allowed the microphone in the browser} \\ \hline
Post-condition  & \multicolumn{3}{p{11cm}|}{User redirected to the recording page} \\ \hline
Actor           & \multicolumn{3}{p{11cm}|}{Unregistered citizen} \\ \hline
Trigger         & \multicolumn{3}{p{11cm}|}{Clicks the record button} \\ \hline
Main flow of events & Step  & User action & System response \\ \hline
                    & 1     & User clicks the record button in the homepage & Configure microphone page is displayed, with explanation on how to allow it \\ \hline
                    & 2     & User user clicks to continue & System prompt for the microphone to be allowed \\ \hline
                    & 3     & User allows the usage of the microphone & User is redirected to. the recording page \\ \hline
Exceptions      & \multicolumn{3}{p{11cm}|}{If the user does not allow the microphone, they are redirected to a page detailing the app has no access to the mic} \\ \hline
\end{tabular}
\caption*{Source: Author}
\end{table}

\begin{table}[h]
\centering
\caption{UC12 - Record Phrases From Homepage}
\label{uc:12}
\begin{tabular}{|p{3cm}|p{1cm}|p{5cm}|p{5cm}|}
\hline
Use Case:       & \multicolumn{3}{p{11cm}|}{Record Phrases From Homepage} \\ \hline
Use Case ID     & \multicolumn{3}{p{11cm}|}{UC12} \\ \hline
Description     & \multicolumn{3}{p{11cm}|}{A citizen entering the homepage would like to read a science phrase with definitions and curiosities, so that they learn about subjects a they are contributing.} \\ \hline
Rationale       & \multicolumn{3}{p{11cm}|}{Contributing to science and learning topics} \\ \hline
Precondition    & \multicolumn{3}{p{11cm}|}{Accepted the terms of service, allowed the microphone, is in the homepage} \\ \hline
Post-condition  & \multicolumn{3}{p{11cm}|}{Phrase is recorded} \\ \hline
Actor           & \multicolumn{3}{p{11cm}|}{Unregistered citizen} \\ \hline
Trigger         & \multicolumn{3}{p{11cm}|}{Clicks the record button} \\ \hline
Main flow of events & Step  & User action & System response \\ \hline
                    & 1     & User clicks the record button in the homepage & Recording page is displayed, as well as an animation and a tutorial explanation on how to record \\ \hline
                    & 2     & User clicks and holds the recording button & The tutorial message is changed and the page focuses on the phrase to be read. There is also an animation to focus on the change \\ \hline
                    & 3     & User reads the phrase out loud and releases the recording button & The phrase is locally recorded and an animated confirmation is shown \\ \hline
                    & 4     & User clicks on the play button to hear the recording & Recording is played back \\ \hline
                    & 5     & User confirms that the recording is ok & Next phrase is shown \\ \hline
Exceptions      & \multicolumn{3}{p{11cm}|}{The user may not find the recording correct. If so, he may reject the recording at step 4 and record again.} \\ \hline
\end{tabular}
\caption*{Source: Author}
\end{table}

\begin{table}[h]
\centering
\caption{UC13 - Record Phrases From Dashboard}
\label{uc:13}
\begin{tabular}{|p{3cm}|p{1cm}|p{5cm}|p{5cm}|}
\hline
Use Case:       & \multicolumn{3}{p{11cm}|}{Record Phrases From Dashboard} \\ \hline
Use Case ID     & \multicolumn{3}{p{11cm}|}{UC13} \\ \hline
Description     & \multicolumn{3}{p{11cm}|}{A citizen entering the dashboard would like to choose a theme and record a related phrase with definitions and curiosities, so that he learns about the chosen subject a they are contributing.} \\ \hline
Rationale       & \multicolumn{3}{p{11cm}|}{Contributing to science and learning topics} \\ \hline
Precondition    & \multicolumn{3}{p{11cm}|}{Accepted the terms of service, allowed the microphone, is in the homepage} \\ \hline
Post-condition  & \multicolumn{3}{p{11cm}|}{Phrase is recorded} \\ \hline
Actor           & \multicolumn{3}{p{11cm}|}{Registered citizen} \\ \hline
Trigger         & \multicolumn{3}{p{11cm}|}{Enters the dashboard} \\ \hline
Main flow of events & Step  & User action & System response \\ \hline
                    & 1     & User chooses and clicks on a theme to record & Recording page is displayed, as well as a tutorial explanation on how to record \\ \hline
                    & 2...5     & Remaining steps are identical to the use case at \ref{uc:12} & Remaining steps are identical to the use case at \ref{uc:12} \\ \hline
Exceptions      & \multicolumn{3}{p{11cm}|}{Identical to the use case at \ref{uc:12}} \\ \hline
\end{tabular}
\caption*{Source: Author}
\end{table}

\begin{table}[h]
\centering
\caption{UC14 - Record Theme}
\label{uc:14}
\begin{tabular}{|p{3cm}|p{1cm}|p{5cm}|p{5cm}|}
\hline
Use Case:       & \multicolumn{3}{p{11cm}|}{Record Theme} \\ \hline
Use Case ID     & \multicolumn{3}{p{11cm}|}{UC14} \\ \hline
Description     & \multicolumn{3}{p{11cm}|}{A citizen would like to read phrases grouped by theme, so that he can learn more from each subject as he is contributing. To finish a theme, the citizen must read 6 phrases.} \\ \hline
Rationale       & \multicolumn{3}{p{11cm}|}{Contributing to science and learning topics} \\ \hline
Precondition    & \multicolumn{3}{p{11cm}|}{Has already entered the recording page using \ref{uc:12} or \ref{uc:13}} \\ \hline
Post-condition  & \multicolumn{3}{p{11cm}|}{Group of phrases is recorded} \\ \hline
Actor           & \multicolumn{3}{p{11cm}|}{Unregistered citizen} \\ \hline
Trigger         & \multicolumn{3}{p{11cm}|}{Enters the recording page} \\ \hline
Main flow of events & Step  & User action & System response \\ \hline
                    & 1     & Repeats the recording procedure in \ref{uc:12} 6 times & Is redirected to the dashboard \\ \hline
Exceptions      & \multicolumn{3}{p{11cm}|}{None} \\ \hline
\end{tabular}
\caption*{Source: Author}
\end{table}

\begin{table}[h]
\centering
\caption{UC15 - Skip Phrase}
\label{uc:15}
\begin{tabular}{|p{3cm}|p{1cm}|p{5cm}|p{5cm}|}
\hline
Use Case:       & \multicolumn{3}{p{11cm}|}{Skip Phrase} \\ \hline
Use Case ID     & \multicolumn{3}{p{11cm}|}{UC15} \\ \hline
Description     & \multicolumn{3}{p{11cm}|}{A citizen would like to skip a phrase when they (1) do not know how to pronounce, or (2) find a foreign word, or (3) find another specified reason, so that they only speak the correct phrase. A maximum of 2 skips are allowed.} \\ \hline
Rationale       & \multicolumn{3}{p{11cm}|}{Properly speak all proposed phrases} \\ \hline
Precondition    & \multicolumn{3}{p{11cm}|}{Is in recording page} \\ \hline
Post-condition  & \multicolumn{3}{p{11cm}|}{Current phrase is skipped, new phrase is displayed} \\ \hline
Actor           & \multicolumn{3}{p{11cm}|}{Citizen} \\ \hline
Trigger         & \multicolumn{3}{p{11cm}|}{Enters the recording page} \\ \hline
Main flow of events & Step  & User action & System response \\ \hline
                    & 1     & Press skip button & Confirmation window is shown \\ \hline
                    & 2     & Select reason of skipping phrase (or write one, if not specified) & Enable submission \\ \hline
                    & 3     & Confirm submission & Close confirmation window and skip phrase \\ \hline
Exceptions      & \multicolumn{3}{p{11cm}|}{If the user skips two phrases, the skip button is disabled. If the user cancels skipping the phrase, he can resume the recording process.} \\ \hline
\end{tabular}
\caption*{Source: Author}
\end{table}

\begin{table}[h]
\centering
\caption{UC16 - Stop Recording Session}
\label{uc:16}
\begin{tabular}{|p{3cm}|p{1cm}|p{5cm}|p{5cm}|}
\hline
Use Case:       & \multicolumn{3}{p{11cm}|}{Stop Recording Session} \\ \hline
Use Case ID     & \multicolumn{3}{p{11cm}|}{UC16} \\ \hline
Description     & \multicolumn{3}{p{11cm}|}{A citizen would like to stop this recording session by clicking the logo and confirming the exit, so that he can resume it afterwards.} \\ \hline
Rationale       & \multicolumn{3}{p{11cm}|}{It is no longer convenient to proceed with the recording} \\ \hline
Precondition    & \multicolumn{3}{p{11cm}|}{Is in recording page} \\ \hline
Post-condition  & \multicolumn{3}{p{11cm}|}{Session is recorded and may be resumed later} \\ \hline
Actor           & \multicolumn{3}{p{11cm}|}{Registered Citizen} \\ \hline
Trigger         & \multicolumn{3}{p{11cm}|}{Enters the recording page} \\ \hline
Main flow of events & Step  & User action & System response \\ \hline
                    & 1     & Press home button & Confirmation window is shown \\ \hline
                    & 2     & Confirm exit & Save session and redirect to dashboard \\ \hline
Exceptions      & \multicolumn{3}{p{11cm}|}{If the user does not confirm their exit, return to the recording page. If the user closes the browser, the session is also saved and can be resumed later.} \\ \hline
\end{tabular}
\caption*{Source: Author}
\end{table}


\clearpage
\subsection{Dashboard Use Cases}

\begin{table}[h]
\centering
\caption{UC20 - Recommend Themes}
\label{uc:20}
\begin{tabular}{|p{3cm}|p{1cm}|p{5cm}|p{5cm}|}
\hline
Use Case:       & \multicolumn{3}{p{11cm}|}{View Actions} \\ \hline
Use Case ID     & \multicolumn{3}{p{11cm}|}{UC20} \\ \hline
Description     & \multicolumn{3}{p{11cm}|}{As soon as the citizen enters the dashboard, they would like to see a curated suggestion of themes for them to speak, so that they contribute their voices to the Fale Alguma Coisa dataset. These themes are ordered by newest added to oldest, and if the user is in the middle of one recording session, this theme should be the first} \\ \hline
Rationale       & \multicolumn{3}{p{11cm}|}{Suggest voice recording as first option in the dashboard} \\ \hline
Precondition    & \multicolumn{3}{p{11cm}|}{None} \\ \hline
Post-condition  & \multicolumn{3}{p{11cm}|}{Themes are displayed} \\ \hline
Actor           & \multicolumn{3}{p{11cm}|}{Citizen} \\ \hline
Trigger         & \multicolumn{3}{p{11cm}|}{Enters the homepage logged} \\ \hline
Main flow of events & Step  & User action & System response \\ \hline
                    & 1     & As the user logs in & They are redirected to the dashboard \\ \hline
                    & 2     & They choose a theme to record & They are redirected to the recording page as \ref{uc:13}. \\ \hline
Exceptions      & \multicolumn{3}{p{11cm}|}{} \\ \hline
\end{tabular}
\caption*{Source: Author}
\end{table}

\begin{table}[h]
\centering
\caption{UC21 - View Dashboard Menu}
\label{uc:21}
\begin{tabular}{|p{3cm}|p{1cm}|p{5cm}|p{5cm}|}
\hline
Use Case:       & \multicolumn{3}{p{11cm}|}{View Dashboard Menu} \\ \hline
Use Case ID     & \multicolumn{3}{p{11cm}|}{UC21} \\ \hline
Description     & \multicolumn{3}{p{11cm}|}{A citizen would like to see their actions in their dashboard (such as notifications, themes, user data manipulation, leaderboards), so that they can better contribute to the project. If they are not logged, actions are limited, for example they cant see the leaderboard.} \\ \hline
Rationale       & \multicolumn{3}{p{11cm}|}{Suggest actions to the user} \\ \hline
Precondition    & \multicolumn{3}{p{11cm}|}{None} \\ \hline
Post-condition  & \multicolumn{3}{p{11cm}|}{Actions are displayed} \\ \hline
Actor           & \multicolumn{3}{p{11cm}|}{Citizen} \\ \hline
Trigger         & \multicolumn{3}{p{11cm}|}{Enters the dashboard} \\ \hline
Main flow of events & Step  & User action & System response \\ \hline
                    & 1     & Press the menu button & The user is shown all actions enabled for them \\ \hline
Exceptions      & \multicolumn{3}{p{11cm}|}{If the user is not logged, they are displayed a call to action to register his data.} \\ \hline
\end{tabular}
\caption*{Source: Author}
\end{table}

\clearpage
\subsection{Social Use Cases}

\begin{table}[h]
\centering
\caption{UC30 - Add Friend}
\label{uc:30}
\begin{tabular}{|p{3cm}|p{1cm}|p{5cm}|p{5cm}|}
\hline
Use Case:       & \multicolumn{3}{p{11cm}|}{Add Friend} \\ \hline
Use Case ID     & \multicolumn{3}{p{11cm}|}{UC30} \\ \hline
Description     & \multicolumn{3}{p{11cm}|}{A registered citizen would like to add a friend, so that they can check them in the friends leaderboard afterwards. The added user will receive a notification allowing they to be added back.} \\ \hline
Rationale       & \multicolumn{3}{p{11cm}|}{Interact with other friends} \\ \hline
Precondition    & \multicolumn{3}{p{11cm}|}{None} \\ \hline
Post-condition  & \multicolumn{3}{p{11cm}|}{Friend is added to the friend leaderboard} \\ \hline
Actor           & \multicolumn{3}{p{11cm}|}{Registered citizen} \\ \hline
Trigger         & \multicolumn{3}{p{11cm}|}{Enters the dashboard} \\ \hline
Main flow of events & Step  & User action & System response \\ \hline
                    & 1     & Press the menu button & The user is shown all actions enabled for them \\ \hline
                    & 2     & Press the search for friends button & Search Friends page is displayed \\ \hline
                    & 3     & Type in the user nickname in the search field & Found friends are listed \\ \hline
                    & 4     & Click to add a friend & Friend is added and success message is shown \\ \hline
Exceptions      & \multicolumn{3}{p{11cm}|}{If no users are found, message is displayed.} \\ \hline
\end{tabular}
\caption*{Source: Author}
\end{table}

\begin{table}[h]
\centering
\caption{UC31 - View Notifications}
\label{uc:31}
\begin{tabular}{|p{3cm}|p{1cm}|p{5cm}|p{5cm}|}
\hline
Use Case:       & \multicolumn{3}{p{11cm}|}{View Notifications} \\ \hline
Use Case ID     & \multicolumn{3}{p{11cm}|}{UC31} \\ \hline
Description     & \multicolumn{3}{p{11cm}|}{A registered citizen would like to check notifications, so that he can know what happened when he was away. For instance, if he was added by a friend, he will receive such notification.} \\ \hline
Rationale       & \multicolumn{3}{p{11cm}|}{Check latest news} \\ \hline
Precondition    & \multicolumn{3}{p{11cm}|}{None} \\ \hline
Post-condition  & \multicolumn{3}{p{11cm}|}{Notifications displayed} \\ \hline
Actor           & \multicolumn{3}{p{11cm}|}{Registered citizen} \\ \hline
Trigger         & \multicolumn{3}{p{11cm}|}{Enters the dashboard} \\ \hline
Main flow of events & Step  & User action & System response \\ \hline
                    & 1     & Press the notifications button & Redirected to the notifications page \\ \hline
Exceptions      & \multicolumn{3}{p{11cm}|}{If no notifications are found, message is displayed.} \\ \hline
\end{tabular}
\caption*{Source: Author}
\end{table}

\begin{table}[h]
\centering
\caption{UC32 - Refer Friends}
\label{uc:32}
\begin{tabular}{|p{3cm}|p{1cm}|p{5cm}|p{5cm}|}
\hline
Use Case:       & \multicolumn{3}{p{11cm}|}{Refer Friends} \\ \hline
Use Case ID     & \multicolumn{3}{p{11cm}|}{UC32} \\ \hline
Description     & \multicolumn{3}{p{11cm}|}{A registered citizen would like to refer friends to the application, so that they can play with them afterwards.} \\ \hline
Rationale       & \multicolumn{3}{p{11cm}|}{Suggest application usage by external users} \\ \hline
Precondition    & \multicolumn{3}{p{11cm}|}{None} \\ \hline
Post-condition  & \multicolumn{3}{p{11cm}|}{User added} \\ \hline
Actor           & \multicolumn{3}{p{11cm}|}{Registered citizen} \\ \hline
Trigger         & \multicolumn{3}{p{11cm}|}{Enters the dashboard} \\ \hline
Main flow of events & Step  & User action & System response \\ \hline
                    & 1     & Press the menu button & Menu is displayed \\ \hline
                    & 2     & Press the refer friends button & Refer friends page is displayed \\ \hline
                    & 3     & Click on the share button & The share functionality will be triggered, containing a message inviting other players to register on the Fale Alguma Coisa webapp. \\ \hline
Exceptions      & \multicolumn{3}{p{11cm}|}{None} \\ \hline
\end{tabular}
\caption*{Source: Author}
\end{table}

\clearpage
\subsection{Gamification Use Cases}

\begin{table}[h]
\centering
\caption{UC40 - Earn First Recordings}
\label{uc:40}
\begin{tabular}{|p{3cm}|p{1cm}|p{5cm}|p{5cm}|}
\hline
Use Case:       & \multicolumn{3}{p{11cm}|}{Earn First Recordings} \\ \hline
Use Case ID     & \multicolumn{3}{p{11cm}|}{UC40} \\ \hline
Description     & \multicolumn{3}{p{11cm}|}{A citizen would like to get points when they finish their first recordings (first recording and first theme), earning 100 and 400 points in the application, so that they can engage better with the initiative.} \\ \hline
Rationale       & \multicolumn{3}{p{11cm}|}{Improve engagement} \\ \hline
Precondition    & \multicolumn{3}{p{11cm}|}{First time recording} \\ \hline
Post-condition  & \multicolumn{3}{p{11cm}|}{Points added} \\ \hline
Actor           & \multicolumn{3}{p{11cm}|}{Citizen} \\ \hline
Trigger         & \multicolumn{3}{p{11cm}|}{Enters the recording page} \\ \hline
Main flow of events & Step  & User action & System response \\ \hline
                    & 1     & Finishes their first recording & User earns 100 points and message is displayed  \\ \hline
                    & 2     & Finishes their first 6 recordings (theme) & User earns 400 points and message is displayed \\ \hline
Exceptions      & \multicolumn{3}{p{11cm}|}{None} \\ \hline
\end{tabular}
\caption*{Source: Author}
\end{table}

\begin{table}[h]
\centering
\caption{UC41 - Earn Subsequent Themes}
\label{uc:41}
\begin{tabular}{|p{3cm}|p{1cm}|p{5cm}|p{5cm}|}
\hline
Use Case:       & \multicolumn{3}{p{11cm}|}{Earn Subsequent Themes} \\ \hline
Use Case ID     & \multicolumn{3}{p{11cm}|}{UC41} \\ \hline
Description     & \multicolumn{3}{p{11cm}|}{A citizen would like to get points when they finish their subsequent themes earning 300 points, so that they can engage better with the initiative.} \\ \hline
Rationale       & \multicolumn{3}{p{11cm}|}{Improve engagement} \\ \hline
Precondition    & \multicolumn{3}{p{11cm}|}{Already recorded the first theme} \\ \hline
Post-condition  & \multicolumn{3}{p{11cm}|}{Points added} \\ \hline
Actor           & \multicolumn{3}{p{11cm}|}{Citizen} \\ \hline
Trigger         & \multicolumn{3}{p{11cm}|}{Enters the recording page} \\ \hline
Main flow of events & Step  & User action & System response \\ \hline
                    & 1     & Finishes recording the theme & User earns 300 points and message is displayed \\ \hline
Exceptions      & \multicolumn{3}{p{11cm}|}{None} \\ \hline
\end{tabular}
\caption*{Source: Author}
\end{table}

\begin{table}[h]
\centering
\caption{UC42 - Earn Registration}
\label{uc:42}
\begin{tabular}{|p{3cm}|p{1cm}|p{5cm}|p{5cm}|}
\hline
Use Case:       & \multicolumn{3}{p{11cm}|}{Earn Registration} \\ \hline
Use Case ID     & \multicolumn{3}{p{11cm}|}{UC42} \\ \hline
Description     & \multicolumn{3}{p{11cm}|}{A registered citizen would like to get 500 points when they register their speaker data, so that he can better engage with the application. This can happen in three places: (1) clicking on registration on the login page, (2) clicking on registration on the dashboard page.} \\ \hline
Rationale       & \multicolumn{3}{p{11cm}|}{Improve dataset metadata available} \\ \hline
Precondition    & \multicolumn{3}{p{11cm}|}{Never added their data before} \\ \hline
Post-condition  & \multicolumn{3}{p{11cm}|}{Points added} \\ \hline
Actor           & \multicolumn{3}{p{11cm}|}{Unregistered Citizen} \\ \hline
Trigger         & \multicolumn{3}{p{11cm}|}{Many, refer to description} \\ \hline
Main flow of events & Step  & User action & System response \\ \hline
                    & 1     & Selects their nickname & Shows next step of personal data \\ \hline
                    & 2     & Fills form with gender, age interval, region of origin and dialect & Shows next step of username and password \\ \hline
                    & 3     & Fills username and password with confirmation & User has data saved and points awarded \\ \hline
Exceptions      & \multicolumn{3}{p{11cm}|}{If the user was already logged in with social login, they skip step 3.} \\ \hline
\end{tabular}
\caption*{Source: Author}
\end{table}

\begin{table}[h]
\centering
\caption{UC43 - Calculate Level}
\label{uc:43}
\begin{tabular}{|p{3cm}|p{1cm}|p{5cm}|p{5cm}|}
\hline
Use Case:       & \multicolumn{3}{p{11cm}|}{Calculate Level} \\ \hline
Use Case ID     & \multicolumn{3}{p{11cm}|}{UC43} \\ \hline
Description     & \multicolumn{3}{p{11cm}|}{A registered citizen would like to measure his points through a level, so that they can more easily compare themselves with other users.} \\ \hline
Rationale       & \multicolumn{3}{p{11cm}|}{Compare with other users} \\ \hline
Precondition    & \multicolumn{3}{p{11cm}|}{None} \\ \hline
Post-condition  & \multicolumn{3}{p{11cm}|}{None} \\ \hline
Actor           & \multicolumn{3}{p{11cm}|}{Citizen} \\ \hline
Trigger         & \multicolumn{3}{p{11cm}|}{Enters the dashboard} \\ \hline
Main flow of events & Step  & User action & System response \\ \hline
                    & 1     & Enters the dashboard & Displays a level indicator, as well as a progress bar, showing how many points to the next level \\ \hline
Exceptions      & \multicolumn{3}{p{11cm}|}{None} \\ \hline
\end{tabular}
\caption*{Source: Author}
\end{table}

\begin{table}[h]
\centering
\caption{UC44 - Competition}
\label{uc:44}
\begin{tabular}{|p{3cm}|p{1cm}|p{5cm}|p{5cm}|}
\hline
Use Case:       & \multicolumn{3}{p{11cm}|}{Competition} \\ \hline
Use Case ID     & \multicolumn{3}{p{11cm}|}{UC44} \\ \hline
Description     & \multicolumn{3}{p{11cm}|}{A registered citizen would like to know who are the top contributors among all players or only their friends, so that they can compete against them.} \\ \hline
Rationale       & \multicolumn{3}{p{11cm}|}{Compare with other users} \\ \hline
Precondition    & \multicolumn{3}{p{11cm}|}{None} \\ \hline
Post-condition  & \multicolumn{3}{p{11cm}|}{None} \\ \hline
Actor           & \multicolumn{3}{p{11cm}|}{Citizen} \\ \hline
Trigger         & \multicolumn{3}{p{11cm}|}{Enters the dashboard} \\ \hline
Main flow of events & Step  & User action & System response \\ \hline
                    & 1     & Clicks on menu & Shows menu \\ \hline
                    & 1     & Clicks on Leaderboard button (either Overall or Friends) & Shows selected leaderboard \\ \hline
Exceptions      & \multicolumn{3}{p{11cm}|}{None} \\ \hline
\end{tabular}
\caption*{Source: Author}
\end{table}

\clearpage
\subsection{User Data Use Cases}

\begin{table}[h]
\centering
\caption{UC50 - Social Login}
\label{uc:50}
\begin{tabular}{|p{3cm}|p{1cm}|p{5cm}|p{5cm}|}
\hline
Use Case:       & \multicolumn{3}{p{11cm}|}{Social Login} \\ \hline
Use Case ID     & \multicolumn{3}{p{11cm}|}{UC50} \\ \hline
Description     & \multicolumn{3}{p{11cm}|}{An unregistered citizen would like to login (using social login - Facebook / Google) on the app, so that they can login later and save my progress.} \\ \hline
Rationale       & \multicolumn{3}{p{11cm}|}{Delegation of the authentication process} \\ \hline
Precondition    & \multicolumn{3}{p{11cm}|}{None} \\ \hline
Post-condition  & \multicolumn{3}{p{11cm}|}{None} \\ \hline
Actor           & \multicolumn{3}{p{11cm}|}{Citizen} \\ \hline
Trigger         & \multicolumn{3}{p{11cm}|}{Enters the homepage} \\ \hline
Main flow of events & Step  & User action & System response \\ \hline
                    & 1     & Clicks on login & Shows login page \\ \hline
                    & 2     & Clicks on Facebook or Google button & Starts social login process \\ \hline
                    & 3     & Finishes login process within Facebook or Google & Application is now logged in \\ \hline
Exceptions      & \multicolumn{3}{p{11cm}|}{None} \\ \hline
\end{tabular}
\caption*{Source: Author}
\end{table}

\begin{table}[h]
\centering
\caption{UC51 - Update Account}
\label{uc:51}
\begin{tabular}{|p{3cm}|p{1cm}|p{5cm}|p{5cm}|}
\hline
Use Case:       & \multicolumn{3}{p{11cm}|}{Update Account} \\ \hline
Use Case ID     & \multicolumn{3}{p{11cm}|}{UC51} \\ \hline
Description     & \multicolumn{3}{p{11cm}|}{A registered citizen would like to update his account data, so that they can provide accurate metadata on the recordings.} \\ \hline
Rationale       & \multicolumn{3}{p{11cm}|}{Incorrect metadata provided} \\ \hline
Precondition    & \multicolumn{3}{p{11cm}|}{None} \\ \hline
Post-condition  & \multicolumn{3}{p{11cm}|}{None} \\ \hline
Actor           & \multicolumn{3}{p{11cm}|}{Citizen} \\ \hline
Trigger         & \multicolumn{3}{p{11cm}|}{Enters the dashboard} \\ \hline
Main flow of events & Step  & User action & System response \\ \hline
                    & 1     & Clicks on menu & Menu opens \\ \hline
                    & 2     & Clicks on update account button & Displays account update form \\ \hline
                    & 3     & Fills forms with new data & User account updated \\ \hline
Exceptions      & \multicolumn{3}{p{11cm}|}{None} \\ \hline
\end{tabular}
\caption*{Source: Author}
\end{table}

\begin{table}[h]
\centering
\caption{UC52 - Delete Account}
\label{uc:52}
\begin{tabular}{|p{3cm}|p{1cm}|p{5cm}|p{5cm}|}
\hline
Use Case:       & \multicolumn{3}{p{11cm}|}{Delete Account} \\ \hline
Use Case ID     & \multicolumn{3}{p{11cm}|}{UC52} \\ \hline
Description     & \multicolumn{3}{p{11cm}|}{A registered citizen would like to remove his account data (and their recordings, if wanted), so that they can remove my metadata from this application.} \\ \hline
Rationale       & \multicolumn{3}{p{11cm}|}{Incorrect metadata provided} \\ \hline
Precondition    & \multicolumn{3}{p{11cm}|}{None} \\ \hline
Post-condition  & \multicolumn{3}{p{11cm}|}{None} \\ \hline
Actor           & \multicolumn{3}{p{11cm}|}{Citizen} \\ \hline
Trigger         & \multicolumn{3}{p{11cm}|}{Enters the dashboard} \\ \hline
Main flow of events & Step  & User action & System response \\ \hline
                    & 1     & Clicks on menu & Menu opens \\ \hline
                    & 2     & Clicks on delete account button & Displays confirmation for account deletion \\ \hline
                    & 3     & Confirm deletion & Shows selection form on whether to keep user recordings or not\\ \hline
                    & 4     & Select deletion of metadata & System prompts the reason of account deletion \\ \hline
                    & 5     & User confirms deletion (they may or may not provide a reason) &  of metadata & Account is deleted from the system, with recordings deleted if asked \\ \hline
Exceptions      & \multicolumn{3}{p{11cm}|}{The user may leave the account deletion flow in any step (except for the last one) of the process.} \\ \hline
\end{tabular}
\caption*{Source: Author}
\end{table}


\clearpage
\subsection{Class models}
\label{appendix:srs:domain-model}

Understanding the domain of the proposed application will assist in creating a better data structure. Below are some examples using class diagrams of the most common objects used by the system, such as recordings and phrases.

\subsection{User Recording}

Represents the all data generated by the user, containing the following elements below:

\begin{itemize}
    \item metadata, if the user registered it
    \item notifications, if the user has received any
    \item themes, if the user has at least recorded one phrase (just groups the phrases)
    \item phrase, if the user has at least recorded one phrase
\end{itemize}

\begin{figure}
    \caption{Class diagram for the UserRecording}
    \label{fig:userrecording-class-diagram}
    \begin{center}
    \begin{tikzpicture}
    \begin{umlpackage}{User}
    \umlclass[y=-10]{UserRecording}{}{}
    \umlclass[x=5, y=-5]{UserData}{
        firebaseId : string \\
        nickname : string \\
        gender : string \\
        ageInterval : string \\
        region : string \\
        dialect : string \\
        referralCode : string
    }{}
    \umlclass[x=5,y=-10]{UserNotification}{
        type : string \\
        dismissed : boolean \\
        scoringId : string \\
        name : string
    }{}
    \umlclass[x=5,y=-15]{RecordingTheme}{
        title : string \\
        finished : boolean \\
        recordings : RecordingPhrase[]
    }{}
    \umlclass[x=12,y=-15]{RecordingPhrase}{
        phraseId : string \\
        sampleRate : int \\
        format : string \\
        duration : string \\
        recordingPath : string \\
        skippedReason : string
    }{}
    
    \umlassoc[geometry=|-, mult=1, pos=1.9]{UserRecording}{UserData}
    \umlassoc[geometry=-|, mult=*, pos=0.3]{UserRecording}{UserNotification}
    \umlassoc[geometry=|-, mult=*, pos=1.9]{UserRecording}{RecordingTheme}
    \umlassoc[geometry=|-, mult=*, pos=1.9]{RecordingTheme}{RecordingPhrase}
    \end{umlpackage}
    \end{tikzpicture}
    \end{center}
    \caption*{Source: Author}
\end{figure}

\subsection{Behavioral models}
\label{appendix:srs:behavioral-models}

The inclusion of behavioral models is not deemed necessary to this specification.

\subsection{Flow models}
\label{appendix:srs:flow-models}

The inclusion of flow models is not deemed necessary to this specification.

\section{Design Model}
\label{appendix:srs:design-models}

\section{User Interface Design}
\label{appendix:user-interface-design}

All user interfaces designed are available in the Zeplin interface. Refer to the link \url{https://zpl.io/a3WMkjW} for the complete list.

\section{Nonfunctional Requirements}
\label{appendix:srs:nonfunctional-requirements}

The nonfunctional requirements for a system are typically constraints on the functional requirements – that is, not what the system does, but how it does it (e.g. how quickly, how efficiently, how easily a task is achieved from the user’s perspective, etc.).

Other nonfunctional requirements may be required characteristics that are not part of the system’s functionality, e.g., conformance with legal requirements, scalability, interoperability, etc. They are specified below.

\subsection{System Requirements}

A web application will be developed. If possible, compatible with all modern browsers (Chrome, Firefox, Safari, Edge), in mobile and desktop versions.

\subsection{Usability Requirements}

This system is aimed towards nonprofessionals, so there should be little to no technical jargon present.

\subsection{Performance Requirements}

There are no performance requirements.

\subsection{Security Requirements}

For the WebApp, the logged area must be protected by an authorization layer. The other areas, such as homepage, terms of service and recording page are not necessarily protected.

All the functionality of the backend must be protected from public access. If necessary, the user must provide an anonymous token to vouch for authenticity.

The database must not be accessible from the public, only by the backend or local access.

\subsection{Delivery Requirements}

The application should be delivered to a local USP server, since it will host all applications and data.

\subsection{Legal Requirements}

This research relies on the fact that if consent is given by the user, user data can be collected and utilized for scientific purposes. This is one of the articles of the LGPD law.

\subsection{Interoperability Requirements}

To provide authentication, this system will interact with the authorization provider, Firebase. For logging purposes, Sentry is integrated with the WebApp and the Backend.

\subsection{Scalability Requirements}

The Fale Alguma Coisa has no specific scalability requirements.