\chapter{Software Requirements Specification}
\label{appendix:srs}

\section{Introduction}
\subsection{Purpose}

The purpose of this Requirements Specification is to document the software requirements of Fale Alguma Coisa, a citizen science voice recording app. It should be used by anyone with the intent of understanding what are the capabilities of the system. This document will not focus on how the system will be built, but serves as the basis for the design and implementation. The design is documented within the Design Model in appendix \ref{appendix:design-model}. Details on implementation are documented in the corresponding Software Design Description (SDD) document in the appendix \ref{appendix:software-design-description}.

\subsection{Scope}

\subsubsection{Objectives}

Towards contributing \textbf{nonprofessional scientists}, the Fale Alguma Coisa app should provide an easy gateway for the user to contribute his voice while having fun and learning various science facts and curiosities.

Towards researching \textbf{scientists}, the Fale Alguma Coisa app should provide a database of anonymized voice recordings, for scientists to extract and create speech corpus.

\subsubsection{Outside scope}

However, some elements are outside of the scope of this system. Fale Alguma Coisa should \textbf{not}:
\begin{itemize}
    \item Allow for association of recording data and personal identification data (name, email);
    \item Support internationalization in the WebApp;
    \item Convert audio data into another format
    \item Support offline recording
\end{itemize}

\subsubsection{Artifacts}

This document identifies the following artifacts to be produced:

\begin{itemize}
    \item Context
    \item Domain Model
    \item Use Case Requirements (Actors, Diagrams, Specifications)
    \item Activity Diagrams
    \item Non-Functional Requirements
    \item Interface Requirements
    \item WebApp for user recordings
    \item Backend integrating with database
    \item Database structure
\end{itemize}

\subsection{Overview of Document}

This overview provides a summary of the contents of each section of this document.

\begin{itemize}
    \item Introduction
    \item System Overview
    \item Domain Model
    \item Requirements
\end{itemize}

\subsection{References}

\begin{itemize}
    \item Based on: \cite{naoufel2014requirements}
    \item Supporting literature: \cite{pressman2014software}
\end{itemize}

\section{System Overview}

The System Overview section presents the system context and design, and also discusses the background of the Fale Alguma Coisa project.

\subsection{Name}

This system is called "Fale Alguma Coisa". This is a phrase in brazillian portuguese to order someone to "say something". It is also common to hear "fala alguma coisa", but this was changed to remove the alliteration effect, providing easier spelling.

\subsection{System Context}

This system is originated from the need of recording anonymized user voices throughout the use of gamification and citizen science.

\subsection{General Constraints}

This system is mainly impacted by constraints in data privacy. Recordings should never be associated with the users personal data (such as email), but should have metadata to aid in corpus construction, such as age interval, gender, and dialect.

\subsection{Assumptions and Dependencies}

The system heavily depends on the phrases that will be spoken. Therefore, it is necessary to provide this application with a comprehensive list of phrases before the application general publication.

\section{Requirements}

To accurately describe the software requirements of a project, it is important to understand the requirements from many different perspectives. The analysis model proposed by \cite{pressman2014software} encompasses four different modeling elements (perspectives) that can be used to provide the most effective bridge to software design. They are listed in the subsections below as (1) scenario based models, (2) class-based elements model, (3) behavioral elements, and (4) flow-oriented elements. The analysis can lead to the derivation of one or more of these modeling elements, since one model may not completely describe certain functionalities of the app. Additionally, the actor list will be provided in the table \ref{tab:actor-list} below:

\begin{table}[h]
\centering
\caption{WebApp Fale Alguma Coisa Actor List}
\label{tab:actor-list}
\begin{tabular}{|p{2.5cm}|p{3.5cm}|p{7cm}|}
    \hline 
    Actor & Responsibilities & Description  \\ \hline
    Unregistered Citizen & Record phrases & User that has not logged on in the WebApp, and has not added his metadata \\ \hline
    Registered Citizen & Record phrases with metadata & User that has added his metadata on the WebApp \\ \hline 
    Validation Scientist & Validate recorded phrases & Download and validate recordings \\ \hline
    Corpus User & Download validated phrases & Access the open speech corpus and download it \\ \hline
\end{tabular}
\caption*{Source: Author}
\end{table}

\subsection{Scenario-based models}

This model depicts how the user interacts with the system and the specific sequence of activities that occur as the software is used. One common implementation for this specification is through use cases. To provide scenario-model completeness, the following sections will describe every use case mapped to the WebApp. They are, however, not extensive to the functionality of the application, some use cases are complemented by the models cited above. Each of the use cases are categorized by feature. Furthermore, each feature is described in the table \ref{tab:falealgumacoisa-features}.

\begin{table}[h]
    \centering
    \caption{WebApp Fale Alguma Coisa Features}
    \label{tab:falealgumacoisa-features}
    \begin{tabular}{|p{3cm}|p{10cm}|}
        \hline Epic & Description \\
        \hline Home & Represents the use cases affecting the homepage, such as the splash screen, call to action button, terms of service, etc. \\
        \hline Recording & Represents the most important feature in the application, as the citizen will use it to record his voice. It also provides supporting features, such as skipping phrases and resuming the recording session.\\ 
        \hline Dashboard & Lists all user stories related to the dashboard page, such as where the user will be able to choose themes to record, open the menu, check his level, etc. \\
        \hline Gamification & Details the engagement component of the application. Elements such as leaderboards, points and levels are described. They add a sense of progress to the user experience. \\
        \hline Social & To allow social interaction with other users, this feature lists social features to be added to the application. \\
        \hline Login and Registration & The application should provide user authentication to enable data management and progress saving. \\
        \hline
    \end{tabular}
    \caption*{Source: Author}
\end{table}

\subsubsection{UC01 - View Home}

\begin{table}[h]
\centering
\caption{UC01 - View Home}
\label{uc:01}
\begin{tabular}{|p{3cm}|p{1cm}|p{5cm}|p{5cm}|}
\hline
Use Case:       & \multicolumn{3}{p{11cm}|}{View Home} \\ \hline
Use Case ID     & \multicolumn{3}{p{11cm}|}{UC01} \\ \hline
Description     & \multicolumn{3}{p{11cm}|}{After entering the application, the unregistered citizen is shown an quick animated introductory screen (splash). After the splash, the homepage is displayed with information on the Fale Alguma Coisa citizen science project, as well as statistics on the progress of the recordings and contributors.} \\ \hline
Rationale       & \multicolumn{3}{p{11cm}|}{So that they feel more inside a native app and know more about the initiative} \\ \hline
Precondition    & \multicolumn{3}{p{11cm}|}{None} \\ \hline
Post-condition  & \multicolumn{3}{p{11cm}|}{Application shows the home page} \\ \hline
Actor           & \multicolumn{3}{p{11cm}|}{Unregistered citizen} \\ \hline
Trigger         & \multicolumn{3}{p{11cm}|}{Enters the application} \\ \hline
Main flow of events & Step  & User action & System response \\ \hline
                    & 1     & User enters the application & A full screen image with the logo, title and description of the app is shown \\ \hline
                    & 2     & User waits for a short period of time (<200ms) & The homepage is displayed, with informations and statistics \\ \hline
Exceptions      & \multicolumn{3}{p{11cm}|}{None} \\ \hline
\end{tabular}
\caption*{Source: Author}
\end{table}

\subsubsection{UC02 - Read Home Terms or Privacy}

\begin{table}[h]
\centering
\caption{UC02 - Read Home Terms or Privacy}
\label{uc:02}
\begin{tabular}{|p{3cm}|p{1cm}|p{5cm}|p{5cm}|}
\hline
Use Case:       & \multicolumn{3}{p{11cm}|}{Read Home Terms or Privacy} \\ \hline
Use Case ID     & \multicolumn{3}{p{11cm}|}{UC02} \\ \hline
Description     & \multicolumn{3}{p{11cm}|}{In the homepage, should the user want to read the terms of service or privacy policy of the Fale Alguma Coisa app, a simple link in the footer is displayed.} \\ \hline
Rationale       & \multicolumn{3}{p{11cm}|}{The user understands what he is being subjected when using the service, as well how his data is being used} \\ \hline
Precondition    & \multicolumn{3}{p{11cm}|}{None} \\ \hline
Post-condition  & \multicolumn{3}{p{11cm}|}{Terms of service or privacy policy is shown} \\ \hline
Actor           & \multicolumn{3}{p{11cm}|}{Citizen, not necessarily logged} \\ \hline
Trigger         & \multicolumn{3}{p{11cm}|}{Clicks the footer link} \\ \hline
Main flow of events & Step  & User action & System response \\ \hline
                    & 1     & User enters the application & Splash animation is shown \\ \hline
                    & 2     & User waits for animation & Home page is displayed \\ \hline
                    & 3     & User clicks on the "Terms of Service" or "Privacy Policy" link in the footer & Page is displayed with information\\ \hline
Exceptions      & \multicolumn{3}{p{11cm}|}{None} \\ \hline
\end{tabular}
\caption*{Source: Author}
\end{table}


\subsubsection{UC03 - Navigate Login}

\begin{table}[h]
\centering
\caption{UC03 - Navigate Login}
\label{uc:03}
\begin{tabular}{|p{3cm}|p{1cm}|p{5cm}|p{5cm}|}
\hline
Use Case:       & \multicolumn{3}{p{11cm}|}{Navigate Login} \\ \hline
Use Case ID     & \multicolumn{3}{p{11cm}|}{UC03} \\ \hline
Description     & \multicolumn{3}{p{11cm}|}{A registered citizen wants to navigates from the home-page to the sign-in page, so that they login (or register) to their account} \\ \hline
Rationale       & \multicolumn{3}{p{11cm}|}{Access to the registered application} \\ \hline
Precondition    & \multicolumn{3}{p{11cm}|}{User is registered to the Fale Alguma Coisa app} \\ \hline
Post-condition  & \multicolumn{3}{p{11cm}|}{User reaches login page} \\ \hline
Actor           & \multicolumn{3}{p{11cm}|}{Registered citizen} \\ \hline
Trigger         & \multicolumn{3}{p{11cm}|}{Enters the application} \\ \hline
Main flow of events & Step  & User action & System response \\ \hline
                    & 1     & User enters the application & Splash animation is shown \\ \hline
                    & 2     & User clicks the login button in the header & Login page is displayed \\ \hline
Exceptions      & \multicolumn{3}{p{11cm}|}{None} \\ \hline
\end{tabular}
\caption*{Source: Author}
\end{table}


\subsubsection{UC10 - Accept Terms}

\begin{table}[h]
\centering
\caption{UC10 - Accept Terms}
\label{uc:10}
\begin{tabular}{|p{3cm}|p{1cm}|p{5cm}|p{5cm}|}
\hline
Use Case:       & \multicolumn{3}{p{11cm}|}{Accept Terms} \\ \hline
Use Case ID     & \multicolumn{3}{p{11cm}|}{UC10} \\ \hline
Description     & \multicolumn{3}{p{11cm}|}{An unregistered citizen would like to view and accept the terms of service before recording, so that he understands how his data is being used.} \\ \hline
Rationale       & \multicolumn{3}{p{11cm}|}{Transparency for the user} \\ \hline
Precondition    & \multicolumn{3}{p{11cm}|}{User is in the homepage and has not accepted the terms of service} \\ \hline
Post-condition  & \multicolumn{3}{p{11cm}|}{Acceptance of terms is recorded} \\ \hline
Actor           & \multicolumn{3}{p{11cm}|}{Unregistered citizen} \\ \hline
Trigger         & \multicolumn{3}{p{11cm}|}{Clicks the record button} \\ \hline
Main flow of events & Step  & User action & System response \\ \hline
                    & 1     & User clicks the record button & Terms of service page is displayed \\ \hline
                    & 2     & User reads and accepts the Terms of Service & Acceptance is recorded and the user is redirected to the next step of the recording flow \\ \hline
Exceptions      & \multicolumn{3}{p{11cm}|}{If the user does not accept the terms, they are redirected back to the homepage.} \\ \hline
\end{tabular}
\caption*{Source: Author}
\end{table}



\subsubsection{UC11 - Configure Microphone}

\begin{table}[h]
\centering
\caption{UC11 - Configure Microphone}
\label{uc:11}
\begin{tabular}{|p{3cm}|p{1cm}|p{5cm}|p{5cm}|}
\hline
Use Case:       & \multicolumn{3}{p{11cm}|}{Configure Microphone} \\ \hline
Use Case ID     & \multicolumn{3}{p{11cm}|}{UC11} \\ \hline
Description     & \multicolumn{3}{p{11cm}|}{An unregistered citizen would like to properly configure my microphone before recording, so that he can record without interruption.} \\ \hline
Rationale       & \multicolumn{3}{p{11cm}|}{Proper configuration of the microphone before the recording flow is started.} \\ \hline
Precondition    & \multicolumn{3}{p{11cm}|}{User has accepted the terms of service and has not yet allowed the microphone in the browser} \\ \hline
Post-condition  & \multicolumn{3}{p{11cm}|}{User redirected to the recording page} \\ \hline
Actor           & \multicolumn{3}{p{11cm}|}{Unregistered citizen} \\ \hline
Trigger         & \multicolumn{3}{p{11cm}|}{Clicks the record button} \\ \hline
Main flow of events & Step  & User action & System response \\ \hline
                    & 1     & User clicks the record button in the homepage & Configure microphone page is displayed, with explanation on how to allow it \\ \hline
                    & 2     & User user clicks to continue & System prompt for the microphone to be allowed \\ \hline
                    & 3     & User allows the usage of the microphone & User is redirected to. the recording page \\ \hline
Exceptions      & \multicolumn{3}{p{11cm}|}{If the user does not allow the microphone, they are redirected to a page detailing the app has no access to the mic} \\ \hline
\end{tabular}
\caption*{Source: Author}
\end{table}

\subsubsection{UC12 - Record Phrases From Homepage}

\begin{table}[h]
\centering
\caption{UC12 - Record Phrases From Homepage}
\label{uc:12}
\begin{tabular}{|p{3cm}|p{1cm}|p{5cm}|p{5cm}|}
\hline
Use Case:       & \multicolumn{3}{p{11cm}|}{Record Phrases From Homepage} \\ \hline
Use Case ID     & \multicolumn{3}{p{11cm}|}{UC12} \\ \hline
Description     & \multicolumn{3}{p{11cm}|}{A citizen would like to read science phrases with definitions and curiosities, so that he learns about subjects a she is contributing.} \\ \hline
Rationale       & \multicolumn{3}{p{11cm}|}{Contributing to science and learning topics} \\ \hline
Precondition    & \multicolumn{3}{p{11cm}|}{Accepted the terms of service, allowed the microphone, is in the homepage} \\ \hline
Post-condition  & \multicolumn{3}{p{11cm}|}{Phrase is recorded} \\ \hline
Actor           & \multicolumn{3}{p{11cm}|}{Unregistered citizen} \\ \hline
Trigger         & \multicolumn{3}{p{11cm}|}{Clicks the record button} \\ \hline
Main flow of events & Step  & User action & System response \\ \hline
                    & 1     & User clicks the record button in the homepage & Recording page is displayed, as well as a tutorial explanation on how to record \\ \hline
                    & 2     & User clicks and holds the red recording button & The tutorial message is changed and the page focuses on the phrase to be read \\ \hline
                    & 3     & User speaks the phrase and releases the red recording button & The phrase is locally recorded and a confirmation is shown \\ \hline
                    & 4     & User clicks on the play button & Recording is played back \\ \hline
                    & 5     & User confirms that the recording is ok & Next phrase is shown \\ \hline
Exceptions      & \multicolumn{3}{p{11cm}|}{The user may not find the recording correct. If so, he may reject the recording at step 4 and record again.} \\ \hline
\end{tabular}
\caption*{Source: Author}
\end{table}


\subsection{Class models}
\label{appendix:srs:domain-model}

Understanding the domain of the proposed application will assist in creating a better data structure. Below are some examples using class diagrams of the most common objects used by the system, such as recordings and phrases.

\subsection{User Recording}

Represents the all data generated by the user, containing the following elements below:

\begin{itemize}
    \item metadata, if the user registered it
    \item notifications, if the user has received any
    \item themes, if the user has at least recorded one phrase (just groups the phrases)
    \item phrase, if the user has at least recorded one phrase
\end{itemize}

\begin{figure}
    \caption{Class diagram for the UserRecording}
    \label{fig:userrecording-class-diagram}
    \begin{center}
    \begin{tikzpicture}
    \begin{umlpackage}{User}
    \umlclass[y=-10]{UserRecording}{}{}
    \umlclass[x=5, y=-5]{UserData}{
        firebaseId : string \\
        nickname : string \\
        gender : string \\
        ageInterval : string \\
        region : string \\
        dialect : string \\
        referralCode : string
    }{}
    \umlclass[x=5,y=-10]{UserNotification}{
        type : string \\
        dismissed : boolean \\
        scoringId : string \\
        name : string
    }{}
    \umlclass[x=5,y=-15]{RecordingTheme}{
        title : string \\
        finished : boolean \\
        recordings : RecordingPhrase[]
    }{}
    \umlclass[x=12,y=-15]{RecordingPhrase}{
        phraseId : string \\
        sampleRate : int \\
        format : string \\
        duration : string \\
        recordingPath : string \\
        skippedReason : string
    }{}
    
    \umlassoc[geometry=|-, mult=1, pos=1.9]{UserRecording}{UserData}
    \umlassoc[geometry=-|, mult=*, pos=0.3]{UserRecording}{UserNotification}
    \umlassoc[geometry=|-, mult=*, pos=1.9]{UserRecording}{RecordingTheme}
    \umlassoc[geometry=|-, mult=*, pos=1.9]{RecordingTheme}{RecordingPhrase}
    \end{umlpackage}
    \end{tikzpicture}
    \end{center}
    \caption*{Source: Author}
\end{figure}

\subsection{Behavioral models}

\subsection{Flow models}

\section{Design Model}


\section{Non-Functional Requirements}

This section lists the non-functional requirements, 

\subsection{System Requirements}

The non-functional requirements for a system are typically constraints on the functional requirements – that is, not what the system does, but how it does it (e.g. how quickly, how efficiently, how easily a task is achieved from the user’s perspective, etc.).
· Other non-functional requirements may be required characteristics that are not part of the system’s functionality, e.g., conformance with legal requirements, scalability, interoperability, etc.
· Specify each individual non-functional requirement that must be supported by the system.

\subsection{Usability Requirements}

The application is to be accessed over a browser, with a mobile device or desktop computer.

\subsection{Performance Requirements}

There are no performance requirements.

\subsection{Security Requirements}

The 

\subsection{Delivery Requirements}
\subsection{Legal Requirements}
\subsection{Interoperability Requirements}
\subsection{Scalability Requirements}