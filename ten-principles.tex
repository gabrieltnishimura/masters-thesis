\chapter{Ten principles of citizen science}
\label{app:ten-principles}

\begin{itemize}
    \item Citizen science projects actively involve citizens in scientific endeavour that generates new knowledge or understanding. 
    \subitem Citizens may act as contributors, collaborators or as project leaders and have a meaningful role in the project.
    \item Citizen science projects have a genuine science outcome. For example, answering a research question or informing conservation action, management decisions or environmental policy.
    \item Both the professional scientists and the citizen scientists benefit from taking part. Benefits may include the publication of research outputs, learning opportunities, personal enjoyment, social benefits, satisfaction through contributing to scientific evidence, for example, to address local, national and international issues, and through that, the potential to influence policy.
    \item Citizen scientists may, if they wish, participate in multiple stages of the scientific process. This may include developing the research question, designing the method, gathering and analysing data, and communicating the results.
    \item Citizen scientists receive feedback from the project. For example, how their data are being used and what the research, policy or societal outcomes are.
    \item Citizen science is considered a research approach like any other, with limitations and biases that should be considered and controlled for. However unlike traditional research approaches, citizen science provides opportunity for greater public engagement and democratisation of science.
    \item Citizen science project data and metadata are made publicly available and where possible, results are published in an open-access format. Data sharing may occur during or after the project, unless there are security or privacy concerns that prevent this.
    \item Citizen scientists are acknowledged in project results and publications.
    \item Citizen science programmes are evaluated for their scientific output, data quality, participant experience and wider societal or policy impact.
    \item The leaders of citizen science projects take into consideration legal and ethical issues surrounding copyright, intellectual property, data-sharing agreements, confidentiality, attribution and the environmental impact of any activities.
\end{itemize}

\chapter{Open Source Definition}
\label{appendix:chap:open-source-definition}

The sections below reproduce the document defining the Open Source Definition \cite{perens1999open}:

\section{Introduction}

Open source doesn't just mean access to the source code. The distribution terms of open-source software must comply with the following criteria:

\section{Free Redistribution}
The license shall not restrict any party from selling or giving away the software as a component of an aggregate software distribution containing programs from several different sources. The license shall not require a royalty or other fee for such sale.

\section{Source Code}
The program must include source code, and must allow distribution in source code as well as compiled form. Where some form of a product is not distributed with source code, there must be a well-publicized means of obtaining the source code for no more than a reasonable reproduction cost, preferably downloading via the Internet without charge. The source code must be the preferred form in which a programmer would modify the program. Deliberately obfuscated source code is not allowed. Intermediate forms such as the output of a preprocessor or translator are not allowed.

\section{Derived Works}
The license must allow modifications and derived works, and must allow them to be distributed under the same terms as the license of the original software.

\section{Integrity of The Author's Source Code}
The license may restrict source-code from being distributed in modified form only if the license allows the distribution of "patch files" with the source code for the purpose of modifying the program at build time. The license must explicitly permit distribution of software built from modified source code. The license may require derived works to carry a different name or version number from the original software.

\section{No Discrimination Against Persons or Groups}
The license must not discriminate against any person or group of persons.

\section{No Discrimination Against Fields of Endeavor}
The license must not restrict anyone from making use of the program in a specific field of endeavor. For example, it may not restrict the program from being used in a business, or from being used for genetic research.

\section{Distribution of License}
The rights attached to the program must apply to all to whom the program is redistributed without the need for execution of an additional license by those parties.

\section{License Must Not Be Specific to a Product}
The rights attached to the program must not depend on the program's being part of a particular software distribution. If the program is extracted from that distribution and used or distributed within the terms of the program's license, all parties to whom the program is redistributed should have the same rights as those that are granted in conjunction with the original software distribution.

\section{License Must Not Restrict Other Software}
The license must not place restrictions on other software that is distributed along with the licensed software. For example, the license must not insist that all other programs distributed on the same medium must be open-source software.

\section{License Must Be Technology-Neutral}
No provision of the license may be predicated on any individual technology or style of interface.