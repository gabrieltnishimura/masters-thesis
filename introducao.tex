\chapter*[Introduction]{Introduction}
\addcontentsline{toc}{chapter}{Introduction}

\pgfplotstableread[row sep=\\,col sep=&]{
    year & publications \\
    2010 & 2810 \\
    2011 & 2750 \\
    2012 & 5220 \\
    2013 & 6990 \\
    2014 & 9330 \\
    2015 & 11900 \\
    2016 & 14500 \\
    2017 & 15800 \\
    2018 & 20200 \\
    2019 & 18300 \\
}\publicationdata

The research and advancement of scientific knowledge can be very daunting, especially without a scientific background. Requirements such as a well defined methodology, high data quality and a strict revision process may very well present some barriers to a common citizen. However, with the recent advancement in technology (Newman et al. 2012), connectivity (Newman et al. 2012) and interest in science by the public (Silvertown 2009), there has been a rising trend in the general populace contributing to a new kind of science: Citizen Science (Mkinley).

Citizen Science refers to research that engages nonprofessionals in the process of creating new scientific knowledge (Bonney et al. 2014). This area of interest is not entirely new, as it has existed for more than one century now. However, at that time, its reach was reduced, since it was mostly focused on ecological and environmental sciences. Some initiatives date back to 1980, such as the Cooperative Weather Service; and 1966, the North American Breeding Bird Survey with more than 670 publications referencing it. This practice has shown a rise in (1) the number of citizen science projects \footnote{Extraction from \href{citizenscience.gov}{citizenscience.gov} and Zooniverse.} (figure \ref{fig:growth-citizen-science-projects}) as well as (2) articles published on the topic \footnote{Query String: "citizen science" in Scholar Google, filtered by one year intervals} (as seen in figure \ref{fig:growth-publications}).

\begin{figure}[!h]
    \centering
    \begin{tikzpicture}
        \begin{axis}[
                ybar,
                bar width=.5cm,
                width=\textwidth,
                height=.5\textwidth,
                legend style={at={(0, 10)}, anchor=north,legend columns=-1},
                symbolic x coords={2010,2011,2012,2013,2014,2015,2016,2017,2018,2019,2020},
                xtick=data,
                nodes near coords,
                nodes near coords align={vertical},
                ymin=0,ymax=25000,
                ylabel={publications},
                xlabel={years},
            ]
            \addplot table[x=year,y=publications]{\publicationdata};
        \end{axis}
    \end{tikzpicture}
    \caption{Growth of citizen science initiatives over time}
    \label{fig:growth-citizen-science-projects}
\end{figure}

\begin{figure}[!h]
    \centering
    \begin{tikzpicture}
        \begin{axis}[
                ybar,
                bar width=.5cm,
                width=\textwidth,
                height=.5\textwidth,
                legend style={at={(0, 10)}, anchor=north,legend columns=-1},
                symbolic x coords={2010,2011,2012,2013,2014,2015,2016,2017,2018,2019,2020},
                xtick=data,
                nodes near coords,
                nodes near coords align={vertical},
                ymin=0,ymax=25000,
                ylabel={publications},
                xlabel={years},
            ]
            \addplot table[x=year,y=publications]{\publicationdata};
        \end{axis}
    \end{tikzpicture}
    \caption{Growth of published peer-reviewed articles on citizen science}
    \label{fig:growth-publications}
\end{figure}

This growth in interest also caused an increase in the variety of citizen science projects. Once focused on ecological and environmental sciences, this practice has a much larger range now, covering topics such as linguistics \cite{svendsen2018dynamics}, astronomy \cite{marshall2015ideas}, hydrology \cite{buytaert2014citizen} etc. This work is focused on a specific area of citizen science: natural language processing.
Out of these interest areas, one that could see further development with initiatives using citizen science is natural language processing.

Natural language processing is a vast field that explores how computers can understand and manipulate human language in text or speech format. Researches in this area include (but are not limited to) sentiment analysis, sentence prediction, text translation, text-to-speech conversion and voice recognition. Voice recognition in particular is a class of machine learning that can be stochastically modelled - using Hidden Markov Models \cite{gales2008application} -, Neural Networks \cite{graves2013speech} or even non-stochastically \cite{burget2003nonrandomattr}.

However, these voice modeling strategies are highly dependant on the quality and quantity of data provided. Factors such as noisy speech data, non-homogeneous recordings, different microphones within the same data set and even speech disorders could limit proper analysis, affect accuracy and even change speech predictions. 

There are many robust data sets available in the literature, such as TIMIT \cite{Lamel1992timmit}, DIRHA \cite{Ravanelli2016dirha} and the more recent \cite{chanchaochai2018globaltimit}. These data sets are called Speech Corpora and have a collection of audio recordings of spoken language. Some of them also have additional text files containing transcriptions of the words spoken. Unfortunately, most speech corpora are for the English language \cite{LeRouxVincent2014TRdatasets} and they 

Citizen Science potential could be used

This work will focus on how to create a speech corpus for the Brazilian Portuguese language using citizen science.

To fill this gap, this work presents a systematic literature review of speech corpora to define its characteristics and requirements when creating this kind of data set.

\chapter{Objective}

The main focus of this article lies in defining what characterizes a good and robust speech dataset.

The secondary focus of this work applies what was defined as a robust dataset to generate one in Brazilian Portuguese, uploading it to a public repository. 

\chapter{Background}

This work has many areas of knowledge

\section{Citizen Science}

\section{Natural Language Processing}

\section{Speech Corpus}

\section{Gamification}
