\chapter*[Introduction]{Introduction}
\addcontentsline{toc}{chapter}{Introduction}

\pgfplotstableread[row sep=\\,col sep=&]{
    year & publications \\
    2010 & 2810 \\
    2011 & 2750 \\
    2012 & 5220 \\
    2013 & 6990 \\
    2014 & 9330 \\
    2015 & 11900 \\
    2016 & 14500 \\
    2017 & 15800 \\
    2018 & 20200 \\
    2019 & 18300 \\
}\publicationdata

The research and advancement of scientific knowledge can be very daunting, especially without a scientific background. Requirements such as a well defined methodology, high data quality and a strict revision process may very well present some barriers to a common citizen. However, with the recent advancement in technology (Newman et al. 2012), connectivity (Newman et al. 2012) and interest in science by the public (Silvertown 2009), there has been a rising trend in the general populace contributing to a new kind of science: Citizen Science (Mkinley).

Citizen Science refers to research that engages nonprofessionals in the process of creating new scientific knowledge (Bonney et al. 2014). This area of interest is not entirely new, as it has existed for more than one century now. However, at that time, its reach was reduced, since it was mostly focused on ecological and environmental sciences. Some initiatives date back to 1980, such as the Cooperative Weather Service; and 1966, the North American Breeding Bird Survey with more than 670 publications referencing it. This practice has shown a rise in (1) the number of citizen science projects \footnote{Extraction from \href{citizenscience.gov}{citizenscience.gov} and Zooniverse.} (figure \ref{fig:growth-citizen-science-projects}) as well as (2) articles published on the topic \footnote{Query String: "citizen science" in Scholar Google, filtered by one year intervals} (as seen in figure \ref{fig:growth-publications}).

\begin{figure}[!h]
    \centering
    \begin{tikzpicture}
        \begin{axis}[
                ybar,
                bar width=.5cm,
                width=\textwidth,
                height=.5\textwidth,
                legend style={at={(0, 10)}, anchor=north,legend columns=-1},
                symbolic x coords={2010,2011,2012,2013,2014,2015,2016,2017,2018,2019,2020},
                xtick=data,
                nodes near coords,
                nodes near coords align={vertical},
                ymin=0,ymax=25000,
                ylabel={publications},
                xlabel={years},
            ]
            \addplot table[x=year,y=publications]{\publicationdata};
        \end{axis}
    \end{tikzpicture}
    \caption{Growth of citizen science initiatives over time}
    \label{fig:growth-citizen-science-projects}
\end{figure}

\begin{figure}[!h]
    \centering
    \begin{tikzpicture}
        \begin{axis}[
                ybar,
                bar width=.5cm,
                width=\textwidth,
                height=.5\textwidth,
                legend style={at={(0, 10)}, anchor=north,legend columns=-1},
                symbolic x coords={2010,2011,2012,2013,2014,2015,2016,2017,2018,2019,2020},
                xtick=data,
                nodes near coords,
                nodes near coords align={vertical},
                ymin=0,ymax=25000,
                ylabel={publications},
                xlabel={years},
            ]
            \addplot table[x=year,y=publications]{\publicationdata};
        \end{axis}
    \end{tikzpicture}
    \caption{Growth of published peer-reviewed articles on citizen science}
    \label{fig:growth-publications}
\end{figure}

This growth in interest also caused an increase in the variety of citizen science projects. Once focused on ecological and environmental sciences, this practice has a much larger range now, covering topics such as linguistics \cite{svendsen2018dynamics}, astronomy \cite{marshall2015ideas}, hydrology \cite{buytaert2014citizen} etc. One topic in which  

Zooniverse, a platform in which scientists include  citizen science projects and ask for the help of the public to collect and classify data.

As to some common features of citizen science practice: (a) anyone can participate, (b) participants use the same protocol so data can be combined and be high quality, (c) data can help real scientists come to real conclusions and (d) a wide community of scientists and volunteers work together and share data to which the public, as well as scientists, have access. 

To advance our knowledge of science, 
For scientists, the process of data gathei
With the recent growth of citizen science (cite mckinley), 
For many science areas, it is unavoidable gathering and analyzing large amounts of data to research a subject. 
General participation, same protocol, real contributions and public sharing.
In ecology and biology, this work is conventionally done by nonprofessionals.
To ensure these requirements, scientists often collaborate with normal citizens.
In conventional scenarios, a strict method is used to extract, process, and analyze the data.

Contribution - something that you contribute or do to help produce or achieve something together with other people, or to help make something successful \cite{TCDP2020}. 
When nonprofessionals engage in the process of creating new scientific knowledge - contributing to science -, 

To contribute to science, nonprofessionals have 

How do you contribute to science?
When done correctly, simple tasks such as data gathering and labeling can be done by
Citizen science is 


Citizen science – research that engages non- professionals in the process of creating new scientific knowledge (Bonney et al. 2014) – has expanded greatly in the past decade (Figure 1; McKinley et al. 2015). Rising interest in this approach has been fueled in part by rapid technological developments (Newman et al. 2012), by policy and management needs for large- scale and long-term monitoring datasets (Conrad and Hilchey 2011), and by increased emphasis on science outreach and education (Silvertown 2009). While citizen-science projects vary widely in their subject mat- ter, objectives, activities, and scale (Figures 2–4; Wiggins and Crowston 2015), one common goal is the produc- tion of reliable data that can be used for scientific purposes.

The ecological and environmental sciences have been leaders in citizen science, boasting some of the longest- running projects that have contributed meaningful data to science and conservation, including the Cooperative Weather Service (first year of data collection: 1890), the National Audubon Society’s Christmas Bird Count (1900; >200 publications have relied on the resulting data set), the North American Breeding Bird Survey (1966; >670 publications), the leafing and flowering times of US lilacs and honeysuckles (1956; >50 publications; Rosemartin et al. 2015), and the Butterfly Monitoring Scheme (1976; >100 publications). These and other successful citizen science projects have increased ecological and environmental knowledge at large geographic scales and at high temporal resolution (McKinley et al. 2015).

A citizen scientist is a volunteer who collects and/or processes data as part of a scientific enquiry. Projects that involve citizen scientists are burgeoning, particularly in ecology and environmental sciences, although the roots of citizen science go back to the very beginnings of modern science itself.

“Citizen science,” a range of participatory models for involving nonprofessionals as collaborators in scientific research, is receiving increased attention for its ongoing successes. Citizen science is well developed in fields like ecology and astronomy, where there is a long history of nonprofessionals making substantive contributions to the scholarly knowledge base. Western health and biomed- ical research have historically had a different relationship to nonprofessionals (as subjects or service recipients), however, with protective regulations and oversight further constraining interactions between parties. 

He is not alone. Citizen science is now mainstream, with hundreds of thousands of participants worldwide contributing observations of the natural world to various citizen science projects daily. Among the most popular projects, accumulating millions of observations annually, are those in which citizen scientists record the numbers and types of organisms observed [1,2].

For many fields of science, it is unavoidable gathering and analyzing large amounts of data to research a subject. To ensure these requirements, scientists often collaborate with normal citizens. 
In conventional scenarios, a strict method is used to extract, process and analyze the data.

Natural language processing is a vast field that explores how computers can understand and manipulate human language in text or speech format. Researches in this area include (but are not limited to) sentiment analysis, sentence prediction, text translation, text-to-speech conversion and voice recognition. Voice recognition in particular is a class of machine learning that can be stochastically modelled - using Hidden Markov Models \cite{gales2008application} -, Neural Networks \cite{graves2013speech} or even non-stochastically \cite{burget2003nonrandomattr}.

However, these voice modeling strategies are highly dependant on the quality and quantity of data provided. Factors such as noisy speech data, non-homogeneous recordings, different microphones within the same data set and even speech disorders could limit proper analysis, affect accuracy and even change speech predictions. 

However, the applicability of these strategies is either compromised or ameliorated by the quality and quantity of data used in training stages. Factors such as noisy speech data, non-homogeneous recordings, different microphones within the same data set and even speech disorders could limit proper analysis, affect accuracy and even change speech predictions.

There are many robust data sets available in the literature, such as TIMIT \cite{Lamel1992timmit}, DIRHA \cite{Ravanelli2016dirha} and the more recent \cite{chanchaochai2018globaltimit}. These data sets are called Speech Corpora and have a collection of audio recordings of spoken language. Some of them also have additional text files containing transcriptions of the words spoken. Unfortunately, most speech corpora are for the English language \cite{LeRouxVincent2014TRdatasets} and the literature on the Speech Corpora creation itself is sparse.

This work focuses on 

To fill this gap, this work presents a systematic literature review of speech corpora to define its characteristics and requirements when creating this kind of data set.

\chapter{Objective}

The main focus of this article lies in defining what characterizes a good and robust speech dataset.

The secondary focus of this work applies what was defined as a robust dataset to generate one in Brazilian Portuguese, uploading it to a public repository. 

\chapter{Related Work}



\chapter{Background}

This work has many  