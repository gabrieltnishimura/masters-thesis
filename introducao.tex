\chapter*[Introduction]{Introduction}
\addcontentsline{toc}{chapter}{Introduction}

The research and advancement of scientific knowledge can be very daunting, especially without a scientific background. Requirements such as a well defined methodology, high data quality and a strict revision process may very well present some barriers to a common citizen. However, with the recent advancement in technology \cite{newman2012future}, connectivity \cite{newman2012future} and interest in science by the public \cite{silvertown2009new}, there has been a rising trend in the general populace contributing to a new kind of science: Citizen Science \cite{mckinley2017citizen}.

Citizen science refers to research that engages nonprofessionals in the process of creating new scientific knowledge \cite{bonney2014next}. Referred to as citizen scientists; these nonprofessionals may participate in a variety of tasks ranging in complexity; from simple tasks such as data gathering or classification \cite{barker2013pascal}, to even complex ones such as solving algorithms \cite{cooper2010predicting}. They may act as contributors and collaborators, but can also have a more proactive role as a project leader \cite{robinson2018ten}. 

Citizen science could be classified into three categories, based on volunteer involvement \cite{follett2015analysis}: (1) Contributory, where participants contribute to data collection and sometimes help analyze and disseminate results \cite{bonney2009citizen}, (2) Collaborative, where citizens also analyze samples, design the study, interpret the data, draw conclusions and disseminate results \cite{faridani2009networked} and (3) Co-created, where they participate in all stages of the project, including defining questions, developing the hypotheses, drawing conclusions, discussing results and answering new questions \cite{hill2012notes}.

As to why would a citizen collaborate, there could be several reasons depending on the project itself: contribution to the advancement of science or the project, desire to learn, personal interests, entertainment, among others \cite{tinati2016because}. This poses a engagement challenge, since public participation is vital to the result of the research; and is one of the studies of citizen science theory \cite{bowser2013using}.

At the other end of the spectrum, scientists lead the citizen science initiatives. They enlist amateurs to contribute to projects, but also devise validation techniques to the collected data. They are also central to the role of acquiring sponsors in the government scenario, many of which are also important to the feasibility of the initiatives.

With the rise in interest from the public, this discipline is able to achieve otherwise impossible results. According to \cite{theobald2015global}, in 2014, 1.3 million volunteers participated in 388 research projects related to biodiversity alone, contributing up to \$2.5 billion of in-kind labor annually.

This is just one of the many examples showing the potential of Citizen Science. It also comes with the realization that the public represents a free source of labor, skills, computational power and even finance \cite{silvertown2009new}. Such perception could implicate in ethical concerns depending on how the data is gathered or processed, but also on how should the nonprofessional be (if he is) rewarded. With these among other concerns, the European Citizen Science Association conceived the "Ten principles of citizen science" \cite{robinson2018ten}. They establish some key principles to follow as good practice when applying citizen science concepts to research. Many of the principles focus on the protection of the citizen scientists rights, since those individuals deserve feedback, acknowledgment and participation rights to the research.

% what about computer automation
% potential of citizen science

Citizen science is not new, as it has existed for more than one century now. However, its reach was reduced, since it was mostly focused on ecological and environmental sciences. Some initiatives date back to 1890, such as the Cooperative Weather Service, where amateurs send collected weather data to the National Weather Service; and 1966, the North American Breeding Bird Survey with more than 670 publications referencing it, where nonprofessionals map avian species distribution throughout North America over time. This practice has shown a rise in (1) the number of citizen science projects \footnote{Extraction from \href{citizenscience.gov}{citizenscience.gov} and Zooniverse.} (figure \ref{fig:growth-citizen-science-projects}) as well as (2) articles published on the topic \footnote{Query String: "citizen science" in Scholar Google, filtered by one year intervals} (as seen in figure \ref{fig:growth-publications}).

One significant factor justifying this growth is the availability of technical tools for disseminating information about projects and gathering data from the public \cite{silvertown2009new}. The widespread use of smartphones allows scientists to develop platforms easily accessible to everyone. This is combined with advancements in software usability on these platforms, supporting first time users and recurring contributors.
%Low-cost hardware has also supported this growth, with 
%more is more

Many of the citizen science initiatives are practiced online and do not require specialized hardware, allowing people with access to a computer or smartphone to help. This allows for more participation in the projects, but highlights the engagement problems. To mitigate this, scientists use gamification, adding game elements to enhance user experience and engagement with non-game applications. Foldit, designed by researchers at University of Washington, is a game in which gamers solve protein folding patterns, a central challenge in biochemistry, by virtually wiggling, shaking and pulling shapes to create small stable structures, as well as developing their own algorithms for solving protein folding \cite{bourzac2008enlisting}.

\pgfplotstableread[row sep=\\,col sep=&]{
    year & publications \\
    2010 & 2810 \\
    2011 & 2750 \\
    2012 & 5220 \\
    2013 & 6990 \\
    2014 & 9330 \\
    2015 & 11900 \\
    2016 & 14500 \\
    2017 & 15800 \\
    2018 & 20200 \\
    2019 & 18300 \\
}\publicationdata

\begin{figure}[!h]
    \centering
    \begin{tikzpicture}
        \begin{axis}[
                ybar,
                bar width=.5cm,
                width=\textwidth,
                height=.5\textwidth,
                legend style={at={(0, 10)}, anchor=north,legend columns=-1},
                symbolic x coords={2010,2011,2012,2013,2014,2015,2016,2017,2018,2019,2020},
                xtick=data,
                nodes near coords,
                nodes near coords align={vertical},
                ymin=0,ymax=25000,
                ylabel={publications},
                xlabel={years},
            ]
            \addplot table[x=year,y=publications]{\publicationdata};
        \end{axis}
    \end{tikzpicture}
    \caption{Growth of citizen science initiatives over time}
    \label{fig:growth-citizen-science-projects}
\end{figure}

\begin{figure}[!h]
    \centering
    \begin{tikzpicture}
        \begin{axis}[
                ybar,
                bar width=.5cm,
                width=\textwidth,
                height=.5\textwidth,
                legend style={at={(0, 10)}, anchor=north,legend columns=-1},
                symbolic x coords={2010,2011,2012,2013,2014,2015,2016,2017,2018,2019,2020},
                xtick=data,
                nodes near coords,
                nodes near coords align={vertical},
                ymin=0,ymax=25000,
                ylabel={publications},
                xlabel={years},
            ]
            \addplot table[x=year,y=publications]{\publicationdata};
        \end{axis}
    \end{tikzpicture}
    \caption{Growth of published peer-reviewed articles on citizen science}
    \label{fig:growth-publications}
\end{figure}

This growth in interest also caused an increase in the variety of citizen science projects. Once focused on ecological and environmental sciences, this practice has a much larger range now, covering topics such as linguistics \cite{svendsen2018dynamics}, astronomy \cite{marshall2015ideas}, hydrology \cite{buytaert2014citizen} etc.
% detail more the advancement in variety (graphs?)
% talk crowdsourcing, online and gamification
Out of these interest areas, one that could see further development with initiatives using citizen science is natural language processing.

Natural language processing is a vast field that explores how computers can understand and manipulate human language in text or speech format. Researches in this area include (but are not limited to) sentiment analysis, sentence prediction, text translation, text-to-speech conversion and voice recognition. Voice recognition in particular is a class of machine learning that can be stochastically modelled - using Hidden Markov Models \cite{gales2008application} -, Neural Networks \cite{graves2013speech} or even non-stochastically \cite{burget2003nonrandomattr}.

% add more emphasys on STT
However, these voice modeling strategies are highly dependant on the quality and quantity of data provided. Factors such as noisy speech data, nonhomogeneous recordings, different microphones within the same dataset and even speech disorders could limit proper analysis, affect accuracy and even change speech predictions \cite{wrong}. 

%These voice datasets, also known as Speech Corpus (or Speech Corpora in plural), curate a collection of audio recordings of a spoken language. Some of them also have additional text files with transcriptions of the words spoken.

%Although they are widely found in the literature

There are many robust datasets available in the literature, such as TIMIT \cite{Lamel1992timmit}, DIRHA \cite{Ravanelli2016dirha} and the more recent \cite{chanchaochai2018globaltimit}. These data sets are called Speech Corpora and have a collection of audio recordings of spoken language. Some of them also have additional text files containing transcriptions of the words spoken.
%talk corpus construction here
Unfortunately, most speech corpora are for the English language \cite{LeRouxVincent2014TRdatasets} and they have no formality as to how the corpus was created.
% talk formality here

There are also online datasets that could be constantly updated with user input. These are Vox Forge \cite{asd} and Common Voice \cite{mozilla}. But the former has poor user interface and little usage over time and the latter has no formal analysis whether the data could be used as a voice recognition system.

This work identifies the Citizen Science practice as a potential candidate to create a robust Speech Corpus.
% talk robustness here
This work will focus the creation of a Speech Corpus for the Brazilian Portuguese language harnessing the potential use of the Citizen Science practice.

To fill this gap, this work presents a systematic literature review of speech corpora to define its characteristics and requirements when creating this kind of data set.

\chapter{Objective}

The main focus of this article lies in defining what characterizes a good and robust speech dataset.

The secondary focus of this work applies what was defined as a robust dataset to generate one in Brazilian Portuguese, uploading it to a public repository. 

\chapter{Background}

This work has many areas of knowledge

\section{Citizen Science}

\subsection{Online}
\subsection{Gamification}
\subsection{Crowdfunding}

\section{Natural Language Processing}

\subsection{}

\section{Speech Corpus}
