\chapter[Related Work]{Related Work}

In this section, we discuss how the literature has treated speech corpora creation, as well as the various conditions and variables considered in the process. Speech Corpus crafting itself is well established in the literature by TIMIT \cite{Lamel1992timmit} and SWITCHBOARD \cite{godfrey1992switchboard}. TIMIT creates a dataset of 6300 utterances by 630 speakers from different regions of the United States. The sentences were crafted to fit in one of the three categories: 1) dialect "shibboleth", 2) phonemically compact, and 3) phonetically diverse, but the selection itself was not well defined. Nevertheless, it is a very robust dataset with a time-aligned transcription and a usage guide to automatic speech recognition applications.

The CHiME articles \cite{christensen2010chime} \cite{barker2013pascal}, \cite{barker2018fifth} (and more), are also source of structured speech corpora creation, challenging researchers to better recognize speech within a everyday listening environment using multiple distant microphones. Since the focus of these works lies on nonoptimal recording conditions, detailed information on the noise background, noise level, recording style and speech material has been provided, as well as comprehensive postprocessing work.

A more recent work by \cite{chanchaochai2018globaltimit} attempts to extend the TIMIT functionality to other languages, by providing a method to create "TIMIT-like" datasets. These datasets are caracterized by having 1) Multiple (anonymously) identified speakers, 2) Wide range of phonetically representative inputs, 3) Wideband recordings with good acoustic quality, 4) Time-aligned lexical and phonemic transcripts and 5) Easily availability to anyone. The authors detail the speakers and sessions, the text corpus selection process, the recording procedures, as well as the transcription and alignment methods. At the moment, there have been five datasets created, with more planned or in progress.

As for crowdfunding speech corpora, VoxForge \cite{voxforge} is a viable alternative to contribute speech. Besides being a free speech corpus repository under GPL licence, VoxForge also incorporates an acoustic model archive for open source speech recognition engines. This corpus has over 132h, 39h, 20h, 57h of recorded English, French, Italian, and German \footnote{As of January 20th of 2021, per metric from \url{http://www.voxforge.org/en/Downloads}, \url{http://www.voxforge.org/fr/Downloads}, \url{http://www.voxforge.org/de/Downloads}, and \url{http://www.voxforge.org/de/Downloads}, respectively.}, as well a variety of acoustic models for CMU Sphinx, Julius and HTK. The VoxForge platform presents itself as an alternative to closed source speech recognition engines, since many corpora must be purchased under restrictive licenses. This contribution tool has, however, a very limited user interface and a not a well-defined recording validation process, which could invalidate the corpus robustness.

Common Voice \cite{ardila2019common} is a more recent platform for recording and listening to speech. It is a multilingual speech corpus, containing, as of 11th of December of 2020\footnote{Common Voice Corpus 6.1 accessible on  \url{https://commonvoice.mozilla.org/en/datasets}}, 1,686h of validated speech in English, and over 50h of validated Portuguese speech, in addition to other 27 languages. The main feature of this platform is the crowdsourced validation process, which all recordings must go through so that they are included in the validated corpus. All speech data is free and available under CC-0, the most permissive of licenses, and some recordings contain speaker demographics. One caveat though, is the lack of statistical analysis of the validated data, to ensure it is able to provide a quality dataset for speech recognition services. Nevertheless, the corpus was empirically tested through Mozilla’s DeepSpeech Speech-to-Text, achieving an average Character Error Rate improvement of 5.99 ± 5.48 for twelve target languages (German, French, Italian, Turkish, Catalan, Slovenian, Welsh, Irish, Breton, Tatar, Chuvash, and Kabyle).