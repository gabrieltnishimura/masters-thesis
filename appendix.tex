\apendices 
\chapter{Ten principles of citizen science}
\label{app:ten-principles}

\begin{itemize}
    \item Citizen science projects actively involve citizens in scientific endeavour that generates new knowledge or understanding. 
    \subitem Citizens may act as contributors, collaborators or as project leaders and have a meaningful role in the project.
    \item Citizen science projects have a genuine science outcome. For example, answering a research question or informing conservation action, management decisions or environmental policy.
    \item Both the professional scientists and the citizen scientists benefit from taking part. Benefits may include the publication of research outputs, learning opportunities, personal enjoyment, social benefits, satisfaction through contributing to scientific evidence, for example, to address local, national and international issues, and through that, the potential to influence policy.
    \item Citizen scientists may, if they wish, participate in multiple stages of the scientific process. This may include developing the research question, designing the method, gathering and analysing data, and communicating the results.
    \item Citizen scientists receive feedback from the project. For example, how their data are being used and what the research, policy or societal outcomes are.
    \item Citizen science is considered a research approach like any other, with limitations and biases that should be considered and controlled for. However unlike traditional research approaches, citizen science provides opportunity for greater public engagement and democratisation of science.
    \item Citizen science project data and metadata are made publicly available and where possible, results are published in an open-access format. Data sharing may occur during or after the project, unless there are security or privacy concerns that prevent this.
    \item Citizen scientists are acknowledged in project results and publications.
    \item Citizen science programmes are evaluated for their scientific output, data quality, participant experience and wider societal or policy impact.
    \item The leaders of citizen science projects take into consideration legal and ethical issues surrounding copyright, intellectual property, data-sharing agreements, confidentiality, attribution and the environmental impact of any activities.
\end{itemize}

\chapter{Software Requirements Specification}
\label{appendix:srs}

\section{Introduction}
\subsection{Purpose}

The purpose of this Requirements Specification is to document the software requirements of Fale Alguma Coisa, a citizen science voice recording app. It should be used by anyone with the intent of understanding what are the capabilities of the system. This document will not focus on how the system will be built, but serves as the basis for the design and implementation. The design is documented within the Design Model in appendix \ref{appendix:design-model}. Details on implementation are documented in the corresponding Software Design Description (SDD) document in the appendix \ref{appendix:software-design-description}.

\subsection{Scope}

\subsubsection{Objectives}

Towards contributing \textbf{nonprofessional scientists}, the Fale Alguma Coisa app should provide an easy gateway for the user to contribute his voice while having fun and learning various science facts and curiosities.

Towards researching \textbf{scientists}, the Fale Alguma Coisa app should provide a database of anonymized voice recordings, for scientists to extract and create speech corpus.

\subsubsection{Outside scope}

However, some elements are outside of the scope of this system. Fale Alguma Coisa should \textbf{not}:
\begin{itemize}
    \item Allow for association of recording data and personal identification data (name, email);
    \item Support internationalization in the WebApp;
    \item Convert audio data into another format
    \item Support offline recording
\end{itemize}

\subsubsection{Artifacts}

This document identifies the following artifacts to be produced:

\begin{itemize}
    \item Context
    \item Domain Model
    \item Use Case Requirements (Actors, Diagrams, Specifications)
    \item Activity Diagrams
    \item Non-Functional Requirements
    \item Interface Requirements
    \item WebApp for user recordings
    \item Backend with database
\end{itemize}

\subsection{Overview of Document}

This overview provides a summary of the contents of each section of this document.

\begin{itemize}
    \item Introduction
    \item System Overview
    \item Domain Model
    \item Requirements
\end{itemize}

\subsection{References}

\begin{itemize}
    \item Based on: \cite{naoufel2014requirements}
    \item Supporting literature: \cite{pressman2014software}
\end{itemize}

\section{System Overview}

The System Overview section presents the system context and design, and also discusses the background of the Fale Alguma Coisa project.

\subsection{Name}

This system is called "Fale Alguma Coisa". This is a phrase in brazillian portuguese to order someone to "say something". It is also common to hear "fala alguma coisa", but this was changed to remove the alliteration effect, providing easier spelling.

\subsection{System Context}

This system is originated from the need of recording anonymized user voices throughout the use of gamification and citizen science.

\subsection{General Constraints}

This system is mainly impacted by constraints in data privacy. Recordings should never be associated with the users personal data (such as email), but should have metadata to aid in corpus construction, such as age interval, gender, and dialect.

\subsection{Assumptions and Dependencies}

The system heavily depends on the phrases that will be spoken. Therefore it is necessary to provide this application with a comprehensive list of phrases before the application general publication.

\section{Analysis Model}

To accurately describe the software requirements of a project, it is important to understand the requirements from many different perspectives. The analysis model proposed by \cite{pressman2014software} encompasses four different modeling elements (perspectives) that can be used to provide the most effective bridge to software design. They are listed in the subsections below as (1) scenario based models, (2) class-based elements model, (3) behavioral elements, and (4) flow-oriented elements. The analysis can lead to the derivation of one or more of these modeling elements, since one model may not completely describe certain functionalities of the app.

\subsection{Scenario-based models}

This model depicts how the user interacts with the system and the specific sequence of activities that occur as the software is used. One common implementation for this specification is through user stories. To provide scenario-model completeness, the table at \ref{fig:eyewire-game-interface} describes every scenario mapped of the WebApp. They are, however, not extensive to the functionality of the application, complemented by the remaining elements. Each of the scenarios are categorized by epic. Furthermore, each epic is described in the table \ref{tab:falealgumacoisa-epics}.

\begin{table}[h]
    \centering
    \caption{WebApp Fale Alguma Coisa Epics}
    \label{tab:falealgumacoisa-epics}
    \begin{tabular}{|p{3cm}|p{10cm}|}
        \hline Epic & Description \\
        \hline Points & This epic represents the user stories affecting the homepage, such as the splash screen, call to action button, terms of service, etc. \\
        \hline Recording & The recording epic represents the most important feature in the application, as the citizen will use it to record his voice. It also provides supporting features, such as skipping phrases and resuming the recording session.\\ 
        \hline Dashboard & The dashboard epic lists all user stories related to the dashboard page, such as where the user will be able to choose themes to record, open the menu, check his level, etc. \\
        \hline Gamification & The gamification epic details the engagement component of the application. Elements such as leaderboards, points and levels are described. They add a sense of progress to the user experience. \\
        \hline Social & To allow social interaction with other users, this epic lists social features to be added to the application. \\
        \hline Login and Registration & This application should provide user authentication to enable data management and progress saving. \\
        \hline
    \end{tabular}
    \caption*{Source: Author}
\end{table}

\begin{table}[h]
\caption{User Stories categorized to the home epic for the Fale Alguma Coisa WebApp}
\label{tab:falealgumacoisa-home-epic}
\centering
\begin{tabular}{|p{2cm}|p{1cm}|p{10cm}|}
\hline
Epic & Code & User Story \\ \hline
\multirow{5}{*}{Home} 
    & US01 & I, as an unregistered citizen, would like to view an animated introductory screen in the application, so that I feel more inside a native app. \\ \cline{2-3} 
    & US02 & I, as an unregistered citizen, would like to know more about Fale Alguma Coisa from a explanatory text in the homepage, so that I understand more about the project. \\ \cline{2-3} 
    & US03 & I, as a registered citizen, would like to easily go from the homepage to the sign-in page, so that I can login to my account. \\ \cline{2-3} 
    & US04 & I, as an unregistered citizen, would like to click a call to action button to go from the homepage to the recording page from the homepage, so that I can contribute my voice. \\ \cline{2-3} 
    & US05 & I, as an unregistered citizen, would like to read the terms of service and privacy policy of Fale Alguma Coisa, so that I understand better what the service has to offer and what kind of data will be recorded. \\ \hline
\end{tabular}
\caption*{Source: Author}
\end{table}

\begin{table}[h]
\caption{User Stories categorized to the recording epic for the Fale Alguma Coisa WebApp}
\label{tab:falealgumacoisa-recording-epic}
\centering
\begin{tabular}{|p{2cm}|p{1cm}|p{10cm}|}
\hline
Epic & Code & User Story \\ \hline
\multirow{9}{*}{Recording} 
    & US10 & I, as an unregistered citizen, would like to view and accept the terms of service before recording, so that I understand how my data is being used. \\ \cline{2-3} 
    & US11 & I, as an unregistered citizen, would like to properly configure my microphone before recording, so that I can record without interruption. \\ \cline{2-3} 
    & US12 & I, as a citizen, would like to read science phrases with definitions and curiosities, so that I learn about subjects as I am contributing. \\ \cline{2-3} 
    & US13 & I, as a citizen, would like to read phrases grouped by theme, so that I can learn more from each subject as I am contributing. \\ \cline{2-3} 
    & US14 & I, as a citizen, would like to read a tutorial explaining how to record, so that I learn how to properly record phrases. \\ \cline{2-3} 
    & US15 & I, as a citizen, would like to see animations on each step of the recording (enter the page, start the recording, stop the recording), so that I feel more engaged with the application. \\ \cline{2-3} 
    & US16 & I, as a citizen, would like to skip a phrase when I (1) do not know how to pronounce, or (2) find a foreign word, or (3) another specified reason, so that I only speak the correct phrases. \\ \cline{2-3} 
    & US17 & I, as a unregistered citizen, would like to stop this recording session by returning home when clicking the logo and confirming the exit, so that I can resume it afterwards. \\ \cline{2-3} 
    & US18 & I, as a registered citizen, would like to return to the dashboard after clicking the logo and confirming the exit, so that I can resume it afterwards. \\ \hline
\end{tabular}
\caption*{Source: Author}
\end{table}

\begin{table}[h]
\caption{User Stories categorized to the dashboard epic for the Fale Alguma Coisa WebApp}
\label{tab:falealgumacoisa-dashboard-epic}
\centering
\begin{tabular}{|p{2cm}|p{1cm}|p{10cm}|}
\hline
Epic & Code & User Story \\ \hline
\multirow{6}{*}{Dashboard} 
    & US20 & I, as an unregistered citizen, would like to easily register my data through a button click, so that I can enjoy all features of the logged area. \\ \cline{2-3} 
    & US21 & I, as a registered citizen, would like to see my actions in a dashboard after logging in, so that I can better contribute to the project. \\ \cline{2-3} 
    & US22 & I, as a registered citizen, would like to see a list of recommended themes to speak, so that I can choose one from the list. \\ \cline{2-3} 
    & US23 & I, as a registered citizen, would like to view my progress level, so that I know how far have I progressed in my contributions. \\ \cline{2-3} 
    & US24 & I, as a registered citizen, would like to open the menu, so that I know which are my possible actions in the app. \\ \cline{2-3} 
    & US25 & I, as a registered citizen, would like to check notifications, so that I understand what happened while I was gone. \\ \hline
\end{tabular}
\caption*{Source: Author}
\end{table}

\begin{table}[h]
\caption{User Stories categorized to the gamification epic for the Fale Alguma Coisa WebApp}
\label{tab:falealgumacoisa-gamification-epic}
\centering
\begin{tabular}{|p{2.5cm}|p{1cm}|p{10cm}|}
\hline
Epic & Code & User Story \\ \hline
\multirow{6}{*}{Gamification}
    & US30 & I, as a registered citizen, would like to get 100 points when I record my first phrase, so that I can engage better in the application. \\ \cline{2-3} 
    & US31 & I, as a registered citizen, would like to get 400 points when I record my first theme, so that I can engage better in the application. \\ \cline{2-3} 
    & US32 & I, as a registered citizen, would like to get 300 points when I record subsequent themes, so that I can engage better in the application. \\ \cline{2-3} 
    & US33 & I, as a registered citizen, would like to get 500 points when I register my data, so that I can better engage with the application. \\ \cline{2-3} 
    & US34 & I, as a registered citizen, would like to measure my points through a level, so that I can more easily compare myself with other users. \\ \hline
\end{tabular}
\caption*{Source: Author}
\end{table}

\begin{table}[h]
\caption{User Stories categorized to the social epic for the Fale Alguma Coisa WebApp}
\label{tab:falealgumacoisa-social-epic}
\centering
\begin{tabular}{|p{2cm}|p{1cm}|p{10cm}|}
\hline
Epic & Code & User Story \\ \hline
\multirow{6}{*}{Social}
    & US40 & I, as a registered citizen, would like to know who are the top contributors in the space and where am I in the list, so that I can compete against them. \\ \cline{2-3} 
    & US41 & I, as a registered citizen, would like to know where my friends are in a more customized leaderboard, so that I can compete against them. \\ \cline{2-3} 
    & US42 & I, as a registered citizen, would like to add a friend, so that I can check them in the friends leaderboard afterwards. \\ \cline{2-3} 
    & US43 & I, as a registered citizen, would like to check notifications, so that I can know what happened when I was away. \\ \cline{2-3} 
    & US44 & I, as a registered citizen, would like to know when someone added me through notifications, so that I can add them back later. \\ \cline{2-3} 
    & US45 & I, as a registered citizen, would like to refer friends to the application, so that I can play with them. \\ \hline
\end{tabular}
\caption*{Source: Author}
\end{table}

\begin{table}[h]
\caption{User Stories categorized to the login and registration epic for the Fale Alguma Coisa WebApp}
\label{tab:falealgumacoisa-login-and-registration-epic}
\centering
\begin{tabular}{|p{3cm}|p{1cm}|p{10cm}|}
\hline
Epic & Code & User Story \\ \hline
\multirow{4}{*}{\shortstack[l]{Login and \\ Registration}} 
    & US50 & I, as an unregistered citizen, would like to login (using social login - Facebook / Google) on the app, so that I can login later and save my progress. \\ \cline{2-3} 
    & US51 & I, as a unregistered citizen, would like to register my anonymous data on my first login, so that I can provide better metadata to my recordings afterwards. \\ \cline{2-3} 
    & US52 & I, as a registered citizen, would like to update my account data, so that I can provide accurate metadata on the recordings. \\ \cline{2-3} 
    & US53 & I, as a registered citizen, would like to remove my account data (and my recordings, if necessary), so that I can remove my metadata from this application. \\ \hline
\end{tabular}
\caption*{Source: Author}
\end{table}

\subsection{Class models}

\subsection{Behavioral models}

\subsection{Flow models}

 
